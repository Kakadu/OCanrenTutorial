\section{Conjunctive Partial Deduction}
\label{sec:cpd}
Specialization~\cite{jones1993partial} is a natural way to tackle the problem of redundant computations when a part of the input is known. 
A fully-automatic specialization technique developed in the domain of logic programming is called \emph{partial deduction}~\cite{komorowski1982partial, lloyd1991partial}. 
It is related to the supercompilation of functional languages~\cite{gluck1994partial, turchin1986concept}. 
The particular flavour of the partial deduction we are interested in is called \emph{conjunctive partial deduction}~\cite{de1999conjunctive}.
As opposed to the partial deduction, conjunctive partial deduction handles  conjunctions of atoms, thus being able to perform such optimizations as tupling~\cite{hu1997tupling} and deforestation~\cite{wadler1988deforestation}.
Below we demonstrate by example the features of conjunctive partial deduction.

\emph{Deforestation} is a program transformation which gets rid of intermediate data structures. 
The following example demonstrates deforestation. 
Consider a goal ``\lstinline{append$^o$ xs ys ts /\ append$^o$ ts zs rs}'' (note the shared ``\lstinline{ts}''), where ``\lstinline{append$^o$ x y xy}'' describes concatenation, ``\lstinline{nil ()}'' constructs the empty list, and ``\lstinline{h % t}'' constructs a new list from the value ``\lstinline{h}'' and another list ``\lstinline{t}'' (similarly to ``\lstinline{cons}'' in \textsc{Scheme} and ``\lstinline{::}'' in \textsc{OCaml}).

\begin{lstlisting}[label={cpd:appendo}]
let rec append$^o$ x y xy = conde [
  (x === nil () /\ xy === y);
  (fresh (h t ty) (
     (x  === h % t)  /\  
     (xy === h % ty) /\
     (append$^o$ t y ty)))]
\end{lstlisting}

This goal concatenates three lists: ``\lstinline{xs}'', ``\lstinline{ys}'', ``\lstinline{zs}'', constructing an intermediate list ``\lstinline{ts}''. During the execution of this goal, elements of the list ``\lstinline{xs}'' are examined twice: first when ``\lstinline{ts}'' is constructed, and then when the result ``\lstinline{rs}'' is constructed. What is worse, ``\lstinline{ts}'' is only constructed to be immediately deconstructed. Deforestation gets rid of ``\lstinline{ts}'' in this example.  

A better program would be such that does not construct ``\lstinline{ts}'' at all. 
Such a program be generated from the original definition by conjunctive partial deduction and is shown below: 

\begin{lstlisting}[label={cpd:doubleappendo}]
let rec doubleAppend$^o$ xs ys zs rs = conde [
  (xs === nil () /\ append$^o$ ys zs rs);
  (fresh (h t ts) (
     (xs === h % t)  /\  
     (rs === h % ts) /\
     (doubleAppend$^o$ t ys zs ts)))]
\end{lstlisting}


Conjunctive partial deduction is also capable of \emph{tupling}. 
This transformation makes sure that the same data structure is traversed once even if computing several results. 
The following example demonstrates such a case. 

The goal ``\lstinline{maxLength$^o$ xs m l}'' computes both the maximum value of the list ``\lstinline{xs}'' and its length. 
The elements of the list are Peano numbers with ``\lstinline{zero ()}'' as the zero and ``\lstinline{succ}'' as the successor function.
The third argument ``\lstinline{b}'' of the relation ``\lstinline{le$^o$ x y b}'' is ``\lstinline{^true}'' if ``\lstinline{x}'' is less or equal than ``\lstinline{y}'', and ``\lstinline{^false}'' otherwise. The relation ``\lstinline{gt$^o$ x y b}'' is similar to ``\lstinline{le$^o$ x y b}'', but it checks for ``\lstinline{x}'' to be greater than ``\lstinline{y}''. 

\begin{lstlisting}[label={cpd:maxandlength}]
let maxLength$^o$ xs m l = max$^o$ xs m /\ length$^o$ xs l

let rec length$^o$ xs l = conde [
  (xs === nil () /\ l === zero ());
  (fresh (h t m) (
    xs === h % t /\ l === succ m /\ length$^o$ t m))]

let max$^o$ xs m = max$_1^o$ xs (zero ()) m

let rec max$_1^o$ xs n m = conde [
  (xs === nil () /\ m === n);
  (fresh (h t) (
    (xs === h % t) /\
    (conde [
      (le$^o$ h n ^true /\ max$_1^o$ t n m); 
      (gt$^o$ h n ^true /\ max$_1^o$ t h m)])))]

let rec le$^o$ x y b = conde [
  (x === zero () /\ b === ^true); 
  (fresh (x$_1$) (
    x === succ x$_1$ /\ y === zero () /\ b === ^false)); 
  (fresh (x$_1$ y$_1$) (
    x === succ x$_1$ /\ y === succ y$_1$ /\ le$^o$ x$_1$ y$_1$ b))]

let rec gt$^o$ x y b = conde [
  (x === zero () /\ b === ^false);
  (fresh (x$_1$) (
    x === succ x$_1$ /\ y === zero () /\ b === ^false));
  (fresh (x$_1$ y$_1$) (
    x === succ x$_1$ /\ y === succ y$_1$ /\ gt$^o$ x$_1$ y$_1$ b))]
\end{lstlisting}


Execution of the goal ``\lstinline{maxLength$^o$ xs m l}'' leads to ``\lstinline{xs}'' being traversed twice. 
There is a way to rewrite the program so that ``\lstinline{xs}'' is traversed once, but this requires fusing together the definitions of ``\lstinline{length$^o$}'' and ``\lstinline{max$^o$}'', which either restricts code reuse, or leads to code duplication. 
A better way is to only fuse the definitions when it is needed, and do it automatically by employing tupling. 

The desirable implementation of the ``\lstinline{maxLength$^o$ xs m l}'' relation is the following (the definitions of ``\lstinline{gt$^o$}'' and ``\lstinline{le$^o$}'' are left out for brevity). It can be achieved with conjunctive partial deduction as well: 

\begin{lstlisting}[label={cpd:maxlen}]
let maxLength$^o$ xs m l = maxLength$_1^o$ xs m (zero ()) l

let rec maxLength$_1^o$ xs m n l = conde [
  (xs === nil () /\ m === n /\ l === zero ());
  (fresh (h t l$_1$)
     (xs === h % t) /\
     (l === succ l$_1$) /\
     (conde [
       (le$^o$ h n /\ maxLength$_1^o$ t m n l);
       (gt$^o$ h n /\ maxLength$_1^o$ t m h l)]))]
\end{lstlisting}

\subsection{CPD for Prolog-like languages}

Initially, conjunctive partial deduction was developed for Prolog-like languages.
Conjunctive partial deduction partially evaluates goals, which are conjunctions of atoms, using two levels of control: local and global~\cite{gluck1996controlling}. The global control determines which atoms are to be partially deduced. The local control~--- what the definitions for the atoms selected at the global control shall be.
Both local and global control construct tree structures which represent the input program. 

Local control constructs finite SLD-trees for conjunctions of atoms. 
The construction is guided with an \emph{unfold} operator: it selects a literal from the leaf of the partially constructed SLD-tree and adds its resolvents as children at each step.
Since, in general, SLD-trees are infinite, a decision to stop unfolding should be made at some point. 
There are several techniques for doing this, the most promising of them combine determinacy and either some well-founded or well-quasi order, such as homeomorphic embedding, or other measures. 

Global control determines the set of the conjunctions for which partial SLD-trees are built.
The important goal of the global control is to ensure termination.
The termination is achieved with the \emph{abstraction}.
If there is a goal which is embedded into the current goal, it points to the possibility of nontermination. 
The embedding tells that there is a certain similarity between the two goals, and if a current goal keeps being processed, then their similar subpart will appear again and again, causing nontermination.
Whenever the embedding goals are detected, the current goal is abstracted to remove the common subgoal from consideration. 

When the partial deduction is done, the only thing left is to construct the \emph{residual program}.
The clauses are generated from a partial SLD-tree, one tree per conjunction at the global level. 
A conjunction is uniquely \textit{renamed} to give a name for the predicate being defined. 
All free variables of the root of the tree become arguments of the predicate. 
For each non-failing path in the SLD-tree a clause is generated: a substitution collected along the path is substituted into the head of the clause, and the body is generated from what is in the leaf. 

\subsection{CPD for \textsc{miniKanren}}

In this section we describe how we adapted conjunctive partial deduction for \textsc{miniKanren}. 
We describe the particular unfolding and generalization strategies as well as discuss how the conjunctive partial deduction had to be modified as a response to the differences between \textsc{Prolog} and \textsc{miniKanren}. 

\subsubsection{Local Control}

Goals in \textsc{miniKanren} are different from those in \textsc{Prolog}-like languages: besides conjunction, disjunction and relation calls, there are explicit unification and  introduction of fresh variables. 
We normalize the input goal so that it was a disjunction of conjunctions of relation calls. 
To do so, we first pop all the fresh variables to the top level (``\lstinline{fresh (x) (p (x) /\ fresh (y) (q(x) \/ r(y, x)))}'' becomes ``\lstinline{fresh (x y) (p(x) \/ q(x) /\ r(y, x))}''). 
Then we transform the goal to be a disjuction of conjunctions of relation calls or unifications. 
All unifications in each conjunction are evaluated to some substitution (or the conjunct is discarded, if some unification fails). 
The normalization allows us to only consider conjunctions of relation calls while doing conjunctive partial deduction.

The local control constructs the following tree structure which represents the goal:

\begin{lstlisting}
type local_tree = 
    Fail
  | Success of subst
  | Leaf    of goal list * subst
  | Disj    of local_tree list
  | Conj    of local_tree * goal list
\end{lstlisting}

Leaf nodes can be either ``\lstinline{Fail}'', ``\lstinline{Success}'' or ``\lstinline{Leaf}''. 
The ``\lstinline{Fail}'' node is created whenever the evaluation of the current goal fails. 
When the current goal evaluates to some substitution, we create the ``\lstinline{Success}'' node with this substitution. 
The last leaf node is called ``\lstinline{Leaf}'', it corresponds to some partially evaluated goal. 
This type of node contains a substitution which has been computed up to this point, and a residual goal.
The goal in this type of node is then examined at the global level. 

``\lstinline{Disj}'' node corresponds to a disjunction in a goal: its children are the local control trees constructed for all disjuncts. 
The last type of nodes is a ``\lstinline{Conj}'' node. 
It is a transient node, which keeps track of a conjunction being unfolded. 

In general, unfolding replaces some of the relation calls with their bodies and partially evaluates them.
The particular unfolding strategy we adhere to is the following. 
At each step only one relation call is replaced with its body: the leftmost selectable relation call.
The selectable relation call is the one which does not embed any of its predecessors~--- goals which were unfolded in order to get the current goal. 
Embedding here is the modification of the homeomorphic embedding defined for the conjunctions of goals in conjunctive partial deduction literature~\cite{de1999conjunctive}. 
Since using pure embedding to control unfolding leads to hideously big programs, we also allow only one non-deterministic unfold.

\subsubsection{Global Control}

The conjunctions in the ``\lstinline{Leaf}'' nodes are processed at the global level. 
This step is responsible for the termination of the transformation. 
Generally speaking, the danger for nontermination arises whenever we encounter a subgoal which we have encountered before: processing the same thing will lead to itself over and over again. 
To break the vicious circle, one needs to stop unfolding the encountered subgoal, this is what \emph{abstraction} serves for.

The simplest case here is when we come upon the goal which is equal up to variable renaming to any other goal at the global level. 
When this happens, we stop exploring the goal. 
This is called \emph{variant check} in the literature, and it is done both at the global and the local control levels. 

The more complicated case is when a subpart of the goal repeats. 
This case we test with the modification of the homeomorphic embedding relation (strict homeomorphic embedding), initially developed for conjunctions. 
A conjunction $\overline{A}$ is considered embedded into a conjunction $\overline{B}$ when there is an ordered subconjunction within $\overline{A}$, each conjunct of which is embedded into the corresponding conjunct of $\overline{B}$:
\[
\overline{A} = A_0 \wedge A_1 \wedge \dots \wedge A_n \trianglelefteq B_0 \wedge B_1 \wedge \dots \wedge B_m = \overline{B}, \, \myif \, \exists \{ i_0 \dots i_m \mid \forall j.  i_j < i_{j+1} \}: \forall j \in \{0 \dots m\}. A_{i_j} \trianglelefteq B_j 
\]

A single conjunct is embedded into another ($A_i \trianglelefteq B_j$) when the following relation holds and $A_i$ is \emph{not} a strict instance of the second one $B_j$: 
\[
X \trianglelefteq Y, \text{where } X \text{ and } Y \text{ are variables}
\]
\[
f(x_0, x_1, \dots, x_n) \trianglelefteq f (y_0, y_1, \dots, y_n) \Leftrightarrow \forall i \in \{ 0 \dots n \}. x_i \trianglelefteq y_i
\]
\[
f \trianglelefteq g( y_0, y_1, \dots y_m) \Leftrightarrow \exists i \in \{ 0 \dots m \}. f \trianglelefteq y_i
\]

This check determines two major causes of the growth within the conjunctions. 
The conjunction can grow in some argument of a relation call or the number of conjuncts itself can grow. 
To mitigate the first source of the growth, the bigger conjunction can be replaced with a \emph{most specific generalization} of the two conjunctions.
Otherwise we need to \emph{split} the embedded subconjunction from the rest and start processing them separately. 
This process called \emph{abstraction} removes the subconjunctions which cause potential nontermination, and what is left should indeed be processed further.

\subsubsection{Residualization}

After the transformation is finished, a \emph{residual} program is constructed from the global control tree. 
A relation definition is generated for each conjunction at the global level (this is done with the renaming step of the original conjunctive partial deduction).
First, a unique name is given for each conjunction. 
Then free variables of the conjunction are collected to become the arguments of the relation: the constructors and constants are omitted (for example ``\lstinline{f x (succ y) /\ g (zero ()) z}'' becomes ``\lstinline{fG x y z}''.
The body of the definition is generated from the local control tree which corresponds to the conjunction under consideration.
The body is formed as a disjunction of conjunctions for the non-failure nodes of the local control tree. 
A computed substitution is transformed into a conjunction of unifications.
Suitable definitions are chosen for a goal in a leaf, and the conjunction of their applications is generated. 
As a final step we perform redundant argument filtering as described in~\cite{leuschel1996redundant}, and introduce fresh variables where necessary.
