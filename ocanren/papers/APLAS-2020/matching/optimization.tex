\section{Implementation and Optimizations}
\label{sec:optimization}

In this section we address two aspects of our solution: a number of optimizations which make the search more efficient, and
the way it ends up with the optimal solution.

The relational goal in its final form, presented in the previous section, does not demonstrate good performance. Thus, we apply a number
of techniques, some of which require extending the implementation of the search. Namely, we apply the following optimizations:

\begin{itemize}
\item We make use of type information to restrict the subset of constructors which may appear in a certain branch of
  program being synthesized.
\item After a complete set of samples is generated, we use it to put auxiliary constraints on matching expressions. For example,
  if we can detect that a matching expression points to a subexpression of scrutinee which can start with a single constructor (like
  tuples), and prohibit it from being considered during the synthesis.
\item We implement \emph{structural constraints} which allow us to restrict the shape of terms during the search, and
  utilize them to implement pruning.
\end{itemize}

In our formalization we do not make any use of types since as a rule type information does not affect matching. In addition,
utilizing the properties of a concrete type system would make our approach too coupled with this particular type system, hampering
its reusability for other languages. Nevertheless we may use a certain abstraction of type system which would deliver only
that part of information which is essential for our approach to function. Currently, we calculate the type of any matching expression in
the program being synthesized and from this type extract the subset of constructors which can appear when branching on this expression
is performed. The number of these constructors restricts the number of branches which a corresponding $\primi{switch}$ expression can have.
In our implementation we assume the constructor set ordered, and we consider only ordered branches, which restricts branching even more.

Our approach to finding an optimal solution in fact implements branch-and-bound strategy. The birds-eye view of our plan is as follows:

\begin{itemize}
\item We construct a trivial solution, which gives us the first estimation.
\item During the search we prune all partial solutions whose size exceeds current estimation. We can do this due to
  the top-down nature of partial solution construction.
\item When we come up with a better solution we remember it and update current estimation.
\end{itemize}

\noindent This strategy inevitably delivers us the optimal solution since there are only finitely many switch programs, shorter than trivial solution.

In order to implement this strategy we extended \textsc{OCanren} with a new primitive called \emph{structural constraint}, which may
fail on some terms depending on some criterion specified by an end-user. Structural constraints can be seen as a generalization of
some known constraints like \lstinline|absent$^o$| or \lstinline|symbol$^o$|~\cite{Untagged} in existing \textsc{miniKanren} implementations, 
so they can be widely used in solving other problems as well. Note, we could implement other constraints we considered (on the
depth of switch programs, on the type of scrutinee) as structural.
%\footnote{\textcolor{red}{I don't entirely understand this sentence}}.
However, our experience has shown that this leads to
a less efficient implementation. Since these constraints are inherent to the problem, we kept them hardcoded.

\subsection{Reducing the Complete Set of Samples}
\label{sec:reduced-samples}

Although in general our approach requires an exponential number of samples to be generated, in some cases a complete set of examples can be reduced.
For example, for the following pattern matching problem

\[
\begin{array}{l}
\mbox{\lstinline|(_, _ :: _ :: _)|} \\
\mbox{\lstinline|(_, _ :: _)|}
\end{array}
\]

the synthesized program should not investigate the left subtree of the scrutinee since its contents can not alter the behaviour of pattern matching.

The set of admissible matching values $s^\cup$ also can be restricted using the same arguments which we described in Section~\ref{sec:samples}.
This set essentially describes the paths to the ``interesting'' subexpressions of the scrutinee, and it can be computed statically before
the synthesis procedure:

\[
\begin{array}{rcl}
   s\,(m, C\ p_1 \dots p_k)     & = & \{m\}\cup \bigcup\limits_{i=1}^{k} s(m[i], p_i)\\
   s\,(m,\_)                 & = & \varnothing \\
   s^\cup\,(p_1,\dots, p_k) & = & \bigcup\limits_{i=1}^{k} s(\bullet, p_i)
\end{array}
\]

For the example above, the set  $s^\cup$ is

\[
\{\bullet, \bullet[1], \bullet[1][1]\}
\]

The complete set of samples then can be the following 3-element set:

\[
\begin{array}{l}
  \mbox{\lstinline|($\underline{[]}$, [])|}\\
  \mbox{\lstinline|($\underline{[]}$, $\;\underline{42}\;$ :: [])|}\\
  \mbox{\lstinline|($\underline{[]}$, $\;\underline{42}\;$ :: $\;\underline{42}\;$ :: $\;\underline{[]}$)|}
\end{array}
\]

\noindent where underlined expressions are chosen arbitrarily. A straightforward algorithm from the section~\ref{sec:samples} would generate the larger set of $3^2$ examples.

The set $s^\cup$ can be used for sample enumeration in the following manner. During the enumeration we hold current matching expression which will be used to
access current subtree of the sample. If that expression does not belong to $s^\cup$, we can choose an arbitrary inhabitant; if not we enumerate all
possible top-level constructors for this subexpression and recurse. The correctness of this algorithm relies on the fact that if an expression does
not belong to $s^\cup$, then all its extensions also does not belong to $s^\cup$.


