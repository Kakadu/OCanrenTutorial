%\documentclass[a4paper,UKenglish,cleveref,autoref]{lipics-v2019}

\documentclass[runningheads]{llncs}

\usepackage{amsmath,amssymb}
\usepackage[utf8]{inputenc}
\usepackage[english]{babel}
\usepackage{amssymb}
\usepackage{mathtools}
\usepackage{listings}
\usepackage{comment}
\usepackage{indentfirst}
\usepackage{hyperref}
%\usepackage{amsthm}
\usepackage{stmaryrd}
\usepackage{eufrak}
\usepackage{lstcoq}
\usepackage{placeins}
\usepackage[dvipsnames]{xcolor}
\usepackage{tcolorbox}
\usepackage{soul}
\usepackage{enumitem}
\setlist{nosep}
%\usepackage{biblatex}

\let\llncssubparagraph\subparagraph
%% Provide a definition to \subparagraph to keep titlesec happy
\let\subparagraph\paragraph
%% Load titlesec
%% Revert \subparagraph to the llncs definition
\usepackage{titlesec}
\let\subparagraph\llncssubparagraph

\titlespacing*{\section}{0pt}{1ex plus 1ex minus .2ex}{1ex plus .2ex}
\titlespacing*{\subsection}{0pt}{1ex plus 1ex minus .2ex}{1ex plus .2ex}
\titlespacing*{\paragraph}{0pt}{0ex}{1em}

%\newtheorem{theorem}{Theorem}
%\newtheorem{lemma}{Lemma}
%\newtheorem{corollary}{Corollary}
\newtheorem{hyp}{Hypothesis}
%\newtheorem{definition}{Definition}


\lstdefinelanguage{minikanren}{
basicstyle=\normalsize,
keywords={fresh},
sensitive=true,
commentstyle=\itshape\ttfamily, % \footnotesize\itshape\ttfamily,
keywordstyle=\textbf,
identifierstyle=\ttfamily,
basewidth={0.5em,0.5em},
columns=fixed,
fontadjust=true,
literate={fun}{{$\lambda\;\;$}}1 {->}{{$\to$}}3 {===}{{$\,\equiv\,$}}1 {=/=}{{$\not\equiv$}}1 {|>}{{$\triangleright$}}3 {/\\}{{$\wedge$}}2 {\\/}{{$\vee$}}2,
morecomment=[s]{(*}{*)}
}

\lstset{
mathescape=true,
language=minikanren
}

\usepackage{letltxmacro}
\newcommand*{\SavedLstInline}{}
\LetLtxMacro\SavedLstInline\lstinline
\DeclareRobustCommand*{\lstinline}{%
  \ifmmode
    \let\SavedBGroup\bgroup
    \def\bgroup{%
      \let\bgroup\SavedBGroup
      \hbox\bgroup
    }%
  \fi
  \SavedLstInline
}

\usepackage{todonotes}
\newcommand{\rednote}[1]{\todo[inline, color=red!20]{#1}}
\newcommand{\orangenote}[1]{\todo[inline, color=orange!20]{#1}}
\newcommand{\forestnote}[1]{\todo[inline, color=ForestGreen!20]{#1}}
\newcommand{\myhl}[2]{\colorbox{#1!50}{\parbox{\textwidth}{#2}}}

\def\transarrow{\xrightarrow}
\newcommand{\setarrow}[1]{\def\transarrow{#1}}

\def\padding{\phantom{X}}
\newcommand{\setpadding}[1]{\def\padding{#1}}

\def\subarrow{}
\newcommand{\setsubarrow}[1]{\def\subarrow{#1}}

\newcommand{\trule}[2]{\frac{#1}{#2}}
\newcommand{\crule}[3]{\frac{#1}{#2},\;{#3}}
\newcommand{\withenv}[2]{{#1}\vdash{#2}}
\newcommand{\trans}[3]{{#1}\transarrow{\padding{\textstyle #2}\padding}\subarrow{#3}}
\newcommand{\ctrans}[4]{{#1}\transarrow{\padding#2\padding}\subarrow{#3},\;{#4}}
\newcommand{\llang}[1]{\mbox{\lstinline[mathescape]|#1|}}
\newcommand{\pair}[2]{\inbr{{#1}\mid{#2}}}
\newcommand{\inbr}[1]{\left<{#1}\right>}
\newcommand{\highlight}[1]{\color{red}{#1}}
%\newcommand{\ruleno}[1]{\eqno[\scriptsize\textsc{#1}]}
\newcommand{\ruleno}[1]{\mbox{[\textsc{#1}]}}
\newcommand{\rulename}[1]{\textsc{#1}}
\newcommand{\inmath}[1]{\mbox{$#1$}}
\newcommand{\lfp}[1]{fix_{#1}}
\newcommand{\gfp}[1]{Fix_{#1}}
\newcommand{\vsep}{\vspace{-2mm}}
\newcommand{\supp}[1]{\scriptsize{#1}}
\newcommand{\sembr}[1]{\llbracket{#1}\rrbracket}
\newcommand{\cd}[1]{\texttt{#1}}
\newcommand{\free}[1]{\boxed{#1}}
\newcommand{\binds}{\;\mapsto\;}
\newcommand{\dbi}[1]{\mbox{\bf{#1}}}
\newcommand{\sv}[1]{\mbox{\textbf{#1}}}
\newcommand{\bnd}[2]{{#1}\mkern-9mu\binds\mkern-9mu{#2}}
\newcommand{\meta}[1]{{\mathcal{#1}}}
\newcommand{\dom}[1]{\mathtt{dom}\;{#1}}
\newcommand{\primi}[2]{\mathbf{#1}\;{#2}}
\renewcommand{\dom}[1]{\mathcal{D}om\,({#1})}
\newcommand{\ran}[1]{\mathcal{VR}an\,({#1})}
\newcommand{\fv}[1]{\mathcal{FV}\,({#1})}
\newcommand{\tr}[1]{\mathcal{T}r_{#1}}
\newcommand{\step}{\circ}

\newcommand{\searchRule}[6] {
  #1, #2 \vdash (#3, #4) \xRightarrow{#5} #6}
\newcommand{\extSearchRule}[8] {
  #1, #2, #3, #4 \vdash (#5, #6) \xRightarrow{#7}_{e} #8}
\newcommand{\q}{\hspace{0.5em}}
\newcommand{\bigcdot}{\boldsymbol{\cdot}}
\newcommand{\bigslant}[2]{{\raisebox{.2em}{$#1$}\left/\raisebox{-.2em}{$#2$}\right.}}

\let\emptyset\varnothing
\let\eps\varepsilon

\sloppy

\title{Certified Semantics for Relational Programming}

\author{Anonymous author}
%
\authorrunning{Anonymous author}
% First names are abbreviated in the running head.
% If there are more than two authors, 'et al.' is used.
%
\institute{Anonymous institute}

\begin{document}

\setlength{\belowcaptionskip}{-5pt}
\setlength{\abovecaptionskip}{0pt}

\setlength{\abovedisplayskip}{-3pt}
\setlength{\belowdisplayskip}{-2pt}
\setlength{\abovedisplayshortskip}{0pt}
\setlength{\belowdisplayshortskip}{2pt}

\maketitle

%TODO mandatory: add short abstract of the document
\begin{abstract}
  We present a formal study of semantics for relational programming language \textsc{miniKanren}. First,
  we formulate denotational semantics which corresponds to the minimal Herbrand model for definite logic
  programs. Second, we present operational semantics which models the distinctive feature of \textsc{miniKanren}
  implementation~--- interleaving,~--- and prove its soundness and completeness w.r.t. the denotational semantics.
  Our development is supported by a \textsc{Coq} specification, from which a reference interpreter can be
  extracted. We also derive from our main result a certified semantics (and a reference interpreter) for SLD resolution
  with cut and prove its soundness.
\end{abstract}

\section{Introduction}

The introductory book on \textsc{miniKanren}~\cite{TRS} describes the language by means of an evolving set of examples. In the
series of follow-up papers~\cite{MicroKanren,CKanren,CKanren1,AlphaKanren,2016,Guided} various extensions of the language were presented with
their semantics explained in terms of \textsc{Scheme} implementation. We argue that this style of semantic definition is
fragile and not self-evident since it requires the knowledge of semantics of concrete implementation language. In addition the justification of
important properties of relational programs (for example, refutational completeness~\cite{WillThesis}) becomes cumbersome. In the
area of programming languages research a formal definition for the semantics of language of interest is a \emph{de-facto} standard, and
in our opinion in its current state \textsc{miniKanren} deviates from this standard.

There were some previous attempts to define a formal semantics for \textsc{miniKanren}. \citet{RelConversion} present a variant of nondeterministic
operational semantics, and~\citet{DivTest} use another variant of finite-set semantics. None of them was capable of reflecting
the distinctive property of \textsc{miniKanren} search~--- \emph{interleaving}~\cite{Search}, thus deviating from the conventional understanding
of the language.

In this paper we present a formal semantics for core \textsc{miniKanren} and prove some its basic properties. First,
we define denotational semantics similar to the least Herbrand model for definite logic programs~\cite{LHM}; then
we describe operational semantics with interleaving in terms of labeled transition system. Finally, we prove the soundness and
completeness of the operational semantics w.r.t the denotational one. We support our development with a formal specification
using \textsc{Coq}~\cite{Coq} proof assistant\footnote{\url{https://github.com/dboulytchev/miniKanren-coq}}, thus outsourcing
the burden of proof checking to the automatic tool. 

The paper organized as follows. In Section~\ref{language} we give the syntax of the language, describe its semantics
informally and discuss some examples. Section~\ref{denotational} contains the description of denotational semantics for
the language, and Section~\ref{operational}~--- the operational semantics. In Section~\ref{equivalence} we overview the
certified proof for soundness and completeness of operational semantics. The final section concludes.

\begin{figure*}[t]
\[
\begin{array}{cccll}
  &\mathcal{C} & = & \{C_i^{k_i}\} & \mbox{constructors with arities} \\
  &\mathcal{T}_X & = & X \cup \{C_i^{k_i} (t_1, \dots, t_{k_i}) \mid t_j\in\mathcal{T}_X\} & \mbox{terms over the set of variables $X$} \\
  &\mathcal{D} & = & \mathcal{T}_\emptyset & \mbox{ground terms}\\
  &\mathcal{X} & = & \{ x, y, z, \dots \} & \mbox{syntactic variables} \\
  &\mathcal{A} & = & \{ \alpha, \beta, \gamma, \dots \} & \mbox{semantic variables} \\
  &\mathcal{R} & = & \{ R_i^{k_i}\} &\mbox{relational symbols with arities} \\
  &\mathcal{G} & = & \mathcal{T_X}\equiv\mathcal{T_X}   &  \mbox{unification} \\
  &            &   & \mathcal{G}\wedge\mathcal{G}     & \mbox{conjunction} \\
  &            &   & \mathcal{G}\vee\mathcal{G}       &\mbox{disjunction} \\
  &            &   & \mbox{\lstinline|fresh|}\;\mathcal{X}\;.\;\mathcal{G} & \mbox{fresh variable introduction} \\
  &            &   & R_i^{k_i} (t_1,\dots,t_{k_i}),\;t_j\in\mathcal{T_X} & \mbox{relational symbol invocation} \\
  &\mathcal{S} & = & \{R_i^{k_i} = \lambda\;x_1^i\dots x_{k_i}^i\,.\, g_i;\}\; g & \mbox{specification}
\end{array}
\]
\caption{The syntax of the source language}
\label{syntax}
\end{figure*}

\begin{comment}
\begin{figure}[t]
%\centering
\[
\begin{array}{rcl}
  \mathcal{FV}\,(x)&=&\{x\}\\
  \mathcal{FV}\,(C_i^{k_i}\,(t_1,\dots,t_{k_i}))&=&\bigcup\mathcal{FV}\,(t_i)\\
  \mathcal{FV}\,(t_1\equiv t_2)&=&\mathcal{FV}\,(t_1)\cup\mathcal{FV}\,(t_2)\\
  \mathcal{FV}\,(g_1\wedge g_2)&=&\mathcal{FV}\,(g_1)\cup\mathcal{FV}\,(g_2)\\
  \mathcal{FV}\,(g_1\vee g_2)&=&\mathcal{FV}\,(g_1)\cup\mathcal{FV}\,(g_2)\\
  \mathcal{FV}\,(\mbox{\lstinline|fresh|}\;x\;.\;g)&=&\mathcal{FV}\,(g)\setminus\{x\}\\
  \mathcal{FV}\,(R_i^{k_i}\,(t_1,\dots,t_{k_i}))&=&\bigcup\mathcal{FV}\,(t_i)
\end{array}
\]
\caption{Free variables in terms and goals}
\label{free}
\end{figure}
\end{comment}

\section{The Language}
\label{language}
 
In this section, we introduce the syntax of the language we use throughout the paper, describe the informal semantics, and give some examples.

The syntax of the language is shown in Fig.~\ref{syntax}. First, we fix a set of constructors $\mathcal{C}$ with known arities and consider
a set of terms $\mathcal{T}_X$ with constructors as functional symbols and variables from $X$. We parameterize this set with an alphabet of
variables since in the semantic description we will need \emph{two} kinds of variables. The first kind, \emph{syntactic} variables, is denoted
by $\mathcal{X}$. The second kind, \emph{semantic} or \emph{logic} variables, is denoted by $\mathcal{A}$.
We also consider an alphabet of \emph{relational symbols} $\mathcal{R}$ which are used to name relational definitions.
The central syntactic category in the language is \emph{goal}. In our case, there are five types of goals: \emph{unification} of terms,
conjunction and disjunction of goals, fresh variable introduction, and invocation of some relational definition. Thus, unification is used
as a constraint, and multiple constraints can be combined using conjunction, disjunction, and recursion.
The final syntactic category is a \emph{specification} $\mathcal{S}$. It consists of a set
of relational definitions and a top-level goal. A top-level goal represents a search procedure which returns a stream of substitutions for
the free variables of the goal. The definition for a set of free variables for both terms and goals is conventional;
%given in Figure~\ref{free};
as ``\lstinline|fresh|''
is the sole binding construct the definition is rather trivial. The language we defined is first-order, as goals can not be passed as parameters,
returned or constructed at runtime.

We now informally describe how relational search works. As we said, a goal represents a search procedure. This procedure takes a \emph{state} as input and returns a
stream of states; a state (among other information) contains a substitution that maps semantic variables into the terms over semantic variables. Then five types of
scenarios are possible (depending on the type of the goal):

\begin{itemize}
\item Unification ``\lstinline|$t_1$ === $t_2$|'' unifies terms $t_1$ and $t_2$ in the context of the substitution in the current state. If terms are unifiable,
  then their MGU is integrated into the substitution, and a one-element stream is returned; otherwise the result is an empty stream.
\item Conjunction ``\lstinline|$g_1$ /\ $g_2$|'' applies $g_1$ to the current state and then applies $g_2$ to each element of the result, concatenating
  the streams.
\item Disjunction ``\lstinline|$g_1$ \/ $g_2$|'' applies both its goals to the current state independently and then concatenates the results.
\item Fresh construct ``\lstinline|fresh $x$ . $g$|'' allocates a new semantic variable $\alpha$, substitutes all free occurrences of $x$ in $g$ with $\alpha$, and
  runs the goal.
\item Invocation ``$\lstinline|$R_i^{k_i}$ ($t_1$,...,$t_{k_i}$)|$'' finds a definition for the relational symbol \mbox{$R_i^{k_i}=\lambda x_1\dots x_{k_i}\,.\,g_i$}, substitutes
  all free occurrences of a formal parameter $x_j$ in $g_i$ with term $t_j$ (for all $j$) and runs the goal in the current state.
\end{itemize}

We stipulate that the top-level goal is preceded by an implicit ``\lstinline|fresh|'' construct, which binds all its free variables, and that the final substitutions
for these variables constitute the result of the goal evaluation.

Conjunction and disjunction form a monadic~\cite{Monads} interface with conjunction playing role of ``\lstinline|bind|'' and disjunction~--- of ``\lstinline|mplus|''.
In this description, we swept a lot of important details under the carpet~--- for example, in actual implementations the components of disjunction are not evaluated in
isolation, but both disjuncts are being evaluated incrementally with the control passing from one disjunct to another (\emph{interleaving})~\cite{Search};
the evaluation of some goals can be additionally deferred (via so-called ``\emph{inverse-$\eta$-delay}'')~\cite{MicroKanren}; instead of streams
the implementation can be based on ``ferns''~\cite{BottomAvoiding} to defer divergent computations, etc. In the following sections, we present
a complete formal description of relational semantics which resolves these uncertainties in a conventional way.

As an example consider the following specification. For the sake of brevity we
abbreviate immediately nested ``\lstinline|fresh|'' constructs into the one, writing ``\lstinline|fresh $x$ $y$ $\dots$ . $g$|'' instead of
``\lstinline|fresh $x$ . fresh $y$ . $\dots$ $g$|''.

\begin{tabular}{p{5.5cm}p{5.5cm}}
\begin{lstlisting}
append$^o$ = fun x y xy .
 ((x === Nil) /\ (xy === y)) \/
 (fresh h t ty .
   (x  === Cons (h, t))  /\
   (xy === Cons (h, ty)) /\
   (append$^o$ t y ty));

revers$^o$ x x
\end{lstlisting} &
\begin{lstlisting}
revers$^o$ = fun x y .
 ((x === Nil) /\ (y === Nil)) \/
 (fresh h t tr .
   (x === Cons (h, t)) /\
   (append$^o$ tr (Cons (h, Nil)) y) /\
   (revers$^o$ t tr));
\end{lstlisting}
\end{tabular}

Here we defined\footnote{We respect here a conventional tradition for \textsc{miniKanren} programming to superscript all relational names with ``$^o$''.}
two relational symbols~--- ``\lstinline|append$^o$|'' and ``\lstinline|revers$^o$|'',~--- and specified a top-level goal ``\lstinline|revers$^o$ x x|''.
The symbol ``\lstinline|append$^o$|'' defines a relation of concatenation of lists~--- it takes three arguments and performs a case analysis on the first one. If the
first argument is an empty list (``\lstinline|Nil|''), then the second and the third arguments are unified. Otherwise, the first argument is deconstructed into a head ``\lstinline|h|''
and a tail ``\lstinline|t|'', and the tail is concatenated with the second argument using a recursive call to ``\lstinline|append$^o$|'' and additional variable ``\lstinline|ty|'', which
represents the concatenation of ``\lstinline|t|'' and ``\lstinline|y|''. Finally, we unify ``\lstinline|Cons (h, ty)|'' with ``\lstinline|xy|'' to form a final constraint. Similarly,
``\lstinline|revers$^o$|'' defines relational list reversing. The top-level goal represents a search procedure for all lists ``\lstinline|x|'', which are stable under reversing, i.e.
palindromes. Running it results in an infinite stream of substitutions:

\begin{lstlisting}
   $\alpha\;\mapsto\;$ Nil
   $\alpha\;\mapsto\;$ Cons ($\beta_0$, Nil)
   $\alpha\;\mapsto\;$ Cons ($\beta_0$, Cons ($\beta_0$, Nil))
   $\alpha\;\mapsto\;$ Cons ($\beta_0$, Cons ($\beta_1$, Cons ($\beta_0$, Nil)))
   $\dots$
\end{lstlisting}

where ``$\alpha$''~--- a \emph{semantic} variable, corresponding to ``\lstinline|x|'', ``$\beta_i$''~--- free semantics variables. Therefore, each substitution represents a set of all palindromes of a certain length.


\begin{comment}
\begin{figure}[t]
\[
\begin{array}{rcl}
  x\,[t/x] &=& t \\
  y\,[t/x] &=& y,\;\; y\ne x\\
  C_i^{k_i}\,(t_1,\dots,t_{k_i})\,[t/x]&=&C_i^{k_i}\,(t_1\,[t/x],\dots,t_{k_i}\,[t/x])\\
  (t_1 \equiv t_2)\,[t/x]&=&t_1\,[t/x] \equiv t_2\,[t/x]\\
  (g_1 \wedge g_2)\,[t/x]&=&g_1\,[t/x] \wedge g_2\,[t/x]\\
  (g_1 \vee g_2)\,[t/x]&=&g_1\,[t/x] \vee g_2\,[t/x]\\
  (\mbox{\lstinline|fresh|}\;x\,.\,g)\,[t/x]&=&\mbox{\lstinline|fresh|}\;x\,.\,g\\
  (\mbox{\lstinline|fresh|}\;y\,.\,g)\,[t/x]&=&\mbox{\lstinline|fresh|}\;y\,.\,(g\,[t/x]),\;\;y\ne x\\
  (R_i^{k_i}\,(t_1,\dots,t_{k_i})\,[t/x]&=&R_i^{k_i}\,(t_1\,[t/x],\dots,t_{k_i}\,[t/x])
\end{array}
\]
  \caption{Substitutions for terms and goals}
  \label{substitution}
\end{figure}
\end{comment}

\section{Denotational Semantics}
\label{denotational}

In this section, we present a denotational semantics for the language we defined above. We use a simple set-theoretic
approach analogous to the least Herbrand model for definite logic programs~\cite{LHM}.
Strictly speaking, instead of developing it from scratch we could have just described the conversion of specifications
into definite logic form and took their least Herbrand model. However, in that case, we would still need to define
the least Herbrand model semantics for definite logic programs in a certified way. In addition, while for
this concrete language the conversion to definite logic form is trivial, it may become less trivial for
its extensions (with, for example, nominal constructs~\cite{AlphaKanren}) which we plan to do in future.

We also must make the following observations. First, building inductive denotational semantics in a conventional way amounts to
constructing a complete lattice and a monotone function and taking its least fixed point~\cite{TarskiKnaster}.
As we deal with a first-order language with only monotonic constructs (conjunction/disjunction) these steps
are trivial. Moreover, we express the semantics in \textsc{Coq}, where all well-formed inductive definitions already
have proper semantics, which removes the necessity to justify the validity of the steps we perform. Second, 
the least Herbrand model is traditionally defined as the least fixed point of a transition function (defined by a logic program)
which maps sets of ground atoms to sets of ground atoms. We are, however, interested in \emph{relational} semantics which
should map a program into $n$-ary relation over ground terms, where $n$ is the number of free variables in the topmost
goal. Thus, we deviate from the traditional route and describe the denotational semantics in a more specific way.

To motivate further development, we first consider the following example. Let us have the following goal:

\begin{lstlisting}
   x === Cons (y, z)
\end{lstlisting}

There are three free variables, and solving the goal delivers us the following single answer:

\begin{lstlisting}
   $\alpha\mapsto\;$ Cons ($\beta$, $\gamma$)
\end{lstlisting}

where semantic variables $\alpha$, $\beta$ and $\gamma$ correspond to the syntactic ones ``\lstinline|x|'', ``\lstinline|y|'', ``\lstinline|z|''. The
goal does not put any constraints on ``\lstinline|y|'' and ``\lstinline|z|'', so there are no bindings for ``$\beta$'' and ``$\gamma$'' in the answer.
This answer can be seen as the following ternary relation over the set of all ground terms:

\[
\{(\mbox{\lstinline|Cons ($\beta$, $\,\gamma$)|}, \beta, \gamma) \mid \beta\in\mathcal{D},\,\gamma\in\mathcal{D}\}\subset\mathcal{D}^3
\]

The order of ``dimensions'' is important, since each dimension corresponds to a certain free variable. Our main idea is to represent this relation as a set of total
functions 

\[
\mathfrak{f}:\mathcal{A}\mapsto\mathcal{D}
\]

from semantic variables to ground terms. We call these functions \emph{representing functions}. Thus, we may reformulate the same relation as

\[
\{(\mathfrak{f}\,(\alpha),\mathfrak{f}\,(\beta),\mathfrak{f}\,(\gamma))\mid\mathfrak{f}\in\sembr{\mbox{\lstinline|$\alpha$ === Cons ($\beta$, $\,\gamma$)|}}\}
\]

where we use conventional semantic brackets ``$\sembr{\bullet}$'' to denote the semantics. For the top-level goal, we need to substitute its free syntactic
variables with distinct semantic ones, calculate the semantics, and build the explicit representation for the relation as shown above. The relation, obviously,
does not depend on the concrete choice of semantic variables but depends on the order in which the values of representing functions are tupled. This order can be
conventionalized, which gives us a completely deterministic semantics.

Now we implement this idea. First, for a representing function

\[
\mathfrak{f} : \mathcal{A}\to\mathcal{D}
\]

we introduce its homomorphic extension 

\[
  \overline{\mathfrak{f}}:\mathcal{T_A}\to\mathcal{D}
\]

which maps terms to terms:

\[
\begin{array}{rcl}
  \overline{\mathfrak f}\,(\alpha) & = & \mathfrak f\,(\alpha)\\
  \overline{\mathfrak f}\,(C_i^{k_i}\,(t_1,\dots.t_{k_i})) & = & C_i^{k_i}\,(\overline{\mathfrak f}\,(t_1),\dots \overline{\mathfrak f}\,(t_{k_i}))
\end{array}
\]

Let us have two terms $t_1, t_2\in\mathcal{T_A}$. If there is a unifier for $t_1$ and $t_2$ then, clearly, there is a substitution $\theta$ which
turns both $t_1$ and $t_2$ into the same \emph{ground} term (we do not require $\theta$ to be the most general). Thus, $\theta$ maps
(some) variables into ground terms, and its application to $t_{1(2)}$ is exactly $\overline{\theta}(t_{1(2)})$. This reasoning can be
performed in the opposite direction: a unification $t_1\equiv t_2$ defines the set of all representing functions $\mathfrak{f}$ for which
$\overline{\mathfrak{f}}(t_1)=\overline{\mathfrak{f}}(t_2)$.

We will use the conventional notions of pointwise modification of a function $f\,[x\gets v]$
\begin{comment}
\[
f\,[x\gets v]\,(z)=\left\{
\begin{array}{rcl}
  f\,(z) &,& z \ne x \\
  v      &,& z = x
\end{array}
\right.
\]
\end{comment}
and substitution $g\,[t/x]$ of a free variable $x$ with a term $t$ in a goal (or a term) $g$.

%(see Figure~\ref{substitution}).

For a representing function $\mathfrak{f}:\mathcal{A}\to\mathcal{D}$ and a semantic variable $\alpha$ we define
the following \emph{generalization} operation:

\[
\mathfrak{f}\uparrow\alpha = \{ \mathfrak{f}\,[\alpha\gets d] \mid d\in\mathcal D\}
\]

Informally, this operation generalizes a representing function into a set of representing functions in such a way that the
values of these functions for a given variable cover the whole $\mathcal{D}$. We extend the generalization operation for sets of
representing functions $\mathfrak{F}\subseteq\mathcal{A}\to\mathcal{D}$:

\[
  \mathfrak{F}\uparrow\alpha = \bigcup_{\mathfrak{f}\in\mathfrak{F}}(\mathfrak{f}\uparrow\alpha)
\]

Now we are ready to specify the semantics for goals (see Fig.~\ref{denotational_semantics_of_goals}).
We've already given the motivation for
the semantics of unification: the condition $\overline{\mathfrak{f}}(t_1)=\overline{\mathfrak{f}}(t_2)$ gives us the set of all (otherwise
  unrestricted) representing functions which ``equate'' terms $t_1$ and $t_2$.
  Set union and intersection provide a conventional interpretation
for disjunction and conjunction of goals. In the case is of a relational invocation we unfold the definition of the corresponding
relational symbol and substitute its formal parameters with actual ones.

The only non-trivial case is that of ``\lstinline|fresh $x$ . $g$|''. First, we take an arbitrary semantic variable $\alpha$,
not free in $g$, and substitute $x$ with $\alpha$. Then we calculate the semantics of $g\,[\alpha/x]$. The interesting part is the next step:
as $x$ can not be free in ``\lstinline|fresh $x$ . $g$|'', we need to generalize the result over $\alpha$ since in our model the semantics of a
goal specifies a relation over its free variables. We introduce some nondeterminism by choosing arbitrary $\alpha$, but we can prove that with different
choices of free variable the semantics of a goal does not change.

\begin{lemma}
\label{lem:den_sem_change_var}
For any goal \lstinline|fresh $x$ . $g$|, for any two variables $\alpha$ and $\beta$ which are not free in this goal,
if $\mathfrak{f} \in \sembr{g\,[\alpha/x]}$, then for any representing function $\mathfrak{f}'$, such that

\begin{enumerate}
\item $\mathfrak{f}'(\beta) = \mathfrak{f}(\alpha)$
\item $\forall \gamma: \gamma \neq \alpha \land \gamma \neq \beta,\; \mathfrak{f}'(\gamma) = \mathfrak{f}(\gamma)$
\end{enumerate}

\noindent it is true that $\mathfrak{f}' \in \sembr{g\,[\beta/x]}$.
\end{lemma}
%\begin{proof}
  The proof turned out to be the most cumbersome among all others in the case where $g$ is a nested \lstinline|fresh| contruct. In that case, we have to constructively build two representing (including an intermediate one for an intermediate goal) by pointwise modification. The details of this proof can be found in the Appendix~\ref{appendix_den_sem_change_var_proof}, the full proof script is in the specification in Coq.
%\end{proof}

\begin{figure}[t]
  \[
  \begin{array}{cclr}
    \sembr{t_1\equiv t_2}&=&\{\mathfrak f : \mathcal{A}\to\mathcal{D}\mid \overline{\mathfrak{f}}\,(t_1)=\overline{\mathfrak{f}}\,(t_2)\}& \ruleno{Unify$_D$}\\
    \sembr{g_1\wedge g_2}&=&\sembr{g_1}\cap\sembr{g_1}&\ruleno{Conj$_D$}\\
    \sembr{g_1\vee g_2}&=&\sembr{g_1}\cup\sembr{g_1}&\ruleno{Disj$_D$}\\
    \sembr{\mbox{\lstinline|fresh|}\,x\,.\,g}&=&(\sembr{g\,[\alpha/x]})\uparrow\alpha,\;\alpha\not\in FV(g)& \ruleno{Fresh$_D$}\\
    \sembr{R\,(t_1,\dots,t_k)}&=&\sembr{g\,[t_1/x_1,\dots,t_k/x_k]},\;\mbox{where}\;R=\lambda\,x_1\dots x_k\,.\,g & \ruleno{Invoke$_D$}
  \end{array}
  \]
  \caption{Denotational semantics of goals}
  \label{denotational_semantics_of_goals}
\end{figure}

\begin{comment}
Here is an example of denotational semantics of a goal:

%\renewcommand{\overset}[2]{#2}
\[
\begin{array}{lc}
  \sembr{\mbox{\lstinline|fresh y . ($\alpha$ === y) $\,\wedge\,$ (y === Zero)|}}&\overset{\mbox{(by \textsc{Fresh$_D$})}}{=}\\
  (\sembr{\mbox{\lstinline|($\alpha$ === $\beta$) $\,\wedge\,$ ($\beta$ === Zero)|}})\uparrow\beta&\overset{\mbox{(by \textsc{Conj$_D$})}}{=}\\
  (\sembr{\mbox{\lstinline|$\alpha$ === $\beta$|}} \,\cap\, \sembr{\mbox{\lstinline|$\beta$ === Zero)|}})\uparrow\beta&\overset{\mbox{(by \textsc{Unify$_D$})}}{=}\\
  (\{\mathfrak{f}\mid \overline{\mathfrak{f}}\,(\alpha)=\overline{\mathfrak{f}}\,(\beta)\} \,\cap\, \{\mathfrak{f}\mid \overline{\mathfrak{f}}\,(\beta)=\overline{\mathfrak{f}}\,(\mbox{\lstinline|Zero|})\})\uparrow\beta&\overset{\mbox{(by the definition of ``$\overline{\mathfrak{f}}$'')}}{=}\\
  (\{\mathfrak{f}\mid \mathfrak{f}\,(\alpha)=\mathfrak{f}\,(\beta)\} \,\cap\, \{\mathfrak{f}\mid \mathfrak{f}\,(\beta)=\mbox{\lstinline|Zero|}\})\uparrow\beta&\overset{\mbox{(by the definition of ``$\cap$'')}}{=}\\
  (\{\mathfrak{f}\mid \mathfrak{f}\,(\alpha)=\mathfrak{f}\,(\beta)=\mbox{\lstinline|Zero|}\})\uparrow\beta&\overset{\mbox{(by the definition of ``$\uparrow$'')}}{=}\\
  \{\mathfrak{f}\mid \mathfrak{f}\,(\alpha)=\mbox{\lstinline|Zero|}, \mathfrak{f}\,(\beta)=d, d\in\mathcal{D}\}&\overset{\mbox{(by the totality of representing functions)}}{=}\\[1mm]
  \{\mathfrak{f}\mid \mathfrak{f}\,(\alpha)=\mbox{\lstinline|Zero|}\}&
\end{array}
\]

In the end, we've got the set of representing functions, each of which restricts only the value of the free variable $\alpha$.
\end{comment}

We can prove the following important \emph{closedness condition} for the semantics of a goal $g$.

\begin{lemma}[Closedness condition]
\label{lem:closedness_condition}
For any goal $g$ and two representing functions ${\mathfrak f}$ and ${\mathfrak f'}$, such that $\left.{\mathfrak f}\right|_{FV(g)} = \left.{\mathfrak f'}\right|_{FV(g)}$, it is true, that
${\mathfrak f} \in \sembr{g} \Leftrightarrow {\mathfrak f'} \in \sembr{g}$.
\end{lemma}

\begin{comment}
\[
\forall {\mathfrak f}, {\mathfrak f'}:  \left.{\mathfrak f}\right|_{FV(g)} = \left.{\mathfrak f'}\right|_{FV(g)}, \quad {\mathfrak f} \in \sembr{g} \Leftrightarrow {\mathfrak f'} \in \sembr{g}
\]
\end{comment}

In other words, representing functions for a goal $g$ restrict only the values of free variables of $g$ and do not introduce any ``hidden'' correlations.
This condition guarantees that our semantics is closed in the sense that it does not introduce artificial restrictions for the relation it defines.

\section{Operational Semantics}
\label{operational}

In this section we describe the operational semantics of \textsc{miniKanren}, which corresponds to the known
implementations with interleaving search. The semantics is given in the form of a labeled transition system (LTS)~\cite{LTS}. From now on we
assume the set of semantic variables to be linearly ordered ($\mathcal{A}=\{\alpha_1,\alpha_2,\dots\}$).

We introduce the notion of substitution

\[
  \sigma : \mathcal{A}\to\mathcal{T_A}
\]

as a (partial) mapping from semantic variables to terms over the set of semantic variables. We denote $\Sigma$ the
set of all substitutions, $\dom{\sigma}$~--- the domain for a substitution $\sigma$,
$\ran{\sigma}=\bigcup_{\alpha\in\mathcal{D}om\,(\sigma)}\fv{\sigma\,(\alpha)}$~--- its range (the set of all free variables in the image).

The \emph{non-terminal states} in the transition system have the following shape:

\[
S = \mathcal{G}\times\Sigma\times\mathbb{N}\mid S\oplus S \mid S \otimes \mathcal{G}
\]

As we will see later, an evaluation of a goal is separated into elementary steps, and these steps are performed interchangeably for different subgoals. 
Thus, a state has a tree-like structure with intermediate nodes corresponding to partially-evaluated conjunctions (``$\otimes$'') or
disjunctions (``$\oplus$''). A leaf in the form $\inbr{g, \sigma, n}$ determines a goal in a context, where $g$~--- a goal, $\sigma$~--- a substitution accumulated so far,
and $n$~--- a natural number, which corresponds to a number of semantic variables used to this point. For a conjunction node, its right child is always a goal since
it cannot be evaluated unless some result is provided by the left conjunct.

The full set of states also include one separate terminal state (denoted by $\diamond$), which symbolizes the end of the evaluation.

\[
\hat{S} = \diamond \mid S
\]

We will operate with the well-formed states only, which are defined as follows.

\begin{definition}
  Well-formedness condition for extended states:
  
  \begin{itemize}
  \item $\diamond$ is well-formed;
  \item $\inbr{g, \sigma, n}$ is well-formed iff $\fv{g}\cup\dom{\sigma}\cup\ran{\sigma}\subset\{\alpha_1,\dots,\alpha_n\}$;
  \item $s_1\oplus s_2$ is well-formed iff $s_1$ and $s_2$ well-formed;
  \item $s\otimes g$ is well-formed iff $s$ is well-formed and for all leaf triplets $\inbr{\_,\_,n}$ in $s$ it is true that $\fv{g}\subseteq\{\alpha_1,\dots,\alpha_n\}$.
  \end{itemize}
  
\end{definition}

Informally the well-formedness restricts the set of states to those in which all goals use only allocated variables.

Finally, we define the set of labels:

\[
L = \step \mid \Sigma\times \mathbb{N}
\]

The label ``$\step$'' is used to mark those steps which do not provide an answer; otherwise, a transition is labeled by a pair of a substitution and a number of allocated
variables. The substitution is one of the answers, and the number is threaded through the derivation to keep track of allocated variables.

\begin{figure*}[t]
  \renewcommand{\arraystretch}{1.6}
  \[
  \begin{array}{cr}
    \inbr{t_1 \equiv t_2, \sigma, n} \xrightarrow{\step} \Diamond , \, \, \nexists\; mgu\,(t_1 \sigma, t_2 \sigma) &\ruleno{UnifyFail} \\
    \inbr{t_1 \equiv t_2, \sigma, n} \xrightarrow{(mgu\,(t_1 \sigma, t_2 \sigma) \circ \sigma,\, n)} \Diamond & \ruleno{UnifySuccess} \\
    \inbr{g_1 \lor g_2, \sigma, n} \xrightarrow{\step} \inbr{g_1, \sigma, n} \oplus \inbr{g_2, \sigma, n} & \ruleno{Disj} \\
    \inbr{g_1 \land g_2, \sigma, n} \xrightarrow{\step} \inbr{ g_1, \sigma, n} \otimes g_2 & \ruleno{Conj} \\
    \inbr{\mbox{\lstinline|fresh|}\, x\, .\, g, \sigma, n} \xrightarrow{\step} \inbr{g\,[\bigslant{\alpha_{n + 1}}{x}], \sigma, n + 1} & \ruleno{Fresh} \\
    \dfrac{R_i^{k_i}=\lambda\,x_1\dots x_{k_i}\,.\,g}{\inbr{R_i^{k_i}\,(t_1,\dots,t_{k_i}),\sigma,n} \xrightarrow{\step} \inbr{g\,[\bigslant{t_1}{x_1}\dots\bigslant{t_{k_i}}{x_{k_i}}], \sigma, n}} & \ruleno{Invoke}\\
    \dfrac{s_1 \xrightarrow{\step} \Diamond}{(s_1 \oplus s_2) \xrightarrow{\step} s_2} & \ruleno{SumStop}\\
    \dfrac{s_1 \xrightarrow{r} \Diamond}{(s_1 \oplus s_2) \xrightarrow{r} s_2} & \ruleno{SumStopAns}\\
    \dfrac{s \xrightarrow{\step} \Diamond}{(s \otimes g) \xrightarrow{\step} \Diamond} &\ruleno{ProdStop}\\
    \dfrac{s \xrightarrow{(\sigma, n)} \Diamond}{(s \otimes g) \xrightarrow{\step} \inbr{g, \sigma, n}}  & \ruleno{ProdStopAns}\\
    \dfrac{s_1 \xrightarrow{\step} s'_1}{(s_1 \oplus s_2) \xrightarrow{\step} (s_2 \oplus s'_1)} &\ruleno{SumStep}\\
    \dfrac{s_1 \xrightarrow{r} s'_1}{(s_1 \oplus s_2) \xrightarrow{r} (s_2 \oplus s'_1)} &\ruleno{SumStepAns}\\
    \dfrac{s \xrightarrow{\step} s'}{(s \otimes g) \xrightarrow{\step} (s' \otimes g)} &\ruleno{ProdStep}\\
    \dfrac{s \xrightarrow{(\sigma, n)} s'}{(s \otimes g) \xrightarrow{\step} (\inbr{g, \sigma, n} \oplus (s' \otimes g))} & \ruleno{ProdStepAns} 
  \end{array}
  \]
  \caption{Operational semantics of interleaving search}
  \label{lts}
\end{figure*}

The transition rules are shown in Fig.~\ref{lts}. The first two rules specify the semantics of unification. If two terms are not unifiable under the current substitution
$\sigma$ then the evaluation stops with no answer; otherwise, it stops with the most general unifier applied to a current substitution as an answer.

The next two rules describe the steps performed when disjunction or conjunction is encountered on the top level of the current goal. For disjunction, it schedules both goals (using ``$\oplus$'') for
evaluating in the same context as the parent state, for conjunction~--- schedules the left goal and postpones the right one (using ``$\otimes$'').

The rule for ``\lstinline|fresh|'' substitutes bound syntactic variable with a newly allocated semantic one and proceeds with the goal.

The rule for relation invocation finds a corresponding definition, substitutes its formal parameters with the actual ones, and proceeds with the body.

The rest of the rules specify the steps performed during the evaluation of two remaining types of the states~--- conjunction and disjunction. In all cases, the left state
is evaluated first. If its evaluation stops, the disjunction evaluation proceeds with the right state, propagating the label (\textsc{SumStop} and \textsc{SumStep}), and the conjunction schedule the right goal for evaluation in the context of the returned answer (\textsc{ProdStep}) or stops if there is no answer (\textsc{ProdStop}).

The last four rules describe \emph{interleaving}, which occurs when the evaluation of the left state suspends with some residual state (with or without an answer). In the case of disjunction
the answer (if any) is propagated, and the constituents of the disjunction are swapped (\textsc{SumStep}, \textsc{SumStepAns}). In the case of conjunction, if the evaluation step in
the left conjunct did not provide any answer, the evaluation is continued in the same order since there is still no information to proceed with the evaluation of the right
conjunct (\textsc{ProdStep}); if there is some answer, then the disjunction of the right conjunct in the context of the answer and the remaining conjunction is
scheduled for evaluation (\textsc{ProdStepAns}).

The introduced transition system is completely deterministic: there is exactly one transition from any non-terminal state.
There was, however, some freedom in choosing the order of evaluation for conjunction and
disjunction states. For example, instead of evaluating the left substate first, we could choose to evaluate the right one, etc. In each concrete case, we would
end up with a different (but still deterministic) system that would prescribe different semantics to a concrete goal. This choice reflects the inherent
non-deterministic nature of search in relational (and, more generally, logical) programming. However, as long as deterministic search procedures
are sound and complete, we can consider them ``equivalent''.\footnote{There still can be differences in observable behavior of concrete goals under different
sound and complete search strategies. For example, a goal can be refutationally complete~\cite{WillThesis} under one strategy and non-complete under another.}

It is easy to prove that transitions preserve well-formedness of states.

\begin{lemma}{(Well-formedness preservation)}
\label{lem:well_formedness_preservation}
For any transintion $s \xrightarrow{l} \hat{s}$, if $s$ is well-formed then $\hat{s}$ is also well-formed.
\end{lemma}

A derivation sequence for a certain state determines a \emph{trace}~--- a finite or infinite sequence of answers. The trace corresponds to the stream of answers
in the reference \textsc{miniKanren} implementations. We denote a set of answers in the trace for state $\hat{s}$ by $\tr{\hat{s}}$.

We can relate sets of answers for the partially evaluated conjunction and disjunction with sets of answers for their constituents by the two following lemmas.

\begin{lemma}
\label{lem:sum_answers}
For any non-termial states $s_1$ and $s_2$, $\tr{s_1 \oplus s_2} = \tr{s_1} \cup \tr{s_2}$.
\end{lemma}

\begin{lemma}
\label{lem:prod_answers}
For any nontermial state $s$ and goal $g$,  \mbox{$\tr{s \otimes g} = \bigcup_{(\sigma, n) \in \tr{s}} \tr{\inbr{g, \sigma, n}}$}.
\end{lemma}

We also can easily describe the criterion of termination for disjunctions.

\begin{lemma}
\label{lem:disj_termination}
For any goals $g_1$ and $g_2$, sunbstitution $\sigma$, and number $n$, the trace from the state $\inbr{g_1 \vee g_2, \sigma, n}$ is finite iff the traces from both $\inbr{g_1, \sigma, n}$ and $\inbr{g_2, \sigma, n}$ are finite.
\end{lemma}

These simple statements already allow us to prove two important properties of interleaving search as corollaries: the ``fairness'' of disjuction~--- the fact that trace for disjunction contains all the answers from both streams for disjuncts~--- and the ``commutativity'' of disjunctions~--- the fact that swapping two disjuncts (at the top level) does not change the termination of the goal evaluation. 

\section{Semantics Equivalence}
\label{equivalence}

Now when we defined two different kinds of semantics for \textsc{miniKanren} we can relate them and show that the results given by these two semantics are the same for any specification.
This will actually say something important about the search in the language: since operational semantics describes precisely the behavior of the search and denotational semantics
ignores the search and describes what we \emph{should} get from mathematical point of view, by proving their equivalence we establish \emph{completeness} of the search which
means that the search will get all answers satisfying the described specification and only those.

But first, we need to relate the answers produced by these two semantics as they have different forms: a trace of substitutions (along with numbers of allocated variables)
for operational and a set of representing functions for denotational. We can notice that the notion of representing function is close to substitution, with only two differences:

\begin{itemize}
\item representing function is total;
\item terms in the domain of representing function are ground.
\end{itemize}

Therefore we can easily extend (perhaps ambiguously) any substitution to a representing function by composing it with an arbitrary representing function and that will
preserve all variable dependencies in the substitution. So we can define a set of representing functions corresponding to substitution as follows:

\[
[\sigma] = \{\overline{\mathfrak f} \circ \sigma \mid \mathfrak{f}:\mathcal{A}\mapsto\mathcal{D}\}
\]

And \emph{denotational analog} of an operational semantics (a set of representing functions corresponding to answers in the trace) for given extended state $s$ is
then defined as a union of sets for all substitution in the trace:

\[
\sembr{s}_{op} = \cup_{(\sigma, n) \in \tr{s}} [\sigma]
\]

This allows us to state theorems relating two semantics.

\begin{theorem}[Operational semantics soundness]
For any specification $\{\dots\}\; g$, for which the indices of all free variables in $g$ are limited by some number $n$

\[
\sembr{\inbr{g, \epsilon, n}}_{op} \subset \sembr{\{\dots\}\; g}.
\]
\end{theorem}

It can be proven by nested induction, but first, we need to generalize the statement so that the inductive hypothesis would be strong enough for the inductive step.
To do so, we define denotational semantics not only for goals but for arbitrarily extended states. Note that this definition does not need to have any intuitive
interpretation, it is introduced only for proof to go smoothly. The definition of the denotational semantics for extended states is on Figure~\ref{denotational_semantics_of_states}.
The generalized version of the theorem uses it:

\begin{figure}[t]
  \[
  \begin{array}{ccl}
    \sembr{\Diamond}_\Gamma&=&\emptyset\\
    \sembr{\inbr{g, \sigma, n}}_\Gamma&=&\sembr{g}_\Gamma\cap[\sigma]\\
    \sembr{s_1 \oplus s_2}_\Gamma&=&\sembr{s_1}_\Gamma\cup\sembr{s_2}_\Gamma\\
    \sembr{s \otimes g}_\Gamma&=&\sembr{s}_\Gamma\cap\sembr{g}_\Gamma\\
  \end{array}
  \]
  \caption{Denotational semantics of states}
  \label{denotational_semantics_of_states}
\end{figure}

\begin{lemma}[Generalized soundness]
For any top-level environment $\Gamma_0$ acquired from some specification, for any well-formed (w.r.t. that specification) extended state $s$

\[
\sembr{s}_{op} \subset \sembr{s}_{\Gamma_0}.
\]
\end{lemma}

It can be proven by induction on the number of steps in which a given answer (more accurately, the substitution that contains it) occurs in the trace.
The induction step is proven by structural induction on the extended state $s$.

It would be tempting to formulate the completeness of operational semantics as the inverse inclusion, but it does not hold in such generality. The reason for
this is that denotational semantics encodes only dependencies between the free variables of a goal, which is reflected by the completeness condition, while
operational semantics may also contain dependencies between semantic variables allocated in ``\lstinline|fresh|''. Therefore we formulate the completeness
with representing functions restricted on the semantic variables allocated in the beginning (which includes all free variables of a goal). This does not
compromise our promise to prove the completeness of the search as \textsc{miniKanren} provides the result as substitutions only for queried variables,
which are allocated in the beginning.

\begin{theorem}[Operational semantics completeness]
For any specification $\{\dots\}\; g$, for which the indices of all free variables in $g$ are limited by some number $n$

\[
\{\mathfrak{f}|_{\{\alpha_1,\dots,\alpha_n\}} \mid \mathfrak{f} \in \sembr{\{\dots\}\; g}\} \subset \{\mathfrak{f}|_{\{\alpha_1,\dots,\alpha_n\}} \mid \mathfrak{f} \in \sembr{\inbr{g, \epsilon, n}}_{op}\}.
\]
\end{theorem}


Similarly to the soundness, this can be proven by nested induction, but the generalization is required. This time it is enough to generalize it from goals
to states of the shape $\inbr{g, \sigma, n}$. We also need to introduce one more auxiliary semantics --- bounded denotational semantics:

\[
\sembr{\bullet}^l : \mathcal{G} \to 2^{\mathcal{A}\to\mathcal{D}}
\]

Instead of always unfolding the definition of a relation for invocation goal, it does so only given number of times. So for a given set of relational
definitions $\{R_i^{k_i} = \lambda\;x_1^i\dots x_{k_i}^i\,.\, g_i;\}$ the definition of bounded denotational semantics is exactly the same as in usual denotational semantics,
except that for the invocation case:

\[
\sembr{R_i^{k_i}\,(t_1,\dots,t_{k_i})}^{l+1} = \sembr{g_i[t_1/x_1^i, \dots, t_{k_i}/x_{k_i}^i]}^{l}
\]

It is convenient to define bounded semantics for level zero as an empty set:

\[
\sembr{g}^{0} = \emptyset
\]

Bounded denotational semantics is an approximation of a usual denotational semantics and it is clear that any answer in usual denotational semantics will also be in
bounded denotational semantics for some level:

\begin{lemma}
$\sembr{g}_{\Gamma_0} \subset \cup_l \sembr{g}^l$
\end{lemma}

Formally it can be proven using the definition of the least fixed point from Tarski-Knaster theorem: the set on the right-hand side is a closed set.

Now the generalized version of the completeness theorem is as follows:

\begin{lemma}[Generalized completeness]
For any set of relational definitions, for any level $l$, for any well-formed (w.r.t. that set of definitions) state $\inbr{g, \sigma, n}$,

\[
\{\mathfrak{f}|_{\{\alpha_1,\dots,\alpha_n\}} \mid \mathfrak{f} \in \sembr{g}^l \cap [\sigma]\} \subset \{\mathfrak{f}|_{\{\alpha_1,\dots,\alpha_n\}} \mid \mathfrak{f} \in \sembr{\inbr{g, \sigma, n}}_{op}\}.
\]
\end{lemma}

It is proven by induction on the level $l$. The induction step is proven by structural induction on the goal $g$.

The proofs of both theorems are certified in \textsc{Coq}, although the proofs for a number of (obvious) technical facts about representing functions and computation of the most
general unifier as well as some properties of denotational semantics, proven informally in Section~\ref{denotational}, are
admitted for now. For completeness we can not just use the induction on proposition \lstinline|in_denotational_sem_goal|, as it would be natural to expect,
because the inductive principle it provides is not flexible enough. So we need to define bounded denotational semantics in our formalization too and perform
induction on the level explicitly:

\begin{lstlisting}[language=Coq]
   Inductive in_denotational_sem_lev_goal : nat -> goal -> repr_fun -> Prop :=
   ...
   | dslgInvoke : forall l r t f,
        in_denotational_sem_lev_goal l (proj1_sig (Prog r) t) f ->
        in_denotational_sem_lev_goal (S l) (Invoke r t) f.
\end{lstlisting}

The lemma relating bounded and unbounded denotational semantics is translated into \textsc{Coq}:

\begin{lstlisting}[language=Coq]
   Lemma in_denotational_sem_some_lev: forall (g : goal) (f : repr_fun),
        in_denotational_sem_goal g f ->
        exists l, in_denotational_sem_lev_goal l g f.
\end{lstlisting}

The statements of the theorems are as follows:

\begin{lstlisting}[language=Coq]
   Theorem search_correctness: forall (g : goal) (k : nat) (f : repr_fun) (t : trace),
      closed_goal_in_context (first_nats k) g) ->
      op_sem (State (Leaf g empty_subst k)) t) ->
      in_denotational_analog t f ->
      in_denotational_sem_goal g f.
      
   Theorem search_completeness: forall (g : goal) (k : nat) (f : repr_fun) (t : trace),
      closed_goal_in_context (first_nats k) g) ->
      op_sem (State (Leaf g empty_subst k)) t) ->
      in_denotational_sem_goal g f ->
      exists (f' : repr_fun), (in_denotational_analog t f') /\
                         forall (x : var), In x (first_nats k) -> f x = f' x.
\end{lstlisting}

One important immediate corollary of these theorems is the correctness of certain program transformations. Since the results obtained by the search on a
specification are exactly the results from the mathematical model of this specification, after the transformations of relations that do not change their
mathematical meaning the search will obtain the same results. Note that this way we guarantee only the stability of results as the set of ground terms,
the other aspects of program behavior, such as termination, may be affected. This allows us to safely (to a certain extent) apply such natural
transformations as:

\begin{itemize}
\item changing the order of constituents in conjunction or disjunction;
\item swapping conjunction and disjunction using distributivity;
\item moving fresh variable introduction.
\end{itemize}

and even transform relational definitions to some kinds of normal form (like all fresh variables introduction on the top level with the
conjunctive normal form inside), which may be convenient, for example, for metacomputation.

\section{Specification in \textsc{Coq}}
\label{specification}

We certified all the definitions and propositions from the previous sections using \textsc{Coq} proof assistant. The \textsc{Coq} specification for the most parts closely follows the formal descriptions we gave by means of inductive definitions (and inductively defined propositions in particular) and structural induction in proofs. The detailed description of the specification, including code snippets, is provided in the Appendix~\ref{appendix_coq}, and in this section we adress only some non-trivial parts of it and some design choices.

The language formalized in \textsc{Coq} has a few non-essential simplifications for the sake of convenience. Specifically, we restrict the arities of all constructors to be either zero or two and require all relations to have exactly one argument. These restrictions do not make the language less expressive in any way since we can always represent a sequence of terms as a list using constructors \lstinline|Nil$^0$| and \lstinline|Cons$^2$|. 

In our formalization of the language we use higher-order abstract syntax~\cite{HOAS} for variable binding, therefore we work explicitly only with semantic variables. We preferred it to the first-order syntax because it gives us the ability to use substitution and the induction principle provided by \textsc{Coq}. On the other hand, we need to explicitly specify a requirement on the syntax representation, which is trivially fulfilled in the first-order case: all bindings have to be ``consistent'', i.e. if we instantiate a higher-order \lstinline|fresh| construct with different semantic variables the results will be the same up to some renaming (provided that both those variables are not free in the body of the binder). Another requirement we have to specify explicitly (independent of HOAS/FOAS dichotomy) is a requirement that the definitions of relations do not contain unbound semantic variables.

To formalize the operational part in \textsc{Coq} we first need to define all preliminary notions from unification theory~\cite{Unification} which our semantics uses. In particular, we need to implement the notion of the most general unifier (MGU). As it is well-known~\cite{StructuralMGU} all standard recursive algorithms for calculating MGU are not decreasing on argument terms, so we can't define them as simple recursive functions in \textsc{Coq} due to the termination check failure. The standard approach to tackle this problem is to define the function through well-founded recursion. We use a distinctive version of this approach, which is more convenient for our purposes: we define MGU as a proposition (for which do not have the termination requirement in \textsc{Coq}) with a dedicated structurally-recursive function for one step of unification, and then we use a well-founded induction to prove the existence of a corresponding result for any arguments and defining properties of MGU. For this well-founded induction, we use the number of distinct free variables in argument terms as a well-founded order on pairs of terms.

In operational semantics, to define traces as (possibly) infinite sequences of transitions we use the standard approach in \textsc{Coq}~--- coinductively defined streams. Operating with them requires a number of well-known tricks, described in~\cite{CPDT}, to be applied, such as the use of a separate coinductive definition of equality on streams.

The final proofs of soundness and completeness of operational semantics are relatively small, but the large amount of work is hidden in the proofs of auxiliary facts that they use (including lemmas from the previous sections and some technical machinery for handling representing functions).

\section{Applications}

In addition to verification of correctness of different implementations of disequality constraints we can use the extended framework to formally
state and prove some of its other important properties. Thanks to the completeness result, we can do it in the denotational context,
where the reasoning is much easier.

For example, we can define meaningless answers with empty interpretation, which we pointed out for Implementation A from the previous section,
and prove their absence for Implementation B.

So, for implementation B the following holds:

\begin{lemma}
If all free variables in a goal $g$ belong to the set $\{\alpha_1,\dots,\alpha_n\}$, then

\[ \forall (\sigma, \cstore_\sigma, n_r) \in Tr_{\inbr{g, \epsilon, \cstoreinit, n}}, \quad \sembr{\sigma} \cap \sembr{\cstore_\sigma} \neq \emptyset \]
\end{lemma}

It is based on the following lemma about combining constraints, which we can prove only when there are infinitely many constructors in the language (otherwise it is not true):

\begin{lemma}
If for a finite constraint store $\cstore_\sigma$
\[ \forall \omega \in \cstore_\sigma,  \sembr{\sigma} \cap \sembr{\omega} \neq \emptyset, \]
then
\[ \sembr{\sigma} \cap \sembr{\cstore_\sigma} \neq \emptyset. \]
\end{lemma}

Another example is the justification of optimizations in constraint store implementation. For example, the following obvious (in denotational context) statement
allows deleting subsumed constraints in Implementation B:

\begin{lemma}
For any constraint store $\cstore_\sigma$ and two constraint substitutions $\omega$ and $\omega'$, if

\[ \exists \tau, \omega' = \omega \tau \]

then

\[ \sembr{\cstore_\sigma \cup \{\omega, \omega'\}} = \sembr{\cstore_\sigma \cup \{\omega\}}. \]
\end{lemma}

\section{Related Works}

The study of formal semantics for logic programming languages, in the first place \textsc{Prolog}, is a well-established research domain. Early
works~\cite{JonesMycroftSemantics,DebrayMishraSemantics} addressed the computational aspects of both pure \textsc{Prolog} and its extension
with the cut construct. Recently, the application of certified/mechanized approaches came into focus as well. In particular,
in~\cite{CertifiedPrologEquivalences} the equivalence of a few differently defined semantics
for pure \textsc{Prolog} is proven, and in~\cite{CeritfiedDenotationalCut} a denotational semantics for \textsc{Prolog} with cut is presented; both
works provide \textsc{Coq}-mechanised proofs. It is interesting that the former one also advocates the use of higher-order
abstract syntax. We are not aware of any prior works on certified semantics for \textsc{Prolog} which contributed a correct-by-construction
interpreter. Our certified description of SLD resolution with cut can be considered as a certified semantics for \textsc{Prolog} modulo
occurs check in unification (which \textsc{Prolog} does not have by default).

The implementation of first-order unification in dependently typed languages constitutes a well-known challenge with a number of
known solutions. The major difficulty comes from the non-structural recursivity of conventional unification algorithms, which
requires to provide a witness for convergence. The standard approach is to define a generally-recursive function and a well-founded order
for its arguments. This route is taken in~\cite{MGUinLCF,MGUinMLTT,IdempMGUinCoq,TextbookMGUinCoq}, where the descriptions of
unification algorithms are given in \textsc{LCF}, \textsc{Alf}, \textsc{Coq} and \textsc{Coq} respectively. As a well-founded
order lexicographically ordered tuples, containing the information about the number of different free variables and the sizes of
the arguments, is used. We implemented a similar approach, but we separated the test for the non-matching case into a dedicated
function. Thus, we make a recursive call only when the current substitution extension is guaranteed, which allows us to use the
number of different free variables as the order. An alternative approach suggested in~\cite{StructuralMGU} gives a structurally recursive definition of
the unification algorithm; this is achieved by indexing the arguments with the numbers of their free variables.

The use of higher-order abstract syntax (HOAS) for dealing with language constructs in \textsc{Coq} was addressed in early work~\cite{HOASinCoq},
where it was employed to describe lambda calculus. The inconsistency phenomenon of HOAS representation, mentioned in Section~\ref{specification}, is called
there ``exotic terms'' and is handled using a dedicated inductive predicate ``\lstinline|Valid_v|''. The predicate has a non-trivial implementation based
on subtle observations on bindings behavior. Our case, however, is much simpler: there is not much variety in ``exotic terms'' (for example, we do not have
reductions in terms), and our consistency predicate can be considered as a limited version of ``\lstinline|Valid_v|'' for a bigger language.

The study of formal semantics for \textsc{miniKanren} is also not a completely novel venture. In~\cite{RelConversion} a non-deterministic
small-step semantics is described, and in~\cite{DivTest} a big-step semantics for a finite number of answers is given;
neither uses proof mechanization and in both works the interleaving is not addressed. 

The work of~\cite{MechanisingMiniKanren} can be considered as our direct predecessor. It also introduces both denotational and
operational semantics and presents a \textsc{HOL}-certified proof for the soundness of the latter w.r.t. the former. The denotational
semantics resembles ours but considers only queries with a single free variable (we do not see this restriction as important).
On the other hand, the operational semantics is nondeterministic (similarly to~\cite{RelConversion}), which makes it
impossible to express interleaving and extract the interpreter in a direct way. In addition, a specific form of ``executable semantics''
is introduced, but its connection to the other two is not established. Finally, no completeness result is presented.
We consider our completeness proof as an essential improvement. 

The most important property of interleaving search~--- completeness~--- was postulated in the introductory paper~\cite{Search}, and is delivered by
all major implementations. In~\cite{2016} a proof of completeness for a specific implementation of \textsc{miniKanren} is presented; however, the completeness is understood there as
preservation of all answers during the interleaving of answer streams, i.e. in a more narrow sense than in our work since no relation
to denotational semantics is established.

\section{Conclusion and Future Work}

In this paper, we presented a certified formal semantics for core \textsc{miniKanren} and proved some of its basic properties
(including interleaving search completeness), which are believed to hold in existing implementations.
We also derived a semantics for conventional SLD resolution with cut and extracted two certified reference interpreters.
We consider our work as the initial setup for the future development of \textsc{miniKanren} semantics.

The language we considered here lacks many important features, which are already introduced
and employed in many implementations. Integrating these extensions~--- in the first hand, disequality constraints,~--- into
the semantics looks a natural direction for future work. We are also going to address the problems of proving some
properties of relational programs (equivalence, refutational completeness, etc.).



%%
%% Bibliography
%%

%% Please use bibtex,

%\renewcommand\bibliographytypesize{\small}
\bibliographystyle{abbrv} %{splncs04}
\bibliography{main}

\appendix
\section{Appendix}
\label{appendix}

In this appendix we present a proof of partial semantic correctness of relational conversion, or, to be precise, 
a number of observations, definitions, and claims, which, we believe, are sufficient to reconstruct
the complete proof. 

We remind, that our goal is to prove the following statement:

\begin{theorem} 
\normalfont For arbitrary functional program $p$ of a ground type $t$, arbitrary value $v$, and
arbitrary variable $x$

$$
\begin{array}{c}
p\leadsto^f v \Rightarrow \lstinline|fresh ($x$) ($\sembr{p}^c x$)| \leadsto^r (\theta, \emptyset)\\
\mbox{and}\\
\theta(\mathfrak{s})=v
\end{array}
$$

\noindent where $\mathfrak{s}$ is a semantic variable, associated with
$x$ on the first step of the relational evaluation.
\end{theorem}
  
We first comment on the empty set as the set of negative substitutions. A disequality constraint can
come only from a polymorphic equality, which is applied when both its operands are reduced to
values. In the relational counterpart, being run in a forward direction, this corresponds to the evaluation of disequality constraints for
closed terms only, which, in turn, means, that they will immediately succeed or fail. Both cases
add nothing to the set of negative substitutions, which is initially empty. 

Next, we cannot prove the theorem, using an induction by a derivation length, since in the case of
application, for example, the type of the term in the head position is not ground. This 
obstacle could be lifted, if we could prove the following generalization:

$$
p\leadsto^f f \Rightarrow \sembr{p}^c\leadsto^r\sembr{f}^c
$$ 

\noindent for arbitrary $p$ of any type. This claim, however, turned out to be false~--- a term
\lstinline|C ((fun x.x) A)| can be taken as an example.  

The origin of the problem is that we \emph{functionalize} the constructors, \lstinline|match|, and
equality expressions, and, hence, change the order of reductions in the relational counterpart in 
comparison with the original functional program. Thus, we need to take this change into account.

First, we develop a modified functional semantics, which corresponds better to the reduction
order in the relational case. We call this semantics \emph{deferred}, as it defers the evaluation
of constructors, \lstinline|match|, and equality expressions. This semantics can be acquired in
two steps: first, we consider a reduced version of the original functional semantics, in which
we treat arbitrary constructor, \lstinline|match|, and equality expressions as values. Then, the
deferred semantics is just an iterative application of the reduced version to the arguments 
of these new values (arguments of constructors or equality operator, or scrutinees of \lstinline|match| 
expressions).

Next, we claim, that if a term of some ground type is reduced to some value by the original semantics,
then it as well is reduced to the same value by the deferred one. This claim is based on the following
observations:

\begin{itemize}
\item progress and type preservation properties for both semantics (which can be proven in a standard
way);
\item Church-Rosser property for lambda-calculus;
\item the fact, that the reduced semantics applies a proper subset of rules of the original one.
\end{itemize}

Now, we are going to prove the theorem by a simulation between the deferred semantics for the original program
and the relational one for the relationally converted. Before that, we formulate the number of lemmas and 
definitions.

\begin{lemma}
\label{stack_split}
\normalfont Let us separate all the contexts into two disjoint kinds: 

\begin{itemize}
\item functional

$$
C_f = \Box\;e\mid v\;\Box\mid\lstinline|let $x$ = $\Box$ in $e$|
$$

\item ground

$$
C_g = \lstinline|match $\;\Box\;$ with $\{p_i$->$e_i\}$|\mid C^n(\bar{v},\Box,\bar{e})\mid\Box\lstinline|=e|\mid\lstinline|v=|\Box
$$

Let $\left<{\mathcal S},\,e\right>$ be an arbitrary state in a derivation sequence w.r.t. the deferred
semantics. Then $\mathcal S=C_f^*C_g^*$.
\end{itemize}

In other words, during the evaluation w.r.t. the deferred semantics, the stack of contexts is separated into the two
(possibly empty) segments: all ground contexts reside below all functional. The proof is by the induction on the
length of derivation sequence.
\end{lemma}

\begin{definition}
\normalfont
We as well separate all terms of the source language into the two disjoint kinds:

\begin{itemize}
\item functional

$$
e_1\,e_2\mid \lambda x.e \mid \mu f.\lambda x.e \mid \lstinline|let $x$ = $e_1$ in $e_2$| \mid \lstinline|let rec $f$ = $\lambda x.e_1$ in $e_2$|
$$

\item ground

$$
e_1 = e_2 \mid \lstinline|match $e$ with {$p_i$ -> $e_i$ }| \mid \lstinline|C$^k$ ($e_1\dots e_k$)|
$$

\end{itemize}

\end{definition}

\begin{definition}
\normalfont Augmented conversion of a term w.r.t. to a substitution $\sembr{\bullet}_\theta$ is defined as follows: 

$$
\begin{array}{rcl}
\sembr{p}_\theta&=&\sembr{p}^c\\
\sembr{v}_\theta&=&(\lambda x.x\equiv\mathfrak{s}),\,\mbox{if}\;\;\theta(\mathfrak s)=v
\end{array}
$$

Here $\theta$ is a substitution, $p$~--- arbitrary functional term, $v$~--- arbitrary value of a
ground type in the sense of the original semantics (i.e. the composition of constructors). Note, the
cases in this definition are not disjoint, and in the second case there can be more, than one
variable with the requested property, so augmented conversion defines a set of relational terms.
\end{definition}

\begin{lemma}
\label{substitution}
\normalfont Let $f$, $e$ be two arbitrary terms of the source language, $\theta$~--- arbitrary
substitution. Then

$$
\sembr{f[x\gets e]}_\theta=\sembr{f}_\theta[x\gets\sembr{e}_\theta]
$$

The equality here is understood in a set-theoretic sense. The proof is by structural 
induction.
\end{lemma}

\begin{definition}
\normalfont For arbitrary substitution $\theta$ define a conversion of a functional context  
$\sembr{\bullet}_\theta$ as follows:

$$
\begin{array}{rcl}
\sembr{\Box\,e}_\theta&=&\Box\,\sembr{e}_\theta\\
\sembr{v\,\Box}_\theta&=&\sembr{v}_\theta\,\Box\\
\sembr{\lstinline|let $\;x\; = \;\Box\;$ in $\;e$|}_\theta&=&\lstinline|let $\;x\; = \;\Box\;$ in $\;\sembr{e}_\theta$|
\end{array}
$$

Here $e$ is an arbitrary functional term, $v$~--- abstraction. This conversion is an extension of augmented
conversion for functional contexts, hence the same denotation.
\end{definition}

\begin{definition}
\normalfont For arbitrary semantic variables ${\mathfrak s}_1$, ${\mathfrak s}_2$ and arbitrary substitution $\theta$ 
define a conversion of ground context $\sembr{\bullet}^{{\mathfrak s}_1{\mathfrak s}_2}_\theta$ as follows:

$$ 
\begin{array}{rcl}
\sembr{C^k(v_1, \ldots, v_{i-1}, \Box, e_{i+1}, \ldots, e_k)}^{{\mathfrak s}_1{\mathfrak s}_2}_\theta&=&\Box \; \wedge \\
       & & (\sembr{e_{i+1}}_\theta \; {\mathfrak s}^\prime_{i+1}) \; \wedge \\
       & & \ldots  \\
       & & (\sembr{e_k}_\theta \; {\mathfrak s}^\prime_k) \; \wedge \\
       & & ({\mathfrak s}_2 \equiv\; \uparrow C^k({\mathfrak s}^\prime_1, \ldots, {\mathfrak s}^\prime_{i-1}, {\mathfrak s}_1, {\mathfrak s}^\prime_{i+1}, \ldots, {\mathfrak s}_k)),\,\mbox{if}\;\theta({\mathfrak s}^\prime_j)=v_j,\,j<i
\end{array}
$$

$$
\begin{array}{rcl}
\sembr{\Box = e}^{{\mathfrak s}_1{\mathfrak s}_2}_\theta&=&\Box\, \wedge \\
 & & (\sembr{e}_\theta\; {\mathfrak s}^\prime) \wedge \\
 & & ((({\mathfrak s}_1 \equiv {\mathfrak s}^\prime) \wedge ({\mathfrak s}_2 \equiv \lstinline|^true|))\, \vee \\ 
 & & (({\mathfrak s}_1 \not \equiv {\mathfrak s}^\prime) \wedge ({\mathfrak s}_2 \equiv \lstinline|^false|))) 
\end{array}
$$

$$
\begin{array}{rcl}
\sembr{v = \Box}^{{\mathfrak s}_1{\mathfrak s}_2}_\theta&=&\Box\,\wedge \\
 & & ((({\mathfrak s}^\prime \equiv {\mathfrak s}_1) \wedge ({\mathfrak s}_2 \equiv \lstinline|^true|))\, \vee \\ 
 & & (({\mathfrak s}^\prime \not \equiv {\mathfrak s}_1) \wedge ({\mathfrak s}_2 \equiv \lstinline|^false|))),\,\mbox{if}\;\theta({\mathfrak s})=v 
\end{array}
$$

$$
\begin{array}{rcl}
\sembr{\lstinline|match $\;\Box\;$ with \{$C^{n_i}_i$($y^i_1$, ..., $y^i_{n_i}$) -> $\;e_i$\}|}^{{\mathfrak s}_1{\mathfrak s}_2}_\theta&=&\Box \; \wedge \bigvee_i\\
& &(\lstinline|fresh ($s^i_1 \ldots s^i_{n_i}$)| \\
& &\qquad({\mathfrak s}_1 \equiv \;\uparrow C_i^{n_i}(s^i_1, \ldots, s^i_{n_i})) \\
& &\qquad(\lambda y^i_1. \ldots \lambda  y^i_{n_i}. \sembr{e_i}_\theta) \; (\equiv s^i_1) \ldots (\equiv s^i_{n_i})\;{\mathfrak s}_2)
\end{array}
$$

Here we assume ${\mathfrak s}^\prime$ and ${\mathfrak s}^\prime_i$ to be arbitrary semantic variables, $v_i$~--- arbitrary values w.r.t. the original 
functional semantics, $e_i$~--- arbitrary terms of the source language. We also claim, that $\theta$ is
undefined for all mentioned semantic variables, unless the opposite is specified explicitly.

\end{definition}

\begin{definition}
\normalfont For arbitrary substitution $\theta$, arbitrary semantic variable ${\mathfrak s}_m$ and a functional 
term $e$ define a conversion of a stack $\sembr{\bullet}^{e,{\mathfrak s}_m}_\theta$ as follows:

$$
\def\arraystretch{1.5}
\sembr{f_n\dots f_1g_m\dots g_1}^{e,{\mathfrak s}_m}_\theta=\left\{
\begin{array}{lcl}
\sembr{g_m}^{{\mathfrak s}_m{\mathfrak s}_{m-1}}_\theta\dots\sembr{g_1}^{{\mathfrak s}_1{\mathfrak s}_0}_\theta&,&n=0\;\;\mbox{and $e$~--- ground}\\
\sembr{f_n}_\theta\dots\sembr{f_1}_\theta(\Box\,{\mathfrak s}_m)\sembr{g_m}^{{\mathfrak s}_m{\mathfrak s}_{m-1}}_\theta\dots\sembr{g_1}^{{\mathfrak s}_1{\mathfrak s}_0}_\theta&,&\mbox{otherwise}
\end{array}
\right.
$$

Here ${\mathfrak s}_0\dots {\mathfrak s}_{m-1}$ designate arbitrary distinct semantic variables.
\end{definition}

\begin{definition}
\normalfont For arbitrary substitution $\theta$ and arbitrary semantic variable ${\mathfrak s}_m$ define a simulation
conversion $\sembr{\bullet}^{{\mathfrak s}_m}_\theta$ of the source language term as follows:

$$
\begin{array}{rcl}
\sembr{e_1 = e_2}^{{\mathfrak s}_m}_\theta&=& (\sembr{e_1}_\theta\; {\mathfrak s}^\prime_1) \wedge \\
                           & & (\sembr{e_2}_\theta\; {\mathfrak s}^\prime_2) \wedge \\
                           & & ((({\mathfrak s}^\prime_1 \equiv {\mathfrak s}^\prime_2) \wedge ({\mathfrak s}_m \equiv \lstinline|^true|))\, \vee \\ 
                           & & (({\mathfrak s}^\prime_1 \not \equiv {\mathfrak s}^\prime_2) \wedge ({\mathfrak s}_m \equiv \lstinline|^false|)))
\end{array}
$$

$$
\begin{array}{rcl}
\sembr{v = e}^{{\mathfrak s}_m}_\theta&=& (\sembr{e}_\theta\; {\mathfrak s}^\prime_2) \wedge \\
                        & & ((({\mathfrak s}^\prime_1 \equiv {\mathfrak s}^\prime_2) \wedge ({\mathfrak s}_m \equiv \lstinline|^true|))\, \vee \\ 
                        & & (({\mathfrak s}^\prime_1 \not \equiv {\mathfrak s}^\prime_2) \wedge ({\mathfrak s}_m \equiv \lstinline|^false|))),\,\mbox{if}\;\theta({\mathfrak s}^\prime_1)=v
\end{array}
$$

$$
\begin{array}{rcl}
\sembr{v_1 = v_2}^{{\mathfrak s}_m}_\theta&=& ((({\mathfrak s}^\prime_1 \equiv {\mathfrak s}^\prime_2) \wedge ({\mathfrak s}_m \equiv \lstinline|^true|))\, \vee \\ 
                           & & (({\mathfrak s}^\prime_1 \not \equiv {\mathfrak s}^\prime_2) \wedge ({\mathfrak s}_m \equiv \lstinline|^false|))),\,\mbox{if}\;\theta({\mathfrak s}^\prime_j)=v_j
\end{array}
$$

$$ 
\begin{array}{rcl}
\sembr{C^k(v_1, \ldots, v_{i-1}, e_i, \ldots, e_k)}^{{\mathfrak s}_m}_\theta&=&(\sembr{e_i}_\theta \; {\mathfrak s}^\prime_i) \; \wedge \\
       & & \ldots  \\
       & & (\sembr{e_k}_\theta \; {\mathfrak s}^\prime_k) \; \wedge \\
       & & ({\mathfrak s}_m \equiv\; \uparrow C^k({\mathfrak s}^\prime_1, \ldots, {\mathfrak s}^\prime_k)),\,\mbox{if}\;\theta({\mathfrak s}^\prime_j)=v_j,\,j<i
\end{array}
$$

$$ 
\sembr{C^k(v_1, \ldots, v_k)}^{{\mathfrak s}_m}_\theta = ({\mathfrak s}_m \equiv\; \uparrow C^k({\mathfrak s}^\prime_1, \ldots, {\mathfrak s}^\prime_k)),\,\mbox{if}\;\theta({\mathfrak s}^\prime_j)=v_j
$$

$$ 
\sembr{C^k(v_1, \ldots, v_k)}^{{\mathfrak s}_m}_\theta = ({\mathfrak s}_m \equiv\; {\mathfrak s}^\prime),\;\mbox{if}\;\theta({\mathfrak s}^\prime)=C^k(v_1, \ldots, v_k)
$$

$$
\begin{array}{rcl}
\sembr{\lstinline|match $\;e\;$ with \{$C^{n_i}_i$($y^i_1$, ..., $y^i_{n_i}$) -> $\;e_i$\}|}^{{\mathfrak s}_m}_\theta&=&\sembr{e}_\theta\;{\mathfrak s}^\prime\;\wedge\;\bigvee_i\\
& &(\lstinline|fresh ($s^i_1 \ldots s^i_{n_i}$)| \\
& &\qquad({\mathfrak s}^\prime \equiv \;\uparrow C_i^{n_i}(s^i_1, \ldots, s^i_{n_i})) \\
& &\qquad(\lambda y^i_1. \ldots \lambda  y^i_{n_i}. \sembr{e_i}_\theta) \; (\equiv s^i_1) \ldots (\equiv s^i_{n_i})\;{\mathfrak s}_m)
\end{array}
$$

$$
\begin{array}{rcl}
\sembr{\lstinline|match $\;v\;$ with \{$C^{n_i}_i$($y^i_1$, ..., $y^i_{n_i}$) -> $\;e_i$\}|}^{{\mathfrak s}_m}_\theta&=&\bigvee_i\\
& &(\lstinline|fresh ($s^i_1 \ldots s^i_{n_i}$)| \\
& &\qquad({\mathfrak s}^\prime \equiv \;\uparrow C_i^{n_i}(s^i_1, \ldots, s^i_{n_i})) \\
& &\qquad(\lambda y^i_1. \ldots \lambda  y^i_{n_i}. \sembr{e_i}_\theta) \; (\equiv s^i_1) \ldots (\equiv s^i_{n_i})\;{\mathfrak s}_m),\,\mbox{if}\;\theta({\mathfrak s}^\prime)=v
\end{array}
$$

Here all ${\mathfrak s}^\prime$ and ${\mathfrak s}^\prime_i$ designate arbitrary semantic variables, $e$~--- arbitrary term, $v$~--- arbitrary value w.r.t. the
original semantics. We also claim, that $\theta$ is undefined for all mentioned semantic variables, unless the opposite is specified explicitly.
\end{definition}

\begin{definition}
\normalfont Let 
\begin{itemize}
\item \mbox{$\left<\mathcal S,\,e\right>$}~--- a state w.r.t. the deferred semantics;
\item \mbox{$\left<\Sigma, \hat{\mathcal S}, \hat{e}, (\theta, \emptyset)\right>$}~--- a state w.r.t. the
relational semantics.
\end{itemize} 

We say, that these states are connected, if there exists a semantic variable $q_m$, such, that:\vspace{1mm}

\begin{enumerate}
\item \mbox{$\hat{\mathcal S}\in\sembr{\mathcal S}^{e,{\mathfrak s}_m}_\theta$}\vspace{1mm}
\item \mbox{$\hat{e}\in\left\{
                          \begin{array}{lcl}
                            \sembr{e}^{{\mathfrak s}_m}_\theta&,&e\mbox{~--- ground and }\mathcal S\mbox{ does not contain functional contexts}\\[1mm]
                            \sembr{e}_\theta&,&\mbox{otherwise}
                          \end{array}
                       \right.
            $} 
\item $\Sigma$ contains all semantic variables from $\hat{e}$, $\hat{\mathcal S}$, and $\theta$.
\end{enumerate}

\end{definition}

\begin{lemma}
\label{constructor}
\normalfont Let $v=\lstinline|C$^k$($v_1$,...,$v_k$)|$ be a value. Then
for arbitrary $\Sigma$, $\mathcal S$, $\theta$, $\hat{v}\in \sembr{v}_\theta$, and 
semantic variable ${\mathfrak s}$, such, that ${\mathfrak s}\not\in dom(\theta)$ either

$$
\left<\Sigma,\,\mathcal S, (\hat{v}\,{\mathfrak s}),\, (\theta,\,\emptyset)\right>\leadsto^*\left<\Sigma^\prime,\,\mathcal S,\,{\mathfrak s}\equiv\lstinline|C$^k$(${\mathfrak s}^\prime_1$,...,${\mathfrak s}^\prime_k$)|,\,(\theta^\prime,\,\emptyset)\right>\;\mbox{and}\;\theta^\prime({\mathfrak s}^\prime_i)=v_i
$$

or

$$
\left<\Sigma,\,\mathcal S, (\hat{v}\,{\mathfrak s}),\, (\theta,\,\emptyset)\right>\leadsto\left<\Sigma,\,\mathcal S,\,{\mathfrak s}\equiv {\mathfrak s}^\prime,\,(\theta,\,\emptyset)\right>\;\mbox{and}\;\theta({\mathfrak s}^\prime)=v
$$
 
The proof is by induction on the height of $v$.
\end{lemma}

\begin{lemma}
\label{evaluation_lemma}
\normalfont Let $s=\left<\mathcal S=g_m\dots g_1,\,e\right>$ be a state w.r.t. the deferred semantics, 
$g_i$~--- ground contexts, $e$~--- expression of a ground type, $\theta$~--- some substitution,
${\mathfrak s}_m$~--- some semantic variable, \mbox{$\hat{\mathcal{S}}\in\sembr{\mathcal S}^{e,\,{\mathfrak s}_m}_\theta$}, 
\mbox{$\hat{e} \in \sembr{e}_\theta$}. Then there is a sequence of steps w.r.t. the relational
semantics, such, that

$$
\left<\Sigma, \hat{\mathcal S}, (\hat{e} \, {\mathfrak s}_m), (\theta,\,\emptyset) \right>\leadsto^*\hat{s}
$$

\noindent and $s$ and $\hat{s}$ are connected. Here we assume $\Sigma$ to contain all semantic variables from
$\hat{\mathcal S}$ and $\theta$. The proof is by case analysis on $e$, using Lemma~\ref{constructor}.
\end{lemma}

\begin{lemma} 
\label{connection}
\normalfont Let \mbox{$s_1 \to s_2$}~--- a single evaluation step w.r.t. the deferred semantics,
$\hat{s_1}$~--- a state of the relational semantics, such, that $s_1$ and $\hat{s_1}$ are connected. Then
there exists a sequence of steps in the relational semantics \mbox{$\hat{s_1}\leadsto^*\hat{s_2}$}, such, 
that $s_2$ and $\hat{s_2}$ are connected. The proof is by case analysis and definition of connection
relation, using Lemmas~\ref{substitution},~\ref{constructor},~\ref{evaluation_lemma}. 
\end{lemma}

\begin{lemma}
\label{prefix}
\normalfont Let $s_0=\left<\emptyset,\,\epsilon,\,\lstinline|fresh ($x$) $(\sembr{e}^c\;x)$|,\,\iota\right>$ be an
initial state of evaluation w.r.t. the relational semantics. Then there is a sequence of steps
\mbox{$s_0\leadsto^*\hat{s}$}, such, that \mbox{$\left<\epsilon,\,e\right>$} (an initial state of
evaluation of $e$ w.r.t. the deferred semantics) and $\hat{s}$ are connected. Immediately follows from
Lemma~\ref{evaluation_lemma}.
\end{lemma}

Now we can prove the partial correctness theorem. Let us have a term $e$ of a ground type in the source language, which
reduces to a value $v=\lstinline|C$^k$($v_1$,...,$v_k$)|$ w.r.t. the original call-by-value semantics. Then it reduces to the same value w.r.t. the
deferred semantics: 

$$
\left<\epsilon,\,e\right>\to^*\left<\epsilon,\,v\right>
$$

By Lemma~\ref{prefix} 

$$
\left<\emptyset,\,\epsilon,\lstinline|fresh ($x$) $(\sembr{e}^c\;x)$|,\iota\right>\leadsto^*\hat{s}
$$

\noindent where \mbox{$\left<\epsilon,\,e\right>$} and $\hat{s}$ are connected. By Lemma~\ref{connection}, there is
a state $\hat{s^\prime}$ w.r.t. the relational semantics, such, that

$$
\hat{s}\leadsto^*\hat{s^\prime}
$$

\noindent and \mbox{$\left<\epsilon,\,v\right>$} and $\hat{s^\prime}$ are connected. By the definition of
the connection relation, $\hat{s^\prime}$ has one of the following forms:

$$
\left<\Sigma,\,\epsilon,\,{\mathfrak s}_0\equiv\lstinline|C$^k$(${\mathfrak s}^\prime_1$,...,${\mathfrak s}^\prime_k$)|,\,(\theta,\,\emptyset)\right>,\,\theta({\mathfrak s}^\prime_i)=v_i
$$

\noindent or

$$
\left<\Sigma,\,\epsilon,\,{\mathfrak s}_0\equiv {\mathfrak s}^\prime,\,(\theta,\,\emptyset)\right>,\,\theta({\mathfrak s}^\prime)=v
$$

\noindent where ${\mathfrak s}_0$ is the first semantic variable, added to $\Sigma$, and \mbox{${\mathfrak s}_0\not\in dom(\theta)$}. In
both cases, we can make the one last step in the relational semantics, which completes the proof. 


\end{document}




