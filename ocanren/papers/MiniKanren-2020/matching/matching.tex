\documentclass[acmlarge]{acmart}
\usepackage[
    type={CC},           % your choice
    modifier={by-sa},    % your choice
    version={4.0},       % your choice
]{doclicense}            % your choice, see \doclicenseThis below


\usepackage{alltt}
%\usepackage{pslatex}
%\usepackage{epigraph}
%\usepackage{verbatim}
\usepackage{latexsym}
\usepackage{array}
%\usepackage{comment}
%\usepackage{makeidx}
%\usepackage{indentfirst}
%\usepackage{verbatim}
%\usepackage{color}
%\usepackage{url}
%\usepackage{xspace}
%\usepackage{hyperref}
%\usepackage{stmaryrd}
\usepackage{amsmath, amsthm}
%\usepackage{graphicx}
%\usepackage{euscript}
\usepackage{mathtools}
%\usepackage{mathrsfs}
%\usepackage{multirow,bigdelim}
%\usepackage{subcaption}
%\usepackage{placeins}
\usepackage{csvsimple}
\usepackage{array}

%\geometry{
%     top=18pt, bottom=14pt, inner=21pt, outer=21pt,
%     paperwidth=5.5in, paperheight=8.5in,
%     }
     
\settopmatter{printacmref=false}
\fancyfoot{}
 
\makeatletter
\def\@formatdoi#1{}
\def\@permissionCodeOne{miniKanren.org/workshop}
\def\@copyrightpermission{\doclicenseThis} 
\def\@copyrightowner{Copyright held by the author(s).}
\makeatother

\copyrightyear{2019}
\setcopyright{rightsretained}

\acmMonth{8}
\acmArticle{3} % your article number, same as in HotCRP



%% Bibliography style
\bibliographystyle{ACM-Reference-Format}
%% Citation style
%% Note: author/year citations are required for papers published as an
%% issue of PACMPL.
\citestyle{acmauthoryear}   %% For author/year citations


%%%%%%%%%%%%%%%%%%%%%%%%%%%%%%%%%%%%%%%%%%%%%%%%%%%%%%%%%%%%%%%%%%%%%%
%% Note: Authors migrating a paper from PACMPL format to traditional
%% SIGPLAN proceedings format must update the '\documentclass' and
%% topmatter commands above; see 'acmart-sigplanproc-template.tex'.
%%%%%%%%%%%%%%%%%%%%%%%%%%%%%%%%%%%%%%%%%%%%%%%%%%%%%%%%%%%%%%%%%%%%%%


%% Some recommended packages.
\usepackage{booktabs}   %% For formal tables:
                        %% http://ctan.org/pkg/booktabs
\usepackage{subcaption} %% For complex figures with subfigures/subcaptions
                        %% http://ctan.org/pkg/subcaption
\usepackage{multirow}

\usepackage{placeins}

\usepackage{listings}
\lstdefinelanguage{ocanren}{
keywords={run, conde, fresh, let, in, match, with, when, class, type,
object, method, of, rec, repeat, until, while, not, do, done, as, val, inherit,
new, module, sig, deriving, datatype, struct, if, then, else, open, private, virtual, include, success, failure,
true, false},
sensitive=true,
commentstyle=\small\itshape\ttfamily,
keywordstyle=\ttfamily\textbf,
identifierstyle=\ttfamily,
basewidth={0.5em,0.5em},
columns=fixed,
mathescape=true,
fontadjust=true,
literate={fun}{{$\lambda$}}1 {->}{{$\to$}}3 {===}{{$\equiv$}}1 {=/=}{{$\not\equiv$}}1 {|>}{{$\triangleright$}}3 {\\/}{{$\vee$}}2 {/\\}{{$\wedge$}}2 {^}{{$\uparrow$}}1,
morecomment=[s]{(*}{*)}
}

\lstset{
%mathescape=true,
%basicstyle=\small,
%identifierstyle=\ttfamily,
%keywordstyle=\bfseries,
%commentstyle=\scriptsize\rmfamily,
%basewidth={0.5em,0.5em},
%fontadjust=true,
language=ocanren
}

\newcommand{\lstquot}[1]{``\lstinline{#1}''}
\newcommand{\sembr}[1]{\llbracket{#1}\rrbracket}
\newcommand\false{$f\!alse$}
\newcommand\myif{i\!f}


\def\transarrow{\xrightarrow}
\newcommand{\setarrow}[1]{\def\transarrow{#1}}

\def\padding{\phantom{X}}
\newcommand{\setpadding}[1]{\def\padding{#1}} 

\def\subarrow{}
\newcommand{\setsubarrow}[1]{\def\subarrow{#1}}

\newcommand{\trule}[2]{\dfrac{#1}{#2}}
\newcommand{\crule}[3]{\dfrac{#1}{#2},\;{#3}}
\newcommand{\withenv}[2]{{#1}\vdash{#2}}
\newcommand{\trans}[3]{{#1}\transarrow{\padding{\textstyle #2}\padding}\subarrow{#3}}
\newcommand{\ctrans}[4]{{#1}\transarrow{\padding#2\padding}\subarrow{#3},\;{#4}}
\newcommand{\llang}[1]{\mbox{\lstinline[mathescape]|#1|}}
\newcommand{\pair}[2]{\inbr{{#1}\mid{#2}}}
\newcommand{\inbr}[1]{\left<{#1}\right>}
\newcommand{\highlight}[1]{\color{red}{#1}}
%\newcommand{\ruleno}[1]{\eqno[\scriptsize\textsc{#1}]}
\newcommand{\ruleno}[1]{\mbox{[\textsc{#1}]}}
\newcommand{\rulename}[1]{\textsc{#1}}
\newcommand{\inmath}[1]{\mbox{$#1$}}
\newcommand{\lfp}[1]{fix_{#1}}
\newcommand{\gfp}[1]{Fix_{#1}}
\newcommand{\vsep}{\vspace{-2mm}}
\newcommand{\supp}[1]{\scriptsize{#1}}
\renewcommand{\sembr}[1]{\llbracket{#1}\rrbracket}
\newcommand{\cd}[1]{\texttt{#1}}
\newcommand{\free}[1]{\boxed{#1}}
\newcommand{\binds}{\;\mapsto\;}
\newcommand{\dbi}[1]{\mbox{\bf{#1}}}
\newcommand{\sv}[1]{\mbox{\textbf{#1}}}
\newcommand{\bnd}[2]{{#1}\mkern-9mu\binds\mkern-9mu{#2}}
\newcommand{\meta}[1]{{\mathcal{#1}}}
\newcommand{\dom}[1]{\mathtt{dom}\;{#1}}
%\newcommand{\primi}[2]{\mathbf{#1}\;{#2}}
\renewcommand{\dom}[1]{\mathcal{D}om\,({#1})}
\newcommand{\ran}[1]{\mathcal{VR}an\,({#1})}
\newcommand{\fv}[1]{\mathcal{FV}\,({#1})}
\newcommand{\tr}[1]{\mathcal{T}r_{#1}}
\newcommand{\diseq}{\not\equiv}
\newcommand{\reprfunset}{\mathcal{R}}
\newcommand{\reprfun}{\mathfrak{f}}
\newcommand{\cstore}{\Omega}
\newcommand{\cstoreinit}{\cstore_\epsilon^{init}}
\newcommand{\csadd}[3]{add(#1, #2 \diseq #3)}  %{#1 + [#2 \diseq #3]}
\newcommand{\csupdate}[2]{update(#1, #2)}  %{#1 \cdot #2}
\newcommand{\primi}[1]{\mathbf{#1}}
\newcommand{\sem}[1]{\llbracket #1 \rrbracket}
\newcommand{\ir}{\ensuremath{\mathcal{S}}}
\usepackage{tikz}
\newcommand*\circled[1]{\tikz[baseline=(char.base)]{
    \node[shape=circle,draw,inner sep=1pt] (char) {#1};}}

\let\emptyset\varnothing
\let\eps\varepsilon

\sloppy 

\newtheorem{Observation}{Observation}

\begin{document}

\title[Relational Synthesis of Pattern Matching]{Relational Synthesis for Pattern Matching}    

\titlenote{This work was partially supported by the grant 18-01-00380 from The Russian Foundation for Basic Research} %% \titlenote is optional;


\author{Dmitry Kosarev}
\email{Dmitrii.Kosarev@pm.me}

\author{Dmitry Boulytchev}
\email{dboulytchev@math.spbu.ru}    

\affiliation{
  \institution{Saint Petersburg State University}
  \country{Russia}                   
}

\affiliation{
  \institution{JetBrains Research}   
  \country{Russia}                   
}


%% Abstract
%% Note: \begin{abstract}...\end{abstract} environment must come
%% before \maketitle command
\begin{abstract}
  We apply relational programming techniques to the problem of synthesizing efficient implementation for a pattern matching construct. Although in principle
  pattern matching can be implemented in a trivial way, the result suffers from inefficiency in terms of both performance and code size. Thus, in implementing functional languages alternative, more elaborate  approaches are widely used. However, as there are multiple kinds and flavors of pattern
  matching constructs, these approaches have to be specifically developed and justified for each concrete inhabitant of the pattern matching ``zoo.'' We formulate the
  pattern matching synthesis problem in relational terms and develop optimizations which improve the efficiency of the synthesis and guarantee the
  optimality of the result. Our approach is based on relational representations of both the high-level semantics of pattern matching and the semantics of
  an intermediate-level implementation language. This choice make our approach, in principle, more scalable as we only need to modify the high-level semantics in order
  to synthesize the implementation of a new feature. Our evaluation on a set of small samples, partially taken from existing literature shows, that our framework is
  capable of synthesizing optimal implementations quickly. Our in-depth stress evaluation on a number of artificial benchmarks, however,
  has shown the need for future improvements.
\end{abstract}


%% 2012 ACM Computing Classification System (CSS) concepts
%% Generate at 'http://dl.acm.org/ccs/ccs.cfm'.
\begin{CCSXML}
<ccs2012>
<concept>
<concept_id>10011007.10011006.10011008.10011009.10011015</concept_id>
<concept_desc>Software and its engineering~Constraint and logic languages</concept_desc>
<concept_significance>500</concept_significance>
</concept>
<concept>
<concept_id>10011007.10011006.10011041.10011047</concept_id>
<concept_desc>Software and its engineering~Source code generation</concept_desc>
<concept_significance>500</concept_significance>
</concept>
</ccs2012>
\end{CCSXML}

\ccsdesc[500]{Software and its engineering~Constraint and logic languages}
\ccsdesc[500]{Software and its engineering~Source code generation}
%% End of generated code


%% Keywords
%% comma separated list
\keywords{relational programming, relational interpreters, pattern matching}  %% \keywords are mandatory in final camera-ready submission


%% \maketitle
%% Note: \maketitle command must come after title commands, author
%% commands, abstract environment, Computing Classification System
%% environment and commands, and keywords command.
\maketitle

\thispagestyle{empty}

\section{Introduction}
\label{sec:intro}

Verifying a solution for a problem is much easier than finding one~--- this common wisdom can be confirmed by anyone who used 
both to learn and to teach. This observation can be justified by its theoretical applications, thus being more than informal knowledge. For example, let us have a language $\mathcal{L}$. If there is a predicate $V_\mathcal{L}$ such~that
\[
\forall\omega\;:\;\omega\in\mathcal{L}\;\Longleftrightarrow\;\exists p_\omega\;:\;V_\mathcal{L}(\omega,p_\omega)
\]
(with $p_\omega$ being of size, polynomial on $\omega$) and we can recognize $V_\mathcal{L}$ in a polynomial time, then we call $\mathcal{L}$ to be in the class $NP$~\cite{Garey:1990:CIG:574848}. Here $p_\omega$ plays role of a justification (or proof) for the fact $\omega\in\mathcal{L}$. For example, if
$\mathcal{L}$ is a language of all hamiltonian graphs, then $V_\mathcal{L}$ is a predicate which takes a graph $\omega$ and some path $p_\omega$ and verifies that $p_\omega$ is indeed a hamiltonial path in $\omega$. The implementation of the predicate $V_\mathcal{L}$, however, tells us very little about the \emph{search procedure} which would calculate $p_\omega$ as a function of $\omega$. For the whole class of $NP$-complete problems no polynomial search procedures are known, and their existence at all is a long-standing problem in the complexity theory.

There is, however, a whole research area of \emph{relational interpreters}, in which a very close problem is addressed. Given a language $\mathcal{L}$, its \emph{interpreter} is a function \lstinline|eval$_\mathcal{L}$| which takes a program $p^\mathcal{L}$ in the language $\mathcal{L}$ and an input $i$ and calculates some output such that
\[
\mbox{\lstinline|eval$_\mathcal{L}$|}(p^\mathcal{L}, i)=\sembr{p^\mathcal{L}}_{\mathcal L}\,(i)
\]
where $\sembr{\bullet}_{\mathcal L}$ is the semantics of the language $\mathcal{L}$. In these terms, a verification predicate $V_\mathcal{L}$ can be
considered as an interpreter which takes a program $\omega$, its input $p_\omega$ and returns $true$ or \false. A \emph{relational} interpreter is an interpreter which is implemented not as a function \lstinline|eval$_\mathcal{L}$|, which calculates the output from a program and its input, but as a relation \lstinline|eval$^o_\mathcal{L}$|
which connects a program with its input and output. This alone would not have much sense, but if we allow the arguments of \lstinline|eval$^o_\mathcal{L}$|
to contain \emph{variables} we can consider relational interpreter as a generic search procedure which determines the values for these variables making the
relation hold. Thus, with relational interpreter it is possible not only to calculate the output from an input, but also to run a program in 
an opposite ``direction'', or to synthesize a program from an input-output pair, etc. In other words, relational verification predicate is capable
(in theory) to both \emph{verify} a solution and \emph{search} for it.

Implementing relational interpreters amounts to writing it in a relational language. In principle, any conventional language for logic programming
(Prolog~\cite{lozov:prolog}, Mercury~\cite{somogyi1996execution}, etc.) would make the job. However, the abundance of extra-logical features and the incompleteness of default search
strategy put a number of obstacles on the way. There is, however, a language specifically designed for pure relational programming, and, in a
narrow sense, for implementing relational interpreters~--- \textsc{miniKanren}~\cite{lozov:TheReasonedSchemer}. Relational interpreters, implemented
in \textsc{miniKanren}, demonstrate all their expected potential: they can synthesize programs by example, search for errors in partially defined programs~\cite{lozov:seven}, produce self-evaluated programs~\cite{lozov:quines}, etc. However, all these results are obtained for a family
of closely related Scheme-like languages and require a careful implementation and even some \emph{ad-hoc} optimizations in the relational
engine. 

From a theoretical standpoint a single relational interpreter for a Turing-complete language is sufficient: indeed, any other interpreter
can be turned into a relational one just by implementing it in a language, for which relational interpreter already exists. However, the overhead
of additional interpretation level can easily make this solution impractical. The standard way to tackle the problem is partial evaluation or specialization~\cite{jones1993partial}.
A \emph{specializer} \lstinline|spec$_\mathcal{M}$| for a language $\mathcal{M}$ for any program $p^\mathcal{M}$ in this language and its partial input $i$ returns some program which, being applied to the residual input $x$, works exactly as the original program on both $i$ and~$x$:
\[
\forall x\;:\;\sembr{\mbox{\lstinline|spec$_\mathcal{M}$|}\,(p^\mathcal{M}, i)}_\mathcal{M}\,(x)=\sembr{p^\mathcal{M}}_\mathcal{M}\,(i, x).
\]

If we apply a specializer to an interpreter and a source program, we obtain what is called \emph{the first Futamura projection}~\cite{futamura1971partial}:
\[
\forall i\;:\; \sembr{\mbox{\lstinline|spec$_\mathcal{M}$|}\,(\mbox{\lstinline|eval$^\mathcal{M}_\mathcal{L}$|}, p^\mathcal{L})}_\mathcal{M}\,(i)=\sembr{\mbox{\lstinline|eval$^\mathcal{M}_\mathcal{L}$|}}_\mathcal{M}\,(p^\mathcal{L}, i).
\]
Here we added an upper index $\mathcal{M}$ to \lstinline|eval$_\mathcal{L}$| to indicate that we consider it as a program in 
the language $\mathcal{M}$. In other words, the first Futamura projection specializes an interpreter for a concrete program, 
delivering the implementation of this program in the language of interpreter implementation. An important property of
a specializer is \emph{Jones-optimality}~\cite{jones1993partial}, which holds when it is capable to completely
eliminate the interpretation overhead in the first Futamura projection. In our case $\mathcal{M}=\mbox{\textsc{miniKanren}}$, 
from which we can conclude that in order to eliminate the interpretation overhead we need a Jones-optimal specializer for \textsc{miniKanren}. 
Although implementing a Jones-optimal specializer is not an easy task even for simple functional languages, there is a Jones-optimal specializer for a logical language~\cite{leuschel2004specialising}, but not for \textsc{miniKanren}. 

The contribution of this paper is as follows:

\begin{itemize}
\item We demonstrate the applicability of relational programming and, in particular, relational interpreters for the task of
turning verifiers into solvers.
\item To obtain a relational verifier from a functional specification we apply \emph{relational conversion}~\cite{lozov:miniKanren,lozov:conversion}~---
a technique which for a first-order functional program directly constructs its relational counterpart. Thus, we introduce a number
of new relational interpreters for concrete search problems.
\item We employ supercompilation in the form of conjunctive partial deduction (CPD)~\cite{de1999conjunctive} to
eliminate the redundancy of a generic search algorithm caused by partial knowledge of its input.
\item We give a number of examples and perform an evaluation of various solutions for the approach we address.
\end{itemize}

Both relational conversion and conjunctive partial deduction are done in an automatic manner. The only thing one needs to specify is the known arguments or the execution direction of a relation. 

As concrete implementation of \textsc{miniKanren} we use \textsc{OCanren}~\cite{lozov:ocanren}~--- its embedding in \textsc{OCaml}; we use
\textsc{OCaml} to write functional verifiers; our prototype implementation of conjunctive partial deduction is written in \textsc{Haskell}.

The paper is organized as follows. In Section~\ref{sec:example} we give a complete example of solving a concrete problem~--- searching for
a path in a graph,~--- with relational verifier. Section \ref{sec:conversion} recalls the cornerstones of relational programming in 
\textsc{miniKanren} and the relational conversion technique. In Section~\ref{sec:cpd} we describe how conjunctive partial deduction was 
adapted for relational programming. Section~\ref{sec:eva} presents the evaluation results for concrete solvers built using the technique
in question. The final section concludes.

\section{The Pattern Matching Synthesis Problem}

We start from a simplified view on pattern matching which does not incorporate some practically important aspects of the construct such as
name bindings in patterns, guards or even semantic actions in branches. In a purified form, however, it  represents the essence of pattern
matching as an ``inspect-and-branch'' procedure. Once we come up with the solution for the essential part of the problem we embellish it with
other features until it reaches a complete form.

First, we introduce a finite set of \emph{constructors} $\mathcal C$, equipped with arities, a set of values $\mathcal{V}$
and a set of patterns $\mathcal{P}$:
 
\[
 \begin{array}{rcll}
    \mathcal{C} & = & \{ C_1^{k_1}, \dots, C_n^{k_n} \}\\
    \mathcal{V} & = & \mathcal{C}\,\mathcal{V}^*\\  
    \mathcal{P} & = & \_ \mid \mathcal{C}\,\mathcal{P}^*
 \end{array}
\]

We define a matching of a value $v$ (\emph{scrutinee}) against an ordered non-empty sequence of patterns $p_1,\dots,p_k$ by means of the following
relation

\[
\setarrow{\xrightarrow}
\trans{\inbr{v;\,p_1,\dots,p_k}}{}{i},\,1\le i\le k+1
\]

which gives us the index of the leftmost matched pattern or $k+1$ if no such pattern exists. We use an auxiliary relation $\inbr{;}\subseteq\mathcal{V}\times\mathcal{P}$
to specify the notion of a value matched by an individual pattern (see Fig.~\ref{fig:match1pat}). The rule \ruleno{Wildcard} says that
a wildcard pattern ``\_'' matches any value, and \ruleno{Constructor} specifies that a constructor pattern matches exactly those values which
have the same constructor at the top level and all subvalues matched by corresponding subpatterns. The definition of ``$\xrightarrow{}{\!\!}$'' is
shown on Fig.~\ref{fig:matchpatts}. An auxiliary relation
 ``$\xrightarrow{}{}_{\!\!*}$'' 
is introduced to specify the left-to-right matching strategy, and we
use current index as an environment. An important rule, $\ruleno{MatchOtherwise}$ specifies that if we exhausted all the patterns with no matching we stop with
the current index (which in this case is equal to the number of patterns plus one).

\begin{figure}
   \renewcommand*{\arraystretch}{2}
   \[
   \begin{array}{cr}
     \inbr{v;\,\_} & \ruleno{Wildcard} \\
     \trule{\forall i\;\inbr{v_i;\,p_i}}{\inbr{C^k\,v_1\dots v_k;\,C^k\,p_1\dots p_k}},\,k\ge 0 & \ruleno{Constructor}
   \end{array}
   \]
   \caption{Matching against a single pattern}
   \label{fig:match1pat}
\end{figure}

\begin{figure}
   \renewcommand*{\arraystretch}{3}
   \setarrow{\xrightarrow}
   \setsubarrow{_*}
   \[
   \begin{array}{cr}
     \trule{\inbr{v;\,p_1}}
           {\withenv{i}{\trans{\inbr{v;\,p_1,\dots,p_k}}{}{i}}} & \ruleno{MatchHead}\\
     \trule{\neg\inbr{v;\,p_1}\qquad\withenv{i+1}{\trans{\inbr{v;\,p_2,\dots,p_k}}{}{j}}}
           {\withenv{i}{\trans{\inbr{v;\,p_1,\dots,p_k}}{}{j}}} & \ruleno{MatchTail}\\
     \withenv{i}{\trans{\inbr{v;\,\varepsilon}}{}{i}} & \ruleno{MatchOtherwise}\\
     \trule{\withenv{1}{\trans{\inbr{v;\,p_1,\dots,p_k}}{}{i}}}
           {\setsubarrow{}\trans{\inbr{v;\,p_1,\dots,p_k}}{}{i}} & \ruleno{Match}
   \end{array}
   \]
   \caption{Matching against an ordered sequence of patterns}
   \label{fig:matchpatts}
\end{figure}

The relation ``$\xrightarrow{}{}\!\!$'' gives us a \emph{declarative} semantics of pattern matching. Since we are interested in
synthesizing implementations, we need a \emph{programmatical} view on the same problem. Thus, we introduce a language $\mathcal S$
(the ``switch'' language) of test-and branch constructs:

\[
\begin{array}{rcl}
  \mathcal M & = & \bullet \\
  &   & \mathcal M\,[\mathbb{N}] \\
  \ir & = & \primi{return}\,\mathbb{N} \\
  &   & \primi{switch}\;\mathcal{M}\;\primi{with}\; [\mathcal{C}\; \primi{\rightarrow}\; \ir]^*\;\primi{otherwise}\;\ir
\end{array}
\]
 
Here $\mathcal{M}$ stands for a \emph{matching expression}, which is either a reference to a scrutinee ``$\bullet$'' or
an indexed subexpression of scrutinee. Programs in the switch language can discriminate based on the
structure of matching expressions, testing their top-level constructors and eventually returning natural numbers as results.
The switch language is similar to the intermediate representations for pattern matching code used in 
previous works on pattern matching implementation~\cite{maranget2001,maranget2008}.

The semantics of the switch language is given by mean of relations ``$\xrightarrow{}{}_{\!\!\!\mathcal M}$'' and ``$\xrightarrow{}{}_{\!\!\mathcal S}$''
(see Fig.~\ref{fig:matchexpr} and \ref{fig:test-and-branch}). The first one describes the semantics of matching expression, while
the second describes the semantics of the switch language itself. In both cases the scrutinee $v$ is used as an environment ($v\vdash$).


\begin{figure}
  \renewcommand*{\arraystretch}{2}
  \setarrow{\xrightarrow}
  \setsubarrow{_{\mathcal M}}
  \[
  \begin{array}{cr}
    \withenv{v}{\trans{\bullet}{}{v}} & \ruleno{Scrutinee} \\
    \trule{\withenv{v}{\trans{m}{}{C^k v_1\dots v_k}}}{\withenv{v}{\trans{m[i]}{}{v_i}}} & \ruleno{SubMatch} 
  \end{array}
  \]
  \caption{Semantics of matching expression}
  \label{fig:matchexpr}
\end{figure}

\begin{figure}
  \renewcommand*{\arraystretch}{3}
  \setarrow{\xrightarrow}
  \setsubarrow{_{\mathcal S}}
  \[
  \begin{array}{cr}
    \withenv{v}{\trans{\primi{return}\;i}{}{i}} & \ruleno{Return}\\
    \trule{{\setsubarrow{_{\mathcal M}}\withenv{v}{\trans{m}{}{C^k\ v_1 \dots v_k}}}\qquad \withenv{v}{\trans{s}{}{i}}}
          {\withenv{v}{\trans{\primi{switch}\;m\;\primi{with}\;[C^k\to s]s^*\;\primi{otherwise}\;s^\prime}{}{i}}} & \ruleno{SwitchMatched}\\
    \trule{{\setsubarrow{_{\mathcal M}}\withenv{v}{\trans{m}{}{D^n\  v_1 \dots v_n}}}\qquad C^k\ne D^n\qquad \withenv{v}{\trans{\primi{switch}\;m\;\primi{with}\;s^*\;\primi{otherwise}\;s^\prime}{}{i}}}
          {\withenv{v}{\trans{\primi{switch}\;m\;\primi{with}\;[C^k\to s]s^*\;\primi{otherwise}\;s^\prime}{}{i}}} & \ruleno{SwitchNotMatched}\\
    \trule{\withenv{v}{\trans{s}{}{i}}}{\withenv{v}{\trans{\primi{switch}\;m\;\primi{with}\;\varepsilon\;\primi{otherwise}\;s}{}{i}}} & \ruleno{SwitchOtherwise}
  \end{array}
  \]
  \caption{Semantics of switch programs}
  \label{fig:test-and-branch}
\end{figure}

The following observations can be easily proven by structural induction.

\begin{Observation}
  For arbitrary pattern the set of matching values is non-empty:

  \[
  \forall p\in\mathcal P : \{v\in\mathcal V\mid \inbr{v;\,p}\}\ne\emptyset
  \]
\end{Observation}

\begin{Observation}
  Relations ``$\xrightarrow{}{}\!\!$'' and ``$\xrightarrow{}{}_{\!\!\mathcal S}$'' are functional and deterministic respectively:

  \[
  \setarrow{\xrightarrow}
  \begin{array}{rcl}
    \forall p_1,\dots,p_k\in\mathcal P,\,\forall v\in \mathcal V,\,\forall \pi\in\mathcal S & : & |\{i\in\mathbb N\mid \trans{\inbr{v;\,p_1,\dots,p_k}}{}{i}\}|=1 \\
                                                                 &  & {\setsubarrow{_{\mathcal S}}|\{i\in\mathbb N\mid \withenv{v}{\trans{\pi}{}{i}}\}|\le 1}
  \end{array}
  \]
\end{Observation}

With these definitions, we can formulate the \emph{pattern matching synthesis problem} as follows: for a given ordered sequence of patterns $p_1,\dots,p_k$ find
a switch program $\pi$, such that

\[
\setarrow{\xrightarrow}
\forall v\in \mathcal V,\; \forall 1\le i\le n+1 : \trans{\inbr{v;\,p_1,\dots,p_n}}{}{i}\Longleftrightarrow{\setsubarrow{_{\mathcal S}}\withenv{v}{\trans{\pi}{}{i}}}\eqno{(\star)}
\]

In other words, program $\pi$ delivers a correct and complete implementation for pattern matching.

\section{Pattern Matching Synthesis, Relationally}
\label{sec:relationally}

In this section we describe a relational formulation for the pattern matching synthesis problem. Practically,
this amounts to constructing a goal with a free variable corresponding to the switch program to synthesize
for (arbitrary) list of patterns. In order to come up with a tractable goal certain steps have to be performed.
We first describe the general idea, and then consider these steps is details.

Our idea of using relational programming for pattern matching synthesis is based on the following observations:

\begin{itemize}
\item For the switch language we can implement a relational interpreter $eval^o_\ir$ with the following property: for
  arbitrary $v\in\mathcal V$, $\pi\in\ir$ and $i\in\mathbb N$
 
  \[
  \setarrow{\xrightarrow}
  \setsubarrow{_\ir}
   eval^o_\ir\, v\, \pi\, i \Longleftrightarrow \withenv{v}{\trans{\pi}{}{i}}
  \]

  In other words, $eval^o_\ir$ interprets a program $\pi$ for a scrutinee $v$ and returns exactly the same branch (if any)
  which is prescribed by the semantics of the switch language. 
  
\item On the other hand, we can directly encode the declarative semantics of pattern matching as a relational
  program $match^o$ such that for arbitrary $v\in\mathcal V$, $p_i\in\mathcal P$ and $i\in\mathbb N$

  \[
  \setarrow{\xrightarrow}
  match^o\,v\,p_1,\dots,p_k\,i \Longleftrightarrow \trans{\inbr{v;\,p_1,\dots,p_k}}{}{i}
  \]

  Again, $match^o$ succeeds with $1\le i\le k$ iff $p_i$ is the leftmost pattern, matching $v$; otherwise it
  succeeds with $i=k+1$.
\end{itemize}

We address the construction of relational interpreters for both semantics in Section~\ref{sec:relints}.

Being relational, both $eval^o_\ir$ and $match^o$ do not just succeed or fail for ground arguments, but also can be \emph{queried} for
arguments with free logical variables, thus performing a search for all substitutions for these variables which make the
relation hold. This observation leads us to the idea of utilizing the definition of the pattern matching
synthesis problem, replacing ``$\xrightarrow{}{}\!\!$'' with $match^o$, ``$\xrightarrow{}{}_{\!\!\!\mathcal S}$`` with $eval^o$,
and $\pi$ with a free logical variable $\circled{?}$, which gives us the goal

\[
\forall v\in \mathcal V,\; \forall 1\le i\le n+1 : match^o\,v\,p_1,\dots,p_n\,i\Longleftrightarrow eval^o\,v\,\circled{?}\,i
\]

\noindent This goal, however, is problematic from relational point of view due to a number of reasons.

First, \textsc{miniKanren} provides rather a limited support for universal quantification. Apart from being inefficient from
a performance standpoint, existing implementations either do not coexist with disequality constraints~\cite{eigen}
or do not support quantified goals with infinite number of answers~\cite{moiseenko}. As we will see below, both restrictions are
violated in our case. Second, there is no direct support for the equivalence of goals (``$\Leftrightarrow$''). Thus,
reducing the original synthesis problem to a viable relational goal involves some ``massaging''.

We eliminate the universal quantification over the infinite set of scrutinees, replacing it by a \emph{finite}
conjunction over a \emph{complete set of samples}. For a sequence of patterns $p_1,\dots,p_k$ a
complete set of samples is a finite set of values $\mathcal{E}(p_1,\dots,p_k)\subseteq\mathcal{V}$ with the following
property:

\[
\setarrow{\xrightarrow}
\begin{array}{rcl}
  \forall\pi\in\mathcal S & : & [\forall v\in\mathcal{E}(p_1,\dots,p_k),\,\forall i\in\mathbb{N} : \trans{\inbr{v;\,p_1,\dots,p_k}}{}{i} \Longleftrightarrow {\setsubarrow{_{\mathcal S}}\withenv{v}{\trans{\pi}{}{i}}}] \Longrightarrow \\
                          &   & [\forall v\in\mathcal V,\,\forall i\in\mathbb{N} : \trans{\inbr{v;\,p_1,\dots,p_k}}{}{i} \Longleftrightarrow  {\setsubarrow{_{\mathcal S}}\withenv{v}{\trans{\pi}{}{i}}}]
\end{array}
\]

In other words, if a program implements a correct and complete pattern matching for all values in a complete set of samples, then this
program implements a correct and complete pattern matching for all values. The idea of using a complete set of samples originates from the following observation: each pattern
describes a (potentially infinite) set of values, and pattern matching splits the set of all values into equivalence classes, each corresponding to a certain matching pattern.
Moreover, the values of different classes can be distinguished only by looking down to a \emph{finite} depth (as different patterns can be distinguished in this way).
The generation of complete sample set will be addressed below (see Section~\ref{sec:samples}).

\setarrow{\xrightarrow}

To eliminate the universal quantification over the set of answers we rely on the functionality of declarative pattern matching semantics. Indeed, given a fixed sequence $p_1,\dots,p_k$
of patterns, for every value $v$ there is exactly one answer value $i$, such that $\trans{\inbr{v;\,p_1,\dots,p_k}}{}{i}$. We can reformulate this property as

\[
\exists i:\, \trans{\inbr{v;\,p_1,\dots,p_k}}{}{i} \Longrightarrow  
\big(\forall j : \trans{\inbr{v;\,p_1,\dots,p_k}}{}{j} \Longrightarrow  j=i\large\big)
\]

Thus, we can replace universal quantification over the sets of answers by existential one, for which we have an efficient relational counterpart~--- the ``\lstinline|fresh|''
construct.

Following the same argument, we may replace the equivalence with conjunction: indeed, if

\[
\setarrow{\xrightarrow}
\trans{\inbr{v;\,p_1,\dots,p_k}}{}{i}
\]

for some $i$, then (by functionality), for any other $j\ne i$

\[
\setarrow{\xrightarrow}
\neg\;(\trans{\inbr{v;\,p_1,\dots,p_k}}{}{j})
\]

A correct pattern matching implementation $\pi$ should satisfy the condition

\[
\setarrow{\xrightarrow}
\setsubarrow{_{\mathcal S}}
\withenv{v}{\trans{\pi}{}{i}}
\]

But, by the determinism of the switch language semantics, it immediately follows, that for arbitrary $j\ne i$

\[
\setarrow{\xrightarrow}
\setsubarrow{_{\mathcal S}}
\neg\;(\withenv{v}{\trans{\pi}{}{j}})
\]

Alternatively\footnote{\color{red} Reviewer N1 said that passage about bool argument is unclear and may be omitted (or described with more details)}, we could switch to a more explicit relational representation of both semantics, adding an extra boolean argument to
both $eval^o_{\mathcal S}$ and $match^o$ and using the same fresh variable $b$ in the query of interest:

\[
match^o\,v\,p_1,\dots,p_k\,i\,b \wedge eval^o_{\mathcal S}\,v\,\pi\,i\,b
\]

Thus, the goal we eventually came up with is

\[
\bigwedge_{v\in\mathcal{E}\,(p_1,\dots,p_k)}\mbox{\lstinline|fresh ($i$)|}\; \{match^o\,v\,p_1,\dots,p_k\,i \wedge eval^o_{\mathcal S}\,v\,\circled{?}\,i\}
\eqno{(\star\star)}
\]

From relational point of view this is a pretty conventional goal which can be solved by virtually any decent \textsc{miniKanren} implementation in
which the relations $eval^o_{\mathcal S}$ and $match^o$ can be encoded.

Finally, we can make the following important observation. Obviously, any pattern matching synthesis problem has at least one trivial solution.
This, due to the completeness of relational interleaving search~\cite{search,certifiedSemantics}, means that the goal above \emph{can not diverge} with
no results. Actually it is rather easy to see that any pattern matching synthesis problem has \emph{infinitely many} solutions: indeed, having just
one it is always possible to ``pump'' it with superfluous ``$\primi{otherwise}$'' clauses; thus, the goal above is \emph{refutationally
complete}~\cite{WillThesis,DivergenceTest}. These observations justify the totality of our synthesis approach. In Section~\ref{sec:optimization} we show
how we can make it provide optimal solution.

\subsection{Constructing Relational Interpreters}
\label{sec:relints}

In this section we address the implementation of relations $eval^o_{\mathcal S}$ and $match^o$. In principle, it amounts to accurate encoding of
relations 
``$\xRightarrow{}{}\!\!$'' and ``$\xRightarrow{}{}_{\!\!\mathcal S}$'' 
in \textsc{miniKanren} (in our case, \textsc{OCanren}). We, however,
make use of a relational conversion~\cite{conversion} tool, called \textsc{noCanren}, which automatically converts a subset of \textsc{OCaml} into
\textsc{OCanren}. Thus, both interpreters are in fact implemented in \textsc{OCaml} and repeat corresponding inference rules almost
literally in a familiar functional style. For example, functional implementation of a declarative semantics looks like follows:

\begin{lstlisting}
   let rec $\inbr{v;\,p}$ =
     match ($v$, $p$) with
     | (_, Wildcard) -> true
     | ($C^k\;v^*$, $C^k\;p^*$) -> list_all $\inbr{;}$ (list_combine $v^*$ $p^*$)
     | _             -> false

  let $match^o$ $v$ $p^*$ =
    let rec inner $i$ $p^*$ =
      match $p^*$ with
      | []      -> $i$
      | $p$ :: $p^*$ -> if $\inbr{v;\,p}$ then $i$ else inner S($i$) $p^*$
    in inner O $p^*$
\end{lstlisting}

We mixed here the concrete syntax of \textsc{OCaml} and mathematical notation, used in the definition of the relation in question; the actual
implementation only a few lines of code longer. Note, we use here natural numbers in Peano form and custom list processing functions in order
to apply relational conversion later.

Using relational conversion saves a lot of efforts as \textsc{OCanren} specifications tend to be much more verbose; in addition
relational conversion implements some ``best practices'' in relational programming (for example, moves unifications forward in
conjunctions and puts recursive calls last). Finally, it has to be taken into account that relational conversion of pattern matching introduces
disequality constraints.

\subsection{Dealing with a Complete Set of Samples}
\label{sec:samples}

As we mentioned above, a complete set of samples plays an important role in our approach: it allows us to eliminate universal quantification over the
set of all values. As we replace the universal quantifier with a finite conjunction with one conjunct per sample value reducing the size of
the set is an important task. At the present time, however, we build an excessively large (worst case exponential of depth) number of samples. We discuss
the issues with this choice in Section~\ref{sec:eval} and consider developing a more advanced approach as the main direction for
improvement.

Our construction of a complete set of samples is based upon the following simple observations. We simultaneously define the \emph{depth} measure
for patterns and sequences of patterns as follows:

\[
\begin{array}{rcl}
   d\,(p_1,\dots,p_k)     & = & max\, \{ d\,(p_i)\}\\
   d\,(\_)                 & = & 0 \\
   d\,(C^k\,p_1,\dots p_k) & = & 1 + d\,(p_1,\dots,p_k)
\end{array}
\]

\noindent As a sequence of patterns is the single input in our synthesis approach we will call its depth \emph{synthesis depth}.

Similarly, we define the depth of matching expressions

\[
\begin{array}{rcl}
  d_{\mathcal M}\,(\bullet) & = & 1 \\
  d_{\mathcal M}\,(m\,[i]) & = & 1 + d_{\mathcal M}\,(m)\\
\end{array}
\]

and switch programs:

\[
\begin{array}{rcl}
  d_{\mathcal S}\,(\primi{return}\;i)&=&0\\
  d_{\mathcal S}\,(\primi{switch}\;m\;\primi{of}\;C_1\to s_1,\dots,C_k\to s_k\;\primi{otherwise}\;s)&=&max\,\{d_{\mathcal M}\,(m),\,d_{\mathcal S}\,(s_i),\,d_{\mathcal S}\,(s)\}
\end{array}
\]

Informally, the depth of a switch program tells us how deep the program can look into a value. 

From the definition of $\inbr{;}$ it immediately follows that a pattern $p$ can only discriminate values up to its depth $d\,(p)$: changing a value at the depth greater
or equal than $d\,(p)$ cannot affect the fact of matching/non matching. This means that we need only consider switch programs of depth no greater than the synthesis depth.
But for these programs the set of all values with height no greater than the synthesis depth forms a complete set of samples. Indeed, if the height of a value less or
equal to the synthesis depth, then this value is a member of complete set of samples and by definition the behavior of the synthesized program on this value is
correct. Otherwise there exists some value $s$ from the complete set of samples, such that given value can be obtained as an ``extension'' of $s$ at the
depth greater than the synthesis depth. Since neither declarative semantics nor switch programs can discriminate values at this depth, they behavior for a given value
will coincide with the correct-by-definition behavior for  $s$.

The implementation of complete set generation, again, is done using relational conversion. The enumeration of all terms up to a certain depth
can be acquired from a function which calculates the depth of a term: indeed, converting it into a relation and then running with \emph{fixed} depth
and \emph{free} term arguments delivers what we need. Thus, we add an extra conjunct which performs the enumeration of all values to the
relational goal $(\star\star)$, arriving at

\[
depth^o\,v\,n\wedge\mbox{\lstinline|fresh ($i$)|}\; \{match^o\,v\,p_1,\dots,p_k\,i \wedge eval^o_{\mathcal S}\,v\,\circled{?}\,i\}
\eqno{(\star\star\star)}
\]

Here $n$ is a precomputed synthesis depth in Peano form.

\begin{comment}
\begin{figure}[ht]
\begin{subfigure}[t]{0.2\linewidth}
  \[
  \{A^1,\,B^0,\,C^1,\,D^0\}
  \]
\vskip6mm
\caption{Constructors}
\label{fig:constructors}
\end{subfigure}
\hspace{0.5cm}
\begin{subfigure}[t]{0.26\linewidth}
  \[
  \begin{array}{c}
    C^1\,(A^1\,(B^0))\\
    C^1\,(\_)\\
    \_
  \end{array}
\]
\caption{Patterns}
\label{fig:patterns}
\end{subfigure}
\hspace{0.5cm}
\begin{subfigure}[t]{0.33\linewidth}
  \[
  \begin{array}{lcl}
     B             & \mapsto & 2 \\
     D             & \mapsto & 2 \\
     A\, (B)       & \mapsto & 2 \\
     A\, (D)       & \mapsto & 2 \\
     C\, (B)       & \mapsto & 1 \\
     C\, (D)       & \mapsto & 1 \\
     A\, (A\, (B)) & \mapsto & 2 \\
     A\, (A\, (D)) & \mapsto & 2 \\
     A\, (C\, (B)) & \mapsto & 2 \\
     A\, (C\, (D)) & \mapsto & 2 \\
     C\, (A\, (B)) & \mapsto & 0 \\
     C\, (A\, (D)) & \mapsto & 1 \\
     C\, (C\, (B)) & \mapsto & 1 \\
     C\, (C\, (D)) & \mapsto & 1 
  \end{array}
  \]
\caption{Generated samples}
\label{fig:samples}
\end{subfigure}
\caption{Complete set of samples example} 
\label{fig:complete-set-example}
\end{figure}
\end{comment}

\section{Implementation and Optimizations}
\label{sec:optimization}

In this section we address two aspects of our solution: a number of optimizations which make the search more efficient, and
the way it ends up with the optimal solution.

The relational goal in its final form, presented in the previous section, does not demonstrate good performance. Thus, we apply a number
of techniques, some of which require extending the implementation of the search. Namely, we apply the following optimizations:

\begin{itemize}
\item We make use of type information to restrict the subset of constructors which may appear in a certain branch of
  program being synthesized.
\item After a complete set of samples is generated, we use it to put auxiliary constraints on matching expressions. For example,
  if we can detect that a matching expression points to a subexpression of scrutinee which can start with a single constructor (like
  tuples), and prohibit it from being considered during the synthesis.
\item We implement \emph{structural constraints} which allow us to restrict the shape of terms during the search, and
  utilize them to implement pruning.
\end{itemize}

In our formalization we do not make any use of types since as a rule type information does not affect matching. In addition,
utilizing the properties of a concrete type system would make our approach too coupled with this particular type system, hampering
its reusability for other languages. Nevertheless we may use a certain abstraction of type system which would deliver only
that part of information which is essential for our approach to function. Currently, we calculate the type of any matching expression in
the program being synthesized and from this type extract the subset of constructors which can appear when branching on this expression
is performed. The number of these constructors restricts the number of branches which a corresponding $\primi{switch}$ expression can have.
In our implementation we assume the constructor set ordered, and we consider only ordered branches, which restricts branching even more.

Our approach to finding an optimal solution in fact implements branch-and-bound strategy. The birds-eye view of our plan is as follows:

\begin{itemize}
\item We construct a trivial solution, which gives us the first estimation.
\item During the search we prune all partial solutions whose size exceeds current estimation. We can do this due to
  the top-down nature of partial solution construction.
\item When we come up with a better solution we remember it and update current estimation.
\end{itemize}

\noindent This strategy inevitably delivers us the optimal solution since there are only finitely many switch programs, shorter than trivial solution.

In order to implement this strategy we extended \textsc{OCanren} with a new primitive called \emph{structural constraint}, which may
fail on some terms depending on some criterion specified by an end-user. Structural constraints can be seen as a generalization of
some known constraints like \lstinline|absent$^o$| or \lstinline|symbol$^o$|~\cite{Untagged} in existing \textsc{miniKanren} implementations, 
so they can be widely used in solving other problems as well. Note, we could implement other constraints we considered (on the
depth of switch programs, on the type of scrutinee) as structural.
%\footnote{\textcolor{red}{I don't entirely understand this sentence}}.
However, our experience has shown that this leads to
a less efficient implementation. Since these constraints are inherent to the problem, we kept them hardcoded.

\subsection{Reducing the Complete Set of Samples}
\label{sec:reduced-samples}

Although in general our approach requires an exponential number of samples to be generated, in some cases a complete set of examples can be reduced.
For example, for the following pattern matching problem

\[
\begin{array}{l}
\mbox{\lstinline|(_, _ :: _ :: _)|} \\
\mbox{\lstinline|(_, _ :: _)|}
\end{array}
\]

the synthesized program should not investigate the left subtree of the scrutinee since its contents can not alter the behaviour of pattern matching.

The set of admissible matching values $s^\cup$ also can be restricted using the same arguments which we described in Section~\ref{sec:samples}.
This set essentially describes the paths to the ``interesting'' subexpressions of the scrutinee, and it can be computed statically before
the synthesis procedure:

\[
\begin{array}{rcl}
   s\,(m, C\ p_1 \dots p_k)     & = & \{m\}\cup \bigcup\limits_{i=1}^{k} s(m[i], p_i)\\
   s\,(m,\_)                 & = & \varnothing \\
   s^\cup\,(p_1,\dots, p_k) & = & \bigcup\limits_{i=1}^{k} s(\bullet, p_i)
\end{array}
\]

For the example above, the set  $s^\cup$ is

\[
\{\bullet, \bullet[1], \bullet[1][1]\}
\]

The complete set of samples then can be the following 3-element set:

\[
\begin{array}{l}
  \mbox{\lstinline|($\underline{[]}$, [])|}\\
  \mbox{\lstinline|($\underline{[]}$, $\;\underline{42}\;$ :: [])|}\\
  \mbox{\lstinline|($\underline{[]}$, $\;\underline{42}\;$ :: $\;\underline{42}\;$ :: $\;\underline{[]}$)|}
\end{array}
\]

\noindent where underlined expressions are chosen arbitrarily. A straightforward algorithm from the section~\ref{sec:samples} would generate the larger set of $3^2$ examples.

The set $s^\cup$ can be used for sample enumeration in the following manner. During the enumeration we hold current matching expression which will be used to
access current subtree of the sample. If that expression does not belong to $s^\cup$, we can choose an arbitrary inhabitant; if not we enumerate all
possible top-level constructors for this subexpression and recurse. The correctness of this algorithm relies on the fact that if an expression does
not belong to $s^\cup$, then all its extensions also does not belong to $s^\cup$.



\section{Evaluation}
\label{sec:eva}

In this section we present an evaluation of the proposed approach. 
We have implemented several relational interpreters for different search problems which can be found in the repository mentioned before. 
Some of the simpler interpreters demonstrate good performance for different directions on their own and for them CPD transformation is not needed. 
Thus, we will focus on two search problems which are more complex: searching for a path in a graph and searching for a unifier~\cite{lozov:unification} of two terms. 
For each problem we compare four programs.
\begin{enumerate}
    \item The solver generated by the unnesting alone.
    \item The solver generated by the unnesting strategy aimed at backward execution. 
    \item The solver generated by the unnesting and then specialized by conjunctive partial deduction for the backward direction.
    \item The interpretation of the functional verifier with the relational interpreter implemented in Scheme~\cite{lozov:seven}. 
\end{enumerate}

First, let us compare the performance of the solvers for the path searching problem.
The implementation of the functional verifier for this problem is described in Section~\ref{sec:example}. 
We ran the search on a graph with 20 nodes and 30 edges, consequentially
 searching for paths of the length 5, 7, 9, 11, 13, and 15. 
We averaged the execution times over 10 runs of the same query. 
We the limited the execution time by 300 seconds, and if the execution time of some query exceeded the timeout, we put ``>300s'' in the result table and did not request the execution of queries for longer paths. The results are presented in Table~\ref{tab:isPath}. 

We can conclude that the execution time increases with the length of the path to search, which is expected, since with the length of the path the number of the subpaths to be explored is increasing as well.
The solver generated by the unnesting alone and the interpretation with the relational intepreter demonstrate poor performance. 
The first one is due to its inherently inefficient execution in backward direction, while the second suffers from the interpretation overhead. 
Both the unnesting aimed at the backward execution and the solver automatically transformed with conjunctive partial deduction show good performance. 
Conjunctive partial deduction performs more thorough specialization, thus producing more efficient program. 

\begin{table}
\centering
\begin{tabular}{c|c|c|c|c|c|c}
Path length                   & 5      & 7     & 9      & 11      & 13     & 15        \\
\hline\hline
Only conversion               & 0.01s  & 1.39s &  82.13s & >300s     & ---      & ---     \\
\hline
Backward oriented conversion  & 0.01s & 0.37s &  2.68s & 2.91s   & 4.88s    & 10.63s   \\
\hline
Conversion and CPD            & 0.01s  & 0.06s &  0.34s & 2.66s   & 3.65s    & 6.22s  \\
\hline
Scheme interpreter            & 0.80s  & 8.22s & 88.14s & 191.44s & >300s   & ---   \\
\end{tabular}

 \caption{Searching for paths in the graph}
    \label{tab:isPath}
\end{table}

Now let us consider the problem of finding a unifier of two terms which have free variables.
A term is either a variable ($X, Y, \dots$) or some constructor applied to terms ($nil, cons(H, T), \dots$). 
A substitution maps a variable to a term. 
A unifier of two terms $t$ and $s$ is a substitution $\sigma$ which equalizes them: $t \sigma = s \sigma$ by simultaneously substituting the variables for their images.
For example, a unifier of the terms $cons(42, X) \text{ and } cons(Y, cons(Y, nil)) \text{ is a substitution } \{X \mapsto cons(42, nil), Y \mapsto 42\} $.

We implemented a functional verifier which checks if a substitution equalizes two input terms. 
We represent a variable name as a unique Peano number. 
A substitution is represented as a list of terms, in which the index of the term is equal to the variable name to which the term is bound, so the substitution $\{X \mapsto cons(42, nil), Y \mapsto 42\}$ is represented as a list ``\lstinline{[cons(42, nil); 42]}''.
The verifier returns true if the input terms can be unified with the candidate substitution and false otherwise. 

As in the previous problem, we compare four solvers generated for the verifier described. 
With each solver, we search for a unifier of two terms and compare the execution times. 
The time comparison is presented in Table~\ref{tab:uni}. 
The first two rows of each column contain two terms being unified. 
We use uppercase letters from the end of the alphabet ($X, Y, \dots$) to denote variables, lowercase letters from the beginning of the alphabet ($a, b, \dots$) to denote constants (constructors with zero arguments), identifiers which start from the lowercase letter ($f, g,\dots$) to denote constructors.

Note, we compute a unifier for two terms, but not necessarily the most general unifier. 
We can implement the most general unification in \textsc{miniKanren}, but achieving the comparable performance using 
relational verifiers requires additional check that the unifier is indeed the most general. 
We are currently working on the implementation of such relational verifier. 

\begin{table}
\centering
\begin{tabular}{c|c|c|c}
\multirow{ 2}{*}{Terms} & 
f(X, a) & f(a \% b \% nil, c \% d \% nil, L) & f(X, X, g(Z, t))  \\
\cline{2-4} &
f(a, X) & f(X \% XS, YS, X \% ZS) & f(g(p, L), Y, Y)  \\
\hline\hline
Only conversion               & 0.01s  &  >300s & >300s \\
\hline
Backward oriented conversion  & 0.01s  &  0.11s & 2.26s  \\
\hline
Conversion and CPD            & 0.01s  &  0.07s & 0.90s  \\
\hline

Scheme interpreter            & 0.04s  & 5.15s & >300s    \\
\end{tabular}
 \caption{Searching for a unifier of two terms}
    \label{tab:uni}
\end{table}

Here four solvers compare to each other similarly to the previous problem: unnesting demonstrates the worst execution time, relational interpretation in Scheme is a little better, while unnesting aimed at backward execution and conjunctive partial deduction significantly improve the performance. 

There exist pairs of terms, for which either of the solvers fails to compute a unifier under 300 seconds. 
The example of such terms is ``\lstinline{f(A,B,C,A,B,C,D)}'' and ``\lstinline{f(B,C,D,x(R,S),x(a,T),x(Q,b),x(a,b))}''. 
This is caused by how general and declarative the verifier is: there is nothing in it to restrict the search space. 
We can modify the verifier with the additional check to ensure that there are no bound variables in the candidate unifier. 
This modification restricts the search space when there are many variables in the input terms.
But it also changes the semantics of the initial verifier and, as a consequence, the solvers: only idempotent unifiers can be found. 

To sum up, we demonstrated by two examples that it is possible to create problem solvers from verifiers by using relational conversion 
and conjunctive partial deduction. Currently conjunctive partial deduction improves the performance the most, as compared to 
interpreting verifiers with Scheme relational interpreter or doing relational conversion which is solely aimed at backward or 
forward execution.

\section{Conclusion and Future Work}

In this paper, we presented a certified formal semantics for core \textsc{miniKanren} and proved some of its basic properties
(including interleaving search completeness), which are believed to hold in existing implementations.
We also derived a semantics for conventional SLD resolution with cut and extracted two certified reference interpreters.
We consider our work as the initial setup for the future development of \textsc{miniKanren} semantics.

The language we considered here lacks many important features, which are already introduced
and employed in many implementations. Integrating these extensions~--- in the first hand, disequality constraints,~--- into
the semantics looks a natural direction for future work. We are also going to address the problems of proving some
properties of relational programs (equivalence, refutational completeness, etc.).


%\section{Appendix}
%\input{lst} 

\begin{comment}
%% Acknowledgments
\begin{acks}                            %% acks environment is optional
                                        %% contents suppressed with 'anonymous'
  %% Commands \grantsponsor{<sponsorID>}{<name>}{<url>} and
  %% \grantnum[<url>]{<sponsorID>}{<number>} should be used to
  %% acknowledge financial support and will be used by metadata
  %% extraction tools.
  This material is based upon work supported by the
  \grantsponsor{GS100000001}{Russian Foundation for Basic Research}{https://www.rfbr.ru/rffi/eng} under Grant
  No.~\grantnum{GS100000001}{18-01-00380} and by the grant from JetBrains Research. 
  %Any opinions, findings, and
  %conclusions or recommendations expressed in this material are those
  %of the author and do not necessarily reflect the views of the
  %National Science Foundation.
\end{acks}
\end{comment}

\bibliography{references}

\end{document}
