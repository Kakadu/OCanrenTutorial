\documentclass[sigconf]{acmart}

\usepackage{booktabs} % For formal tables
\usepackage{amssymb}
\usepackage{amsmath}
\usepackage{mathrsfs}
\usepackage{mathtools}
\usepackage{multirow}
\usepackage{listings}
\usepackage{indentfirst}
\usepackage{verbatim}
\usepackage{amsmath, amssymb}
\usepackage{graphicx}
\usepackage{xcolor}
\usepackage{url}
\usepackage{stmaryrd}
\usepackage{xspace}
\usepackage{comment}
\usepackage{wrapfig}
\usepackage[caption=false]{subfig}
\usepackage{placeins}
\usepackage{tabularx}
\usepackage{ragged2e}

\newtheorem{theorem}{Theorem}
\newtheorem{lemma}{Lemma}

\def\transarrow{\xrightarrow}
\newcommand{\setarrow}[1]{\def\transarrow{#1}}

\def\padding{\phantom{X}}
\newcommand{\setpadding}[1]{\def\padding{#1}}

\def\subarrow{}
\newcommand{\setsubarrow}[1]{\def\subarrow{#1}}

\newcommand{\trule}[2]{\frac{#1}{#2}}
\newcommand{\crule}[3]{\frac{#1}{#2},\;{#3}}
\newcommand{\withenv}[2]{{#1}\vdash{#2}}
\newcommand{\trans}[3]{{#1}\transarrow{\padding#2\padding}\subarrow{#3}}
\newcommand{\ctrans}[4]{{#1}\transarrow{\padding#2\padding}\subarrow{#3},\;{#4}}
\newcommand{\llang}[1]{\mbox{\lstinline[mathescape]|#1|}}
\newcommand{\pair}[2]{\inbr{{#1}\mid{#2}}}
\newcommand{\inbr}[1]{\left<{#1}\right>}
\newcommand{\highlight}[1]{\color{red}{#1}}
\newcommand{\ruleno}[1]{\eqno[\scriptsize\textsc{#1}]}
\newcommand{\rulename}[1]{\textsc{#1}}
\newcommand{\inmath}[1]{\mbox{$#1$}}
\newcommand{\lfp}[1]{fix_{#1}}
\newcommand{\gfp}[1]{Fix_{#1}}
\newcommand{\vsep}{\vspace{-2mm}}
\newcommand{\supp}[1]{\scriptsize{#1}}
\renewcommand{\G}{\mathfrak G}
\newcommand{\sembr}[1]{\llbracket{#1}\rrbracket}
\newcommand{\cd}[1]{\texttt{#1}}
\newcommand{\miniKanren}{miniKanren\xspace}
\newcommand{\ocanren}{OCanren\xspace}
\newcommand{\free}[1]{\boxed{#1}}
\newcommand{\binds}{\;\mapsto\;}
\newcommand{\dbi}[1]{\mbox{\bf{#1}}}
\newcommand{\sv}[1]{\mbox{\textbf{#1}}}
\newcommand{\bnd}[2]{{#1}\mkern-9mu\binds\mkern-9mu{#2}}

\newcommand{\meta}[1]{{\mathcal{#1}}}
\renewcommand{\emptyset}{\varnothing}

\lstdefinelanguage{ocanren}{
keywords={fresh, let, in, match, with, when, class, type,
object, method, of, rec, repeat, until, while, not, do, done, as, val, inherit,
new, module, sig, deriving, datatype, struct, if, then, else, open, private, virtual, include, success, failure,
true, false},
sensitive=true,
commentstyle=\small\itshape\ttfamily,
keywordstyle=\ttfamily\underbar,
identifierstyle=\ttfamily,
basewidth={0.5em,0.5em},
columns=fixed,
fontadjust=true,
literate={fun}{{$\lambda$}}1 {->}{{$\to$}}3 {===}{{$\equiv$}}1 {=/=}{{$\not\equiv$}}1 {|>}{{$\triangleright$}}3 {\\/}{{$\vee$}}2 {/\\}{{$\wedge$}}2 {^}{{$\uparrow$}}1,
morecomment=[s]{(*}{*)}
}

\lstset{
mathescape=true,
%basicstyle=\small,
identifierstyle=\ttfamily,
keywordstyle=\bfseries,
commentstyle=\scriptsize\rmfamily,
basewidth={0.5em,0.5em},
fontadjust=true,
language=ocanren
}

\usepackage{letltxmacro}
\newcommand*{\SavedLstInline}{}
\LetLtxMacro\SavedLstInline\lstinline
\DeclareRobustCommand*{\lstinline}{%
  \ifmmode
    \let\SavedBGroup\bgroup
    \def\bgroup{%
      \let\bgroup\SavedBGroup
      \hbox\bgroup
    }%
  \fi
  \SavedLstInline
}

\sloppy

\begin{document}

\title{Improving Refutational Completeness\\
of Relational Search via Divergence Test$^*$}

\thanks{$^*\;$This work is supported by RFBR grant No 18-01-00380.}

%\titlenote{Produces the permission block, and
%  copyright information}

\author{Dmitri Rozplokhas}
\affiliation{%
  \institution{St. Petersburg Academic University}
  \streetaddress{Khlopina st., 8-3-А}
  \city{St. Petersburg}
  \state{Russia}
  \postcode{194021}
}
\email{rozplokhas@gmail.com}

\author{Dmitri Boulytchev}
\affiliation{%
  \institution{St. Petersburg State University}
  \streetaddress{Universitetskaya emb., 7-9}
  \city{St. Petersburg}
  \state{Russia}
  \postcode{199034}
}
\email{dboulytchev@math.spbu.ru}

\begin{abstract}
We describe a search optimization technique for implementation of relational programming language
miniKanren which makes more queries converge. Specifically, we address the problem of conjunction
non-commutativity. Our technique is based on a certain divergence criterion that we use to trigger a
dynamic reordering of conjuncts. We present a formal semantics of a miniKanren-like language and prove
that our optimization does not compromise already converging programs, thus, being a proper improvement.
We also present the prototype implementation of the improved search and demonstrate its application for a
number of realistic specifications.
\end{abstract}

%
% The code below should be generated by the tool at
% http://dl.acm.org/ccs.cfm
% Please copy and paste the code instead of the example below.
%
\begin{CCSXML}
<ccs2012>
<concept>
<concept_id>10003752.10003790.10003795</concept_id>
<concept_desc>Theory of computation~Constraint and logic programming</concept_desc>
<concept_significance>500</concept_significance>
</concept>
<concept>
<concept_id>10003752.10010124.10010131.10010134</concept_id>
<concept_desc>Theory of computation~Operational semantics</concept_desc>
<concept_significance>100</concept_significance>
</concept>
<concept>
<concept_id>10011007.10011006.10011008.10011009.10011015</concept_id>
<concept_desc>Software and its engineering~Constraint and logic languages</concept_desc>
<concept_significance>500</concept_significance>
</concept>
</ccs2012>
\end{CCSXML}

\ccsdesc[500]{Theory of computation~Constraint and logic programming}
\ccsdesc[100]{Theory of computation~Operational semantics}
\ccsdesc[500]{Software and its engineering~Constraint and logic languages}

\keywords{relational programming, refutational completeness, divergence test}

\copyrightyear{2018}
\acmYear{2018}
\setcopyright{acmcopyright}
\acmConference[PPDP '18]{The 20th International Symposium on Principles and Practice of Declarative Programming}{September 3--5, 2018}{Frankfurt am Main, Germany}
\acmBooktitle{The 20th International Symposium on Principles and Practice of Declarative Programming (PPDP '18), September 3--5, 2018, Frankfurt am Main, Germany}
\acmPrice{15.00}
\acmDOI{10.1145/3236950.3236958}
\acmISBN{978-1-4503-6441-6/18/09}

\maketitle

\section{Introduction}
\label{sec:intro}

Verifying a solution for a problem is much easier than finding one~--- this common wisdom can be confirmed by anyone who used 
both to learn and to teach. This observation can be justified by its theoretical applications, thus being more than informal knowledge. For example, let us have a language $\mathcal{L}$. If there is a predicate $V_\mathcal{L}$ such~that
\[
\forall\omega\;:\;\omega\in\mathcal{L}\;\Longleftrightarrow\;\exists p_\omega\;:\;V_\mathcal{L}(\omega,p_\omega)
\]
(with $p_\omega$ being of size, polynomial on $\omega$) and we can recognize $V_\mathcal{L}$ in a polynomial time, then we call $\mathcal{L}$ to be in the class $NP$~\cite{Garey:1990:CIG:574848}. Here $p_\omega$ plays role of a justification (or proof) for the fact $\omega\in\mathcal{L}$. For example, if
$\mathcal{L}$ is a language of all hamiltonian graphs, then $V_\mathcal{L}$ is a predicate which takes a graph $\omega$ and some path $p_\omega$ and verifies that $p_\omega$ is indeed a hamiltonial path in $\omega$. The implementation of the predicate $V_\mathcal{L}$, however, tells us very little about the \emph{search procedure} which would calculate $p_\omega$ as a function of $\omega$. For the whole class of $NP$-complete problems no polynomial search procedures are known, and their existence at all is a long-standing problem in the complexity theory.

There is, however, a whole research area of \emph{relational interpreters}, in which a very close problem is addressed. Given a language $\mathcal{L}$, its \emph{interpreter} is a function \lstinline|eval$_\mathcal{L}$| which takes a program $p^\mathcal{L}$ in the language $\mathcal{L}$ and an input $i$ and calculates some output such that
\[
\mbox{\lstinline|eval$_\mathcal{L}$|}(p^\mathcal{L}, i)=\sembr{p^\mathcal{L}}_{\mathcal L}\,(i)
\]
where $\sembr{\bullet}_{\mathcal L}$ is the semantics of the language $\mathcal{L}$. In these terms, a verification predicate $V_\mathcal{L}$ can be
considered as an interpreter which takes a program $\omega$, its input $p_\omega$ and returns $true$ or \false. A \emph{relational} interpreter is an interpreter which is implemented not as a function \lstinline|eval$_\mathcal{L}$|, which calculates the output from a program and its input, but as a relation \lstinline|eval$^o_\mathcal{L}$|
which connects a program with its input and output. This alone would not have much sense, but if we allow the arguments of \lstinline|eval$^o_\mathcal{L}$|
to contain \emph{variables} we can consider relational interpreter as a generic search procedure which determines the values for these variables making the
relation hold. Thus, with relational interpreter it is possible not only to calculate the output from an input, but also to run a program in 
an opposite ``direction'', or to synthesize a program from an input-output pair, etc. In other words, relational verification predicate is capable
(in theory) to both \emph{verify} a solution and \emph{search} for it.

Implementing relational interpreters amounts to writing it in a relational language. In principle, any conventional language for logic programming
(Prolog~\cite{lozov:prolog}, Mercury~\cite{somogyi1996execution}, etc.) would make the job. However, the abundance of extra-logical features and the incompleteness of default search
strategy put a number of obstacles on the way. There is, however, a language specifically designed for pure relational programming, and, in a
narrow sense, for implementing relational interpreters~--- \textsc{miniKanren}~\cite{lozov:TheReasonedSchemer}. Relational interpreters, implemented
in \textsc{miniKanren}, demonstrate all their expected potential: they can synthesize programs by example, search for errors in partially defined programs~\cite{lozov:seven}, produce self-evaluated programs~\cite{lozov:quines}, etc. However, all these results are obtained for a family
of closely related Scheme-like languages and require a careful implementation and even some \emph{ad-hoc} optimizations in the relational
engine. 

From a theoretical standpoint a single relational interpreter for a Turing-complete language is sufficient: indeed, any other interpreter
can be turned into a relational one just by implementing it in a language, for which relational interpreter already exists. However, the overhead
of additional interpretation level can easily make this solution impractical. The standard way to tackle the problem is partial evaluation or specialization~\cite{jones1993partial}.
A \emph{specializer} \lstinline|spec$_\mathcal{M}$| for a language $\mathcal{M}$ for any program $p^\mathcal{M}$ in this language and its partial input $i$ returns some program which, being applied to the residual input $x$, works exactly as the original program on both $i$ and~$x$:
\[
\forall x\;:\;\sembr{\mbox{\lstinline|spec$_\mathcal{M}$|}\,(p^\mathcal{M}, i)}_\mathcal{M}\,(x)=\sembr{p^\mathcal{M}}_\mathcal{M}\,(i, x).
\]

If we apply a specializer to an interpreter and a source program, we obtain what is called \emph{the first Futamura projection}~\cite{futamura1971partial}:
\[
\forall i\;:\; \sembr{\mbox{\lstinline|spec$_\mathcal{M}$|}\,(\mbox{\lstinline|eval$^\mathcal{M}_\mathcal{L}$|}, p^\mathcal{L})}_\mathcal{M}\,(i)=\sembr{\mbox{\lstinline|eval$^\mathcal{M}_\mathcal{L}$|}}_\mathcal{M}\,(p^\mathcal{L}, i).
\]
Here we added an upper index $\mathcal{M}$ to \lstinline|eval$_\mathcal{L}$| to indicate that we consider it as a program in 
the language $\mathcal{M}$. In other words, the first Futamura projection specializes an interpreter for a concrete program, 
delivering the implementation of this program in the language of interpreter implementation. An important property of
a specializer is \emph{Jones-optimality}~\cite{jones1993partial}, which holds when it is capable to completely
eliminate the interpretation overhead in the first Futamura projection. In our case $\mathcal{M}=\mbox{\textsc{miniKanren}}$, 
from which we can conclude that in order to eliminate the interpretation overhead we need a Jones-optimal specializer for \textsc{miniKanren}. 
Although implementing a Jones-optimal specializer is not an easy task even for simple functional languages, there is a Jones-optimal specializer for a logical language~\cite{leuschel2004specialising}, but not for \textsc{miniKanren}. 

The contribution of this paper is as follows:

\begin{itemize}
\item We demonstrate the applicability of relational programming and, in particular, relational interpreters for the task of
turning verifiers into solvers.
\item To obtain a relational verifier from a functional specification we apply \emph{relational conversion}~\cite{lozov:miniKanren,lozov:conversion}~---
a technique which for a first-order functional program directly constructs its relational counterpart. Thus, we introduce a number
of new relational interpreters for concrete search problems.
\item We employ supercompilation in the form of conjunctive partial deduction (CPD)~\cite{de1999conjunctive} to
eliminate the redundancy of a generic search algorithm caused by partial knowledge of its input.
\item We give a number of examples and perform an evaluation of various solutions for the approach we address.
\end{itemize}

Both relational conversion and conjunctive partial deduction are done in an automatic manner. The only thing one needs to specify is the known arguments or the execution direction of a relation. 

As concrete implementation of \textsc{miniKanren} we use \textsc{OCanren}~\cite{lozov:ocanren}~--- its embedding in \textsc{OCaml}; we use
\textsc{OCaml} to write functional verifiers; our prototype implementation of conjunctive partial deduction is written in \textsc{Haskell}.

The paper is organized as follows. In Section~\ref{sec:example} we give a complete example of solving a concrete problem~--- searching for
a path in a graph,~--- with relational verifier. Section \ref{sec:conversion} recalls the cornerstones of relational programming in 
\textsc{miniKanren} and the relational conversion technique. In Section~\ref{sec:cpd} we describe how conjunctive partial deduction was 
adapted for relational programming. Section~\ref{sec:eva} presents the evaluation results for concrete solvers built using the technique
in question. The final section concludes.

\begin{figure*}[t]
\[
\begin{array}{cccll}
  &\mathcal{C} & = & \{C_i^{k_i}\} & \mbox{constructors with arities} \\
  &\mathcal{T}_X & = & X \cup \{C_i^{k_i} (t_1, \dots, t_{k_i}) \mid t_j\in\mathcal{T}_X\} & \mbox{terms over the set of variables $X$} \\
  &\mathcal{D} & = & \mathcal{T}_\emptyset & \mbox{ground terms}\\
  &\mathcal{X} & = & \{ x, y, z, \dots \} & \mbox{syntactic variables} \\
  &\mathcal{A} & = & \{ \alpha, \beta, \gamma, \dots \} & \mbox{semantic variables} \\
  &\mathcal{R} & = & \{ R_i^{k_i}\} &\mbox{relational symbols with arities} \\
  &\mathcal{G} & = & \mathcal{T_X}\equiv\mathcal{T_X}   &  \mbox{unification} \\
  &            &   & \mathcal{G}\wedge\mathcal{G}     & \mbox{conjunction} \\
  &            &   & \mathcal{G}\vee\mathcal{G}       &\mbox{disjunction} \\
  &            &   & \mbox{\lstinline|fresh|}\;\mathcal{X}\;.\;\mathcal{G} & \mbox{fresh variable introduction} \\
  &            &   & R_i^{k_i} (t_1,\dots,t_{k_i}),\;t_j\in\mathcal{T_X} & \mbox{relational symbol invocation} \\
  &\mathcal{S} & = & \{R_i^{k_i} = \lambda\;x_1^i\dots x_{k_i}^i\,.\, g_i;\}\; g & \mbox{specification}
\end{array}
\]
\caption{The syntax of the source language}
\label{syntax}
\end{figure*}

\begin{comment}
\begin{figure}[t]
%\centering
\[
\begin{array}{rcl}
  \mathcal{FV}\,(x)&=&\{x\}\\
  \mathcal{FV}\,(C_i^{k_i}\,(t_1,\dots,t_{k_i}))&=&\bigcup\mathcal{FV}\,(t_i)\\
  \mathcal{FV}\,(t_1\equiv t_2)&=&\mathcal{FV}\,(t_1)\cup\mathcal{FV}\,(t_2)\\
  \mathcal{FV}\,(g_1\wedge g_2)&=&\mathcal{FV}\,(g_1)\cup\mathcal{FV}\,(g_2)\\
  \mathcal{FV}\,(g_1\vee g_2)&=&\mathcal{FV}\,(g_1)\cup\mathcal{FV}\,(g_2)\\
  \mathcal{FV}\,(\mbox{\lstinline|fresh|}\;x\;.\;g)&=&\mathcal{FV}\,(g)\setminus\{x\}\\
  \mathcal{FV}\,(R_i^{k_i}\,(t_1,\dots,t_{k_i}))&=&\bigcup\mathcal{FV}\,(t_i)
\end{array}
\]
\caption{Free variables in terms and goals}
\label{free}
\end{figure}
\end{comment}

\section{The Language}
\label{language}
 
In this section, we introduce the syntax of the language we use throughout the paper, describe the informal semantics, and give some examples.

The syntax of the language is shown in Fig.~\ref{syntax}. First, we fix a set of constructors $\mathcal{C}$ with known arities and consider
a set of terms $\mathcal{T}_X$ with constructors as functional symbols and variables from $X$. We parameterize this set with an alphabet of
variables since in the semantic description we will need \emph{two} kinds of variables. The first kind, \emph{syntactic} variables, is denoted
by $\mathcal{X}$. The second kind, \emph{semantic} or \emph{logic} variables, is denoted by $\mathcal{A}$.
We also consider an alphabet of \emph{relational symbols} $\mathcal{R}$ which are used to name relational definitions.
The central syntactic category in the language is \emph{goal}. In our case, there are five types of goals: \emph{unification} of terms,
conjunction and disjunction of goals, fresh variable introduction, and invocation of some relational definition. Thus, unification is used
as a constraint, and multiple constraints can be combined using conjunction, disjunction, and recursion.
The final syntactic category is a \emph{specification} $\mathcal{S}$. It consists of a set
of relational definitions and a top-level goal. A top-level goal represents a search procedure which returns a stream of substitutions for
the free variables of the goal. The definition for a set of free variables for both terms and goals is conventional;
%given in Figure~\ref{free};
as ``\lstinline|fresh|''
is the sole binding construct the definition is rather trivial. The language we defined is first-order, as goals can not be passed as parameters,
returned or constructed at runtime.

We now informally describe how relational search works. As we said, a goal represents a search procedure. This procedure takes a \emph{state} as input and returns a
stream of states; a state (among other information) contains a substitution that maps semantic variables into the terms over semantic variables. Then five types of
scenarios are possible (depending on the type of the goal):

\begin{itemize}
\item Unification ``\lstinline|$t_1$ === $t_2$|'' unifies terms $t_1$ and $t_2$ in the context of the substitution in the current state. If terms are unifiable,
  then their MGU is integrated into the substitution, and a one-element stream is returned; otherwise the result is an empty stream.
\item Conjunction ``\lstinline|$g_1$ /\ $g_2$|'' applies $g_1$ to the current state and then applies $g_2$ to each element of the result, concatenating
  the streams.
\item Disjunction ``\lstinline|$g_1$ \/ $g_2$|'' applies both its goals to the current state independently and then concatenates the results.
\item Fresh construct ``\lstinline|fresh $x$ . $g$|'' allocates a new semantic variable $\alpha$, substitutes all free occurrences of $x$ in $g$ with $\alpha$, and
  runs the goal.
\item Invocation ``$\lstinline|$R_i^{k_i}$ ($t_1$,...,$t_{k_i}$)|$'' finds a definition for the relational symbol \mbox{$R_i^{k_i}=\lambda x_1\dots x_{k_i}\,.\,g_i$}, substitutes
  all free occurrences of a formal parameter $x_j$ in $g_i$ with term $t_j$ (for all $j$) and runs the goal in the current state.
\end{itemize}

We stipulate that the top-level goal is preceded by an implicit ``\lstinline|fresh|'' construct, which binds all its free variables, and that the final substitutions
for these variables constitute the result of the goal evaluation.

Conjunction and disjunction form a monadic~\cite{Monads} interface with conjunction playing role of ``\lstinline|bind|'' and disjunction~--- of ``\lstinline|mplus|''.
In this description, we swept a lot of important details under the carpet~--- for example, in actual implementations the components of disjunction are not evaluated in
isolation, but both disjuncts are being evaluated incrementally with the control passing from one disjunct to another (\emph{interleaving})~\cite{Search};
the evaluation of some goals can be additionally deferred (via so-called ``\emph{inverse-$\eta$-delay}'')~\cite{MicroKanren}; instead of streams
the implementation can be based on ``ferns''~\cite{BottomAvoiding} to defer divergent computations, etc. In the following sections, we present
a complete formal description of relational semantics which resolves these uncertainties in a conventional way.

As an example consider the following specification. For the sake of brevity we
abbreviate immediately nested ``\lstinline|fresh|'' constructs into the one, writing ``\lstinline|fresh $x$ $y$ $\dots$ . $g$|'' instead of
``\lstinline|fresh $x$ . fresh $y$ . $\dots$ $g$|''.

\begin{tabular}{p{5.5cm}p{5.5cm}}
\begin{lstlisting}
append$^o$ = fun x y xy .
 ((x === Nil) /\ (xy === y)) \/
 (fresh h t ty .
   (x  === Cons (h, t))  /\
   (xy === Cons (h, ty)) /\
   (append$^o$ t y ty));

revers$^o$ x x
\end{lstlisting} &
\begin{lstlisting}
revers$^o$ = fun x y .
 ((x === Nil) /\ (y === Nil)) \/
 (fresh h t tr .
   (x === Cons (h, t)) /\
   (append$^o$ tr (Cons (h, Nil)) y) /\
   (revers$^o$ t tr));
\end{lstlisting}
\end{tabular}

Here we defined\footnote{We respect here a conventional tradition for \textsc{miniKanren} programming to superscript all relational names with ``$^o$''.}
two relational symbols~--- ``\lstinline|append$^o$|'' and ``\lstinline|revers$^o$|'',~--- and specified a top-level goal ``\lstinline|revers$^o$ x x|''.
The symbol ``\lstinline|append$^o$|'' defines a relation of concatenation of lists~--- it takes three arguments and performs a case analysis on the first one. If the
first argument is an empty list (``\lstinline|Nil|''), then the second and the third arguments are unified. Otherwise, the first argument is deconstructed into a head ``\lstinline|h|''
and a tail ``\lstinline|t|'', and the tail is concatenated with the second argument using a recursive call to ``\lstinline|append$^o$|'' and additional variable ``\lstinline|ty|'', which
represents the concatenation of ``\lstinline|t|'' and ``\lstinline|y|''. Finally, we unify ``\lstinline|Cons (h, ty)|'' with ``\lstinline|xy|'' to form a final constraint. Similarly,
``\lstinline|revers$^o$|'' defines relational list reversing. The top-level goal represents a search procedure for all lists ``\lstinline|x|'', which are stable under reversing, i.e.
palindromes. Running it results in an infinite stream of substitutions:

\begin{lstlisting}
   $\alpha\;\mapsto\;$ Nil
   $\alpha\;\mapsto\;$ Cons ($\beta_0$, Nil)
   $\alpha\;\mapsto\;$ Cons ($\beta_0$, Cons ($\beta_0$, Nil))
   $\alpha\;\mapsto\;$ Cons ($\beta_0$, Cons ($\beta_1$, Cons ($\beta_0$, Nil)))
   $\dots$
\end{lstlisting}

where ``$\alpha$''~--- a \emph{semantic} variable, corresponding to ``\lstinline|x|'', ``$\beta_i$''~--- free semantics variables. Therefore, each substitution represents a set of all palindromes of a certain length.


\section{Refutational Incompleteness and Conjunction Non-Commutativity}
\label{incompleteness}

The language, defined in the previous section, is expected to allow defining computable relations in a 
very concise and declarative form. In particular, it is expected from a relational 
specification to preserve its behavior regardless the order of conjunction/disjunction 
constituents. Regretfully, in general this is not true, and one of the most important
manifestations of this deficiency is \emph{refutational incompleteness}.  

In the context of relational programming, refutational completeness~\cite{WillThesis} is understood as 
a capability of a program to discover the absence of solutions and stop. At the first glance,
the divergence in the case of solution absence does not seem to be a severe problem. However, as
we see shortly, refutational incompleteness leads to many observable negative effects in numerous
practically important cases. 

We demonstrate the effect of refutational incompleteness with a very simple example. Let us take the
definition of \lstinline{append$^o$} from the previous section and try to evaluate the following query:

\begin{lstlisting}
   fresh ($p\;q$) (append$^o$ $p$ $q$ Nil)
\end{lstlisting}

We would expect this query to converge to the single answer \mbox{$p=\lstinline|Nil|$}, \mbox{$q=\lstinline|Nil|$};
however, in the reality the query diverges. We sketch here the explanation, omitting some non-essential technical
details, such as semantic variables allocation, etc.:

\begin{itemize}
\item First we evaluate the first disjunct of \lstinline|append$^o$|'s body and unify $p$ with \lstinline|Nil| (successfully)
and \lstinline|Nil| with $q$ (successfully), which gives us the first (expected) answer.

\item Then we proceed to the second disjunct, which is a conjunction of three simpler goals:

  \begin{itemize} 
     \item in the first one we unify $p$ with \lstinline|Cons ($h$, $t$)| (successfully);
     \item in the second we encounter a recursive call \lstinline|append$^o$ $t$ $q$ Nil|; since its arguments are merely the renamings of the enclosing one, we repeat from the top and never stop.
  \end{itemize} 
\end{itemize}

The problem is that the semantics of conjunction, in fact, is not commutative: when the first conjunct diverges and the second fails, the whole
conjunction diverges. We stress that this is not a deviation of our semantics, but a well-known phenomenon, manifesting itself in all known
miniKanren implementations. In our example, switching two last conjuncts in the definition of \lstinline|append$^o$| solves the problem~---
now the whole search stops after the unsuccessful attempt to unify \lstinline|Nil| and \lstinline|Cons ($h$, $ty$)| with no recursive call.
This, improved version of \lstinline|append$^o$|, is known to be refutationally complete. In fact, there is a conventional ``rule of thumb''
for miniKanren programming to place the recursive call as far right as possible in a list of conjuncts. 

This convention, however, does not always help; to tell the truth, it often makes the things worse. Consider 
as an example yet another relation on lists:

\begin{lstlisting}
   revers$^o$ $\binds$ $\lambda\;x\;x_r$ . 
     (($x$ === $\;\;$Nil) /\ ($x_r$ === $\;\;$Nil)) \/
     (fresh ($h$ $t$ $t_r$)
        ($x$  === $\;$Cons ($h$, $t$)) /\
        (append$^o$ $t_r$ (Cons ($h$, Nil)) $x_r$) /\
        (revers$^o$ $t$ $t_r$)
     )
\end{lstlisting}

This relation corresponds to a relational list reversing; as we see, the recursive call is placed to
the end. However, the following query

\begin{lstlisting}
   fresh ($q$) (revers$^o$ (Cons (A, Nil)) $q$)
\end{lstlisting}

\noindent diverges, while

\begin{lstlisting}
   fresh ($q$) (revers$^o$ $q$ (Cons (A, Nil)))
\end{lstlisting}

\noindent converges to the expected results. If we switch the two last conjuncts in the definition of
\lstinline|revers$^o$|, the situation reverses: the first query converges, while the second diverges. 
This example demonstrates that the desired position of a recursive call (and, in general, the order of
conjuncts) depends on the direction, in which the relation of interest is evaluated.

There are, however, some cases, when the same relation is evaluated in both directions, regardless
the query. We can take as an example relational permutations, which can be implemented by running
relational list sorting in both directions:

\begin{lstlisting}
   sort$^o$ $\binds\lambda\;x\;x_s\;.\; \dots$
   perm$^o$ $\binds\lambda\;x\;x_p\;.$
     fresh ($x_s$) 
       (sort$^o$ $x$ $x_s$) $\wedge$ (sort$^o$ $x_p$ $x_s$) 
\end{lstlisting}

The idea of this implementation is very simple. Let us want to calculate all permutations of some list $l$.
We first sort $l$, obtaining the sorted version $l^\prime$; then we ask for all lists which, being sorted,
become equal to $l^\prime$. Obviously, all such lists are merely permutations of the original list $l$. The
important observation is that the existence of a single list sorting relation is sufficient to implement this
idea.

The concrete definition of the relational list sorting \lstinline|sort$^o$| is not important, so we
omit it due to the space considerations (an interested reader can refer to~\cite{OCanren}). The important part 
is that it is obviously recursive and not refutationally complete, and it is being evaluated 
in \emph{both} directions within the body of \lstinline|perm$^o$|. So, \lstinline|perm$^o$| is expected 
to perform poorly regardless the order of recursive calls in \lstinline|sort$^o$| implementation; it, 
indeed, does. First, if we request all solutions, both \lstinline|fresh ($q$) (perm$^o$ l $q$)| and \lstinline|fresh ($q$) (perm$^o$ $q$ l)| diverge for arbitrary non-empty list \lstinline|l| regardless the implementation of \lstinline|sort$^o$|; second, even if we request only a first few existing solutions, it does not provide any results in a reasonable time even for very small list lengths (4, 5, etc.).

Interesting, that if we interested in all solutions,
we have to accurately precompute their number in order not to request more, than exists. For some problems,
it may be not so simple, as it looks at a first glance (for example, the number of all permutations is
not a factorial, but a number of permutations with repetitions). Finally, getting the number of solutions can 
itself be an objective for writing a relational specification (we provide some examples in Section~\ref{evaluation}),
and without refutational completeness requesting all solutions to calculate their number is out of
reach.

\section{Search Improvement}
\label{improvement}

As we've seen in the previous section, the non-commutativity of conjunction in the presence of recursion
is one of the reasons for refutational incompleteness. Switching arguments of a certain conjunction
can sometimes improve the results; there is, however, no certain static order, beneficial in all cases.
Thus, we can make the following observations:

\begin{itemize}
\item the conjunction to change has to be properly identified;
\item the order of conjunct evaluation has to be a subject of a \emph{dynamic} choice.
\end{itemize}

Our improvement of the search is based on the idea of switching the order of conjuncts only when
the divergence of the first one is detected. More specifically: 

\begin{itemize}
\item during the search, we keep track of all conjunctions being performed;
\item when we detect the divergence, we roll back to the nearest conjunction, for which 
we did not try all orders of constituents yet, switch its constituents, and rerun 
the search from that conjunction.
\end{itemize}

The important detail is the divergence test. Of course, due to the fundamental results in computability
theory, there is no hope to find a \emph{precise} computable test that constitutes the necessary and 
sufficient condition of divergence. However, in our case a sufficient condition is sufficient. Indeed,  
a sufficient condition identifies a case, when the search, being continued, will lead to an incompleteness 
(since a divergence in our semantics always means incompleteness). Thus, it is no harm to try some other way. 

Another important question is the discipline of conjuncts reordering. Indeed, simply switching any two operands
of, for example, \mbox{$(g_1\wedge g_2)\wedge g_3$}, would not allow us to try \mbox{$(g_1\wedge g_3)\wedge g_2$}.
Thus, we have to flatten each ``cluster'' of nested conjunctions into a list of conjuncts\mbox{$\bigwedge g_i$}, 
where none of the goals $g_i$ is a conjunction. Then, it may seem at the first glance that the number of orderings to try 
is exponential on the number of conjuncts; we are going to show that, fortunately, this is not the case, and
a quadratic number of orders is sufficient.

In the rest of the section we address all these issues in details: first, we formally present the divergence
criterion and prove the necessity property; then, we describe an efficient reordering discipline. Finally, we present a
modified version of the semantics with incorporated divergence test and reordering. This semantics can be
considered as a modified version of the search, and we prove that this modification is a proper improvement in terms
of convergence.

\subsection{The Divergence Test}

Our divergence test is based on the following notion:

\begin{definition}
\normalfont 
We say that a vector of terms $\overline{a^{\phantom{x}}_i}$ is more general, than a vector of terms $\overline{b^{\phantom{x}}_i}$ (notation 
$\overline{a^{\phantom{x}}_i}\succeq\overline{b^{\phantom{x}}_i}$), if there is a substitution $\tau$, such that $\forall i\;b_i = a_i \tau$.
\end{definition}

The idea of the divergence test is rather simple: it identifies a recursive call with more general arguments 
than (some) enclosing one. To state it formally and prove it using the semantics from section~\ref{language}, we need several definitions and lemmas.

\begin{definition}
\normalfont
A semantic variable $v$ is \emph{observable} w.r.t. the interpretation $\iota$ and substitution $\sigma$, if there exists 
a syntactic variable $x$, such that \mbox{$v \in FV(\iota(x) \sigma)$}.
\end{definition}

\begin{definition}
A triplet of interpretation, substitution and a set of allocated semantic variables \mbox{$(\iota,\sigma,\delta)$} is
called \emph{coherent}, if \mbox{$dom(\sigma) \subseteq \delta$}, and any semantic variable, observable w.r.t. $\iota$ and $\sigma$,
belongs to $\delta$.  
\end{definition}

\begin{definition}
\normalfont
A semantic statement 

$$
\otrans{\Gamma,\iota}{(\sigma,\,\delta)}{g}{S}
$$ 

\noindent is \emph{well-formed}, if \mbox{$(\iota,\sigma,\delta)$} is a coherent triplet.
\end{definition}

Note, the root semantic statement \mbox{$\otrans{\Gamma,\bot}{(\epsilon,\,\emptyset)}{g}{S}$} is always well-formed.

\begin{lemma}
\label{one}
\normalfont
 For a well-formed semantic statement, every statement in its derivation tree is also well-formed.
\end{lemma}

The proof is by induction on the derivation tree. Note, we need to generalize the statement of the lemma, adding the condition that
\mbox{$(\iota,\sigma_r,\delta_r)$} is a coherent triplet for any \mbox{$(\sigma_r,\,\delta_r) \in S$}.

The next lemma ensures that any substitution in the RHS of a semantic statement is a correct refinement of that in the LHS:

\begin{lemma}
\label{two}
\normalfont
For a well-formed semantic statement 

$$
\otrans{\Gamma,\iota}{(\sigma,\,\delta)}{g}{S}
$$ 

\noindent and any result \mbox{$(\sigma_r,\,\delta_r) \in S$}, there exists a substitution $\Delta$, such that:
  \begin{enumerate}
    \item \mbox{$\sigma_r = \sigma\circ\Delta$};
    \item any semantic variable \mbox{$v\in dom(\Delta)\cup ran(\Delta)$} either is observable w.r.t. $\iota$ and $\sigma$,
 or does not belong to $\delta$ (where \mbox{$ran(\Delta)=\bigcup_{v\in dom(\Delta)}FV(\Delta(v))$}).
  \end{enumerate}   
\end{lemma}

The proof is by induction on the derivation tree; we as well need to generalize the statement of the lemma, adding the condition that the 
set of all allocated semantic variables $\delta$ can only grow during the evaluation.

The final lemma formalizes the intuitive considerations that the evaluation for a certain state $(\sigma^\prime,\delta^\prime)$ cannot
diverge, if the evaluation for a more general state $(\sigma,\delta)$ doesn't diverge:

\begin{lemma}
\label{three}
\normalfont
Let 

$$
\otrans{\Gamma,\iota}{(\sigma,\,\delta)}{g}{S}
$$ 

\noindent be a well-formed semantic statement, \mbox{$(\iota^\prime,\sigma^\prime,\delta^\prime)$} be a coherent triplet,
and let $\tau$ be a substitution, such that \mbox{$\iota^\prime(x) \sigma^\prime = \iota(x) \sigma \tau$} for any syntactic
variable $x$. Then

$$
\otrans{\Gamma,\iota^\prime}{(\sigma^\prime,\,\delta^\prime)}{g}{S^\prime}
$$

\noindent is well-formed and its derivation height is not greater than that for the first statement.
\end{lemma}

The proof is by induction on the derivation tree for the first statement. We need to generalize the statement of the lemma, adding the requirement that 
for any substitution $s^\prime_r$ in the RHS of the second statement, there has to be a substitution $s_r$ in the RHS of the first statement,
such that there exists a substitution $\tau_r$, such that \mbox{$\iota^\prime(x) \sigma^\prime_r = \iota(x) \sigma_r \tau_r$} for any syntactic variable $x$. 
In the cases of $\textsc{Fresh}$ and $\textsc{Invoke}$ rules, some semantic variables can become non-observable, and we need to define a substitution $\tau_r$ 
separately for these ``forgotten'' variables and those, which remain observable, using Lemma~\ref{two}.

Now we are ready to claim and prove the divergence criterion.

\setcounter{theorem}{0}
\begin{theorem}[Divergence criterion]
\label{criterion}
\normalfont
For any well-formed semantic statement 

$$
\otrans{\Gamma,\iota}{(\sigma,\,\delta)}{r^k\,t_1\dots t_k}{S}
$$ 

if its proper derivation subtree has a semantic statement 

$$
\otrans{\Gamma,\iota^\prime}{(\sigma^\prime,\,\delta^\prime)}{r^k\,t^\prime_1\dots t^\prime_k}{S^\prime}
$$

then \mbox{$\overline{t^\prime_i \iota^\prime \sigma^\prime} \not \succeq \overline{t^{\phantom{\prime}}_i \iota \sigma}$}. 
\end{theorem}
\begin{proof}
Assume that \mbox{$\overline{t^\prime_i \iota^\prime \sigma^\prime}\succeq \overline{t^{\phantom{\prime}}_i \iota \sigma}$}. 

By Lemma~\ref{one}, the semantic statement

$$
\otrans{\Gamma,\iota^\prime}{(\sigma^\prime,\,\delta^\prime)}{r^k\,t^\prime_1\dots t^\prime_k}{S^\prime}
$$

\noindent is well-formed.

By Lemma~\ref{three}, the derivation tree for

$$
\otrans{\Gamma,\iota^\prime}{(\sigma^\prime,\,\delta^\prime)}{r^k\,t^\prime_1\dots t^\prime_k}{S^\prime}
$$

\noindent has greater or equal height than that for

$$
\otrans{\Gamma,\iota}{(\sigma,\,\delta)}{r^k\,t_1\dots t_k}{S}
$$ 

\noindent which contradicts the theorem condition.

\end{proof}

The theorem justifies that, indeed, our test constitutes a sufficient condition for a divergence: if the execution
reaches a relation call with more general arguments, than those of some enclosing one, then it has no derivation
in our semantics, and, thus, it is not terminating.

\setarrow{\xRightarrow}
\setsubarrow{_e}
\begin{figure*}
\begin{minipage}[t]{\textwidth}
\small
\[
\cotrans{\Gamma,\,\iota,\,h}{(\sigma,\,\delta)}{t_1\equiv t_2}{\emptyset}{mgu\,(t_1\iota\sigma,\,t_2\iota\sigma) = \bot}\ruleno{UnifyFail$^+$}
\]
\[
\cotrans{\Gamma,\,\iota,\,h}{(\sigma,\,\delta)}{t_1\equiv t_2}{(\sigma\circ\Delta,\,\delta)}{mgu\,(t_1\iota\sigma,\,t_2\iota\sigma) = \Delta\ne\bot}\ruleno{UnifySuccess$^+$}
\]
\[
\trule{\otrans{\Gamma,\,\iota,\,h}{(\sigma,\,\delta)}{g_1}{S_1};\quad
       \otrans{\Gamma,\,\iota,\,h}{(\sigma,\,\delta)}{g_2}{S_2}
      }
      {\otrans{\Gamma,\,\iota,\,h}{(\sigma,\,\delta)}{g_1\vee g_2}{S_1\cup S_2}}\ruleno{Disj$^+$}
\]
\[
\crule{\otrans{\Gamma,\,\iota[x\gets\alpha],\,h}{(\sigma,\,\delta\cup\{\alpha\})}{g}{S^\dagger}}
      {\otrans{\Gamma,\,\iota,\,h}{(\sigma,\,\delta)}{\lstinline|fresh($x$) $\;g$|}{S^\dagger}}
      {\alpha\in\meta{W}\setminus\delta}\ruleno{Fresh$^+$}
\]
\end{minipage}      
\caption{Improved search: inherited rules}
\label{improved-semantics-normal}
\end{figure*}

\begin{figure*}
\begin{minipage}[t]{\textwidth}
\small
\[
   \cotrans{\Gamma,\,\iota,\,h}{(\sigma,\,\delta)}{r^k t_1 \dots t_k}{\dagger}{v_i = t_i \iota \sigma, \; (v_1, \dots, v_k) \succeq h\,r^k}
   \ruleno{InvokeDiv$^+$}
\]

\[
\crule{\otrans{\Gamma,\,\epsilon[x_i\gets v_i],\,h[r^k\gets(v_1, \dots, v_k)]}{(\epsilon,\,\delta)}{g}{\bigcup_j\{(\sigma_j,\,\delta_j)\}}}
      {\otrans{\Gamma,\,\iota,\,h}{(\sigma,\,\delta)}{r^k t_1 \dots t_k}{\bigcup_j\{(\sigma\circ\sigma_j, \delta_j)\}}}
      {v_i=t_i\iota\sigma,\;\Gamma\,r^k=\lambda x_1 \dots x_k. g,\; (v_1, \dots, v_k) \nsucceq h\,r^k}
      \ruleno{Invoke$^+$}
\]
\end{minipage}      
\caption{Improved search: invocation and divergence detection}
\label{improved-semantics-invoke}
\end{figure*}

\begin{figure*}
\begin{minipage}[t]{\textwidth}
\small
\[
\trule{\otrans{\Gamma,\,\iota,\,h}{(\sigma,\,\delta)}{g_1}{\dagger}}
      {\otrans{\Gamma,\,\iota,\,h}{(\sigma,\,\delta)}{g_1\vee g_2}{\dagger}}\ruleno{DivDisjLeft$^+$}
\]
\[
\trule{\otrans{\Gamma,\,\iota,\,h}{(\sigma,\,\delta)}{g_2}{\dagger}}
      {\otrans{\Gamma,\,\iota,\,h}{(\sigma,\,\delta)}{g_1\vee g_2}{\dagger}}\ruleno{DivDisjRight$^+$}
\]
\[
\crule{\otrans{\Gamma,\,\epsilon[x_i\gets v_i],\,h[r^k\gets(v_1, \dots, v_k)]}{(\epsilon,\,\delta)}{g}{\dagger}}
      {\otrans{\Gamma,\,\iota,\,h}{(\sigma,\,\delta)}{r^k t_1 \dots t_k}{\dagger}}
      {v_i=t_i\iota\sigma,\;\Gamma\,r^k=\lambda x_1 \dots x_k. g,\; (v_1, \dots, v_k) \nsucceq h\,r^k}
      \ruleno{DivInvoke$^+$}
\]      
\end{minipage}      
\caption{Improved search: divergence propagation}
\label{improved-semantics-divergence-prop}
\end{figure*}

\subsection{Conjuncts Reordering}
\label{sec:reordering}

In this section we consider the discipline of conjuncts reordering. Recall, we flatten all nested conjunctions in 
clusters $\wedge g_i$, where none of $g_i$ is a conjunction. To evaluate a cluster, we have to evaluate
its conjuncts one after another, threading the results, starting from the initial substitution. Each time we
evaluate a conjunct, we can have three possible outcomes:

\begin{itemize}
\item The evaluation converges with some result. In this case, we can proceed with the next conjunct.
\item The evaluation diverges undetected. In this case, nothing can be done.
\item A divergence is detected by the test. This is the case when the reordering takes place.
\end{itemize}

In a general case, for each cluster there can be some converging prefix $\omega$ we've managed to evaluate so far (initially empty),
and the rest of the conjuncts $g_i$. Since $\omega$ converges, we have some set of substitutions $S_\omega$ that corresponds to the
result of $\omega$ evaluation.

Suppose none of $g_i$ converges on $S_\omega$ (i.e. for each $g_i$ there is at least one substitution in $S_\omega$, on which
$g_i$ diverges). We claim that reordering conjuncts inside $\omega$ would not help. Indeed, with any other order
of conjuncts, $\omega$ either diverges or converges with the same result (up to the renaming of semantic variables). Thus,
making any permutations inside $\omega$ is superfluous.

Next, suppose we have two different goals $g_1$ and $g_2$, which both converge on $S_\omega$ (i.e. both converge on each
substitution in $S_\omega$). Do we need to try both cases ($g_1$ and $g_2$) to extend the converging prefix?
It is rather easy to see that if, say, $g_2$ converges on $S_\omega$, then it will as well converge on the result of evaluation
of $g_1$ on $S_\omega$. Indeed, for arbitrary \mbox{$(\sigma, \delta)\in S_\omega$} we have

\[
\otrans{\dots}{(\sigma, \delta)}{g_1}{S^\prime_\omega}
\]

where each $\sigma^\prime$ (such that \mbox{$(\sigma^\prime, \delta^\prime)\in S^\prime_\omega$}) is a ``more specific'',
than $\sigma$, by Lemma~\ref{two}. By Lemma~\ref{three}, since $g_2$ converges on \mbox{$(\sigma, \delta)\in S_\omega$},
it converges on each \mbox{$(\sigma^\prime, \delta^\prime)\in S^\prime_\omega$} as well.

In other words, to extend a converging prefix we can choose arbitrary conjunct, which converges immediately
after this prefix, and this choice will never have to be undone.

Now we can specify the reordering discipline. Since we never re-evaluate a converging prefix, we do not
represent it. Thus, each cluster we consider from now on is a suffix of some initial cluster after
evaluation of some converging prefix (and, perhaps, after some reorderings performed so far).

Let us have a cluster \mbox{$\bigwedge_{i=1}^k g_i$}. We evaluate it on some substitution $\sigma$ in the context of some integer
value $p$ (initially $p=1$), which describes, which conjunct we have to try next. We operate as follows:

\begin{enumerate}
\item\label{reorder:top} We try to evaluate $g_p$ on $\sigma$. If the evaluation succeeds with a result $S^\prime$, we 
remove $g_p$ from the cluster and evaluate the rest for each substitution in $S^\prime$ and $p=1$.
  
\item If a divergence is detected, and $p\le k$, then increment $p$, and repeat from step~\ref{reorder:top} (which will try the next goal).
  
\item Otherwise, we give up and rollback to the enclosing cluster (if any).
\end{enumerate}

Thus, we apply a greedy approach: each time we have a converging prefix of conjuncts (possibly empty), and some tail.
We try to put each conjunct from the tail immediately after the prefix. If we find a converging conjunct, we attach
it to the prefix and continue; if no, then the list of conjuncts diverges. Thus, we can find a converging order
(if any) in a quadratic time. Note, for different substitutions in the result of a converging prefix evaluation
the order of remaining conjuncts can be different.

\begin{figure*}
\begin{minipage}[t]{\textwidth}
\small
\[
\trule{\setsubarrow{_r}\otrans{\Gamma,\,\iota,\,h,\,1}{(\sigma,\,\delta)}{\bigwedge\limits_{i=1}^n g_i}{S^\dagger}}
      {\otrans{\Gamma,\,\iota,\,h}{(\sigma,\,\delta)}{\bigwedge\limits_{i=1}^n g_i}{S^\dagger}}
      \ruleno{ClusterStart$^+$}
\]
\vskip3mm
\[
\crule{\otrans{\Gamma,\,\iota,\,h}{(\sigma,\,\delta)}{g_p}{\bigcup_j\{(\sigma_j,\,\delta_j)\}};\quad
       \forall j\;:\;\otrans{\Gamma,\,\iota,\,h}{(\sigma_j,\,\delta_j)}{\bigwedge\limits_{i\ne p}g_i}{S_j}
      }
      {\setsubarrow{_r}\otrans{\Gamma,\,\iota,\,h,\,p}{(\sigma,\,\delta)}{\bigwedge\limits_{i=1}^n g_i}{\bigcup S_j}}
      {1 \le p \le n}
\ruleno{ClusterStep$^+$}
\]
\vskip3mm
\[
\crule{\otrans{\Gamma,\,\iota,\,h}{(\sigma,\,\delta)}{g_p}{\bigcup_j\{(\sigma_j,\,\delta_j)\}};\quad
       \exists j\;:\;\otrans{\Gamma,\,\iota,\,h}{(\sigma_j,\,\delta_j)}{\bigwedge\limits_{i\ne p}g_i}{\dagger}
      }
      {\setsubarrow{_r}\otrans{\Gamma,\,\iota,\,h,\,p}{(\sigma,\,\delta)}{\bigwedge\limits_{i=1}^n g_i}{\dagger}}
      {1 \le p \le n}
\ruleno{ClusterDiv$^+$}
\]
\vskip3mm
\[
\crule{\otrans{\Gamma,\,\iota,\,h}{(\sigma,\,\delta)}{g_p}{\dagger};\quad
       {\setsubarrow{_r}\otrans{\Gamma,\,\iota,\,h,\,p+1}{(\sigma,\,\delta)}{\bigwedge\limits_{i=1}^n g_i}{S^\dagger}}
      }
      {\setsubarrow{_r}\otrans{\Gamma,\,\iota,\,h,\,p}{(\sigma,\,\delta)}{\bigwedge\limits_{i=1}^n g_i}{S^\dagger}}
      {1 \le p \le n}
\ruleno{ClusterNext$^+$}
\]
\vskip3mm
\[
{\setsubarrow{_r}\cotrans{\Gamma,\,\iota,\,h,\,p}{(\sigma,\,\delta)}{\bigwedge\limits_{i=1}^n g_i}{\dagger}{p>n}}
\ruleno{ClusterStop$^+$}
\]
\end{minipage}      
\caption{Improved search: conjuncts reordering}
\label{improved-semantics-reordering}
\end{figure*}

\subsection{Improved Search Semantics}

Here we combine all observations, presented in the preceding subsections~--- the divergence test, conjunct clustering
and reordering,~--- and express the improved search in terms of a big-step operational semantics that is an extension
of the initial one, presented in Section~\ref{language}.

We denote ``$\xRightarrow{}_e$'' the semantic relation for the improved search, and we add another component to the
environment~--- a history $h$,~--- which maps a relational symbol to a list of fully interpreted terms as its arguments.
As we are (sometimes) capable of detecting the divergence, besides a regular set of answers $S$ as a result of evaluation
we can have a divergence signal, which we denote $\dagger$; $S^\dagger$ ranges over both the set of answers $S$ and the divergence
signal $\dagger$.

For the convenience of presentation we split the set of semantic rules into a few groups. The first one is the inherited
rules (see Fig.~\ref{improved-semantics-normal})~--- those, which did not change (except for the extension in the
environment and evaluation result). Note, the rule \rulename{Disj$^+$} does not handle the divergence detection
in either of disjuncts.

The next group describes the invocation and divergence detection (see Fig.~\ref{improved-semantics-invoke}). On
relation invocation, we first consult with the history. If the history indicates that the invocation is performed in the
context of the same relation evaluation with more specific arguments, then we raise the divergence signal; otherwise
we perform normally. Note, the rule \rulename{Invoke$^+$} does not handle the divergence in the \emph{body} of
invoked relation.

The next group describes the divergence signal propagation (see Fig.~\ref{improved-semantics-divergence-prop}). Here
the divergence signal, raised in one of the disjuncts or in the body of relational definition, is propagated to the upper
levels of the derivation tree.

The final group handles the conjunct reordering (see Fig.~\ref{improved-semantics-reordering}). As we need a reordering
parameter $p$ (see Section~\ref{sec:reordering}), we introduce another relation ``$\xRightarrow{}_r$'' with environment,
enriched by $p$.

The rule \rulename{ClusterStart$^+$} describes the case, when we make an attempt to evaluate a cluster. It can happen, when
we either first encounter an original cluster or try to evaluate a suffix of some initial cluster past some converging
prefix. As the reordering starts now, we recurse to the reordering relation with the parameter \mbox{$p=1$} (which means,
that the first conjunct will be tried to evaluate next).

Two next rules describe the case, when the $p$-th conjunct, being tried to evaluate, succeeds with some result. In the rule
\rulename{ClusterStep$^+$} we handle the case, when all other conjuncts can be evaluated in the context of that result: we
combine the outcomes, which completes the evaluation of the whole cluster. In the rule \rulename{ClusterDiv$^+$} we consider
the opposite case: now there is some conjunct $g_j$, which raises a divergence signal, being evaluated in the context of
the results, delivered by the evaluation of $g_p$. As we argued in Section~\ref{sec:reordering}, nothing can be done, and we
propagate the divergence signal.

The rule \rulename{ClusterNext$^+$} describes the case, when the $p$-th conjunct raises the divergence signal, and there are
some other conjuncts to try. We increment $p$ and proceed.

Finally, in the rule \rulename{ClusterStop$^+$} we handle the situation, when all available conjuncts in a cluster were tried to
evaluate first and raised the divergence signal. We propagate the signal in this case.

The following theorem is rather easy to prove:

\begin{theorem} For arbitrary $\Gamma$ and $g$ if

  \[{\setsubarrow{}\otrans{\Gamma,\,\bot}{(\epsilon,\,\emptyset)}{g}{S}}\]

  then
  
  \[{\setsubarrow{_e}\otrans{\Gamma,\,\bot,\,\bot}{(\epsilon,\,\emptyset)}{g}{S}}\]

\end{theorem}

Indeed, due to Theorem~\ref{criterion}, from the condition we can conclude that the divergence signal is
never raised during the evaluation, according to ``$\xRightarrow{}_e$''; but in this case the evaluation
steps coincide with those, according to ``$\xRightarrow{}$''. Thus, the improved search preserves the convergence.

\section{Evaluation}
\label{sec:eva}

In this section we present an evaluation of the proposed approach. 
We have implemented several relational interpreters for different search problems which can be found in the repository mentioned before. 
Some of the simpler interpreters demonstrate good performance for different directions on their own and for them CPD transformation is not needed. 
Thus, we will focus on two search problems which are more complex: searching for a path in a graph and searching for a unifier~\cite{lozov:unification} of two terms. 
For each problem we compare four programs.
\begin{enumerate}
    \item The solver generated by the unnesting alone.
    \item The solver generated by the unnesting strategy aimed at backward execution. 
    \item The solver generated by the unnesting and then specialized by conjunctive partial deduction for the backward direction.
    \item The interpretation of the functional verifier with the relational interpreter implemented in Scheme~\cite{lozov:seven}. 
\end{enumerate}

First, let us compare the performance of the solvers for the path searching problem.
The implementation of the functional verifier for this problem is described in Section~\ref{sec:example}. 
We ran the search on a graph with 20 nodes and 30 edges, consequentially
 searching for paths of the length 5, 7, 9, 11, 13, and 15. 
We averaged the execution times over 10 runs of the same query. 
We the limited the execution time by 300 seconds, and if the execution time of some query exceeded the timeout, we put ``>300s'' in the result table and did not request the execution of queries for longer paths. The results are presented in Table~\ref{tab:isPath}. 

We can conclude that the execution time increases with the length of the path to search, which is expected, since with the length of the path the number of the subpaths to be explored is increasing as well.
The solver generated by the unnesting alone and the interpretation with the relational intepreter demonstrate poor performance. 
The first one is due to its inherently inefficient execution in backward direction, while the second suffers from the interpretation overhead. 
Both the unnesting aimed at the backward execution and the solver automatically transformed with conjunctive partial deduction show good performance. 
Conjunctive partial deduction performs more thorough specialization, thus producing more efficient program. 

\begin{table}
\centering
\begin{tabular}{c|c|c|c|c|c|c}
Path length                   & 5      & 7     & 9      & 11      & 13     & 15        \\
\hline\hline
Only conversion               & 0.01s  & 1.39s &  82.13s & >300s     & ---      & ---     \\
\hline
Backward oriented conversion  & 0.01s & 0.37s &  2.68s & 2.91s   & 4.88s    & 10.63s   \\
\hline
Conversion and CPD            & 0.01s  & 0.06s &  0.34s & 2.66s   & 3.65s    & 6.22s  \\
\hline
Scheme interpreter            & 0.80s  & 8.22s & 88.14s & 191.44s & >300s   & ---   \\
\end{tabular}

 \caption{Searching for paths in the graph}
    \label{tab:isPath}
\end{table}

Now let us consider the problem of finding a unifier of two terms which have free variables.
A term is either a variable ($X, Y, \dots$) or some constructor applied to terms ($nil, cons(H, T), \dots$). 
A substitution maps a variable to a term. 
A unifier of two terms $t$ and $s$ is a substitution $\sigma$ which equalizes them: $t \sigma = s \sigma$ by simultaneously substituting the variables for their images.
For example, a unifier of the terms $cons(42, X) \text{ and } cons(Y, cons(Y, nil)) \text{ is a substitution } \{X \mapsto cons(42, nil), Y \mapsto 42\} $.

We implemented a functional verifier which checks if a substitution equalizes two input terms. 
We represent a variable name as a unique Peano number. 
A substitution is represented as a list of terms, in which the index of the term is equal to the variable name to which the term is bound, so the substitution $\{X \mapsto cons(42, nil), Y \mapsto 42\}$ is represented as a list ``\lstinline{[cons(42, nil); 42]}''.
The verifier returns true if the input terms can be unified with the candidate substitution and false otherwise. 

As in the previous problem, we compare four solvers generated for the verifier described. 
With each solver, we search for a unifier of two terms and compare the execution times. 
The time comparison is presented in Table~\ref{tab:uni}. 
The first two rows of each column contain two terms being unified. 
We use uppercase letters from the end of the alphabet ($X, Y, \dots$) to denote variables, lowercase letters from the beginning of the alphabet ($a, b, \dots$) to denote constants (constructors with zero arguments), identifiers which start from the lowercase letter ($f, g,\dots$) to denote constructors.

Note, we compute a unifier for two terms, but not necessarily the most general unifier. 
We can implement the most general unification in \textsc{miniKanren}, but achieving the comparable performance using 
relational verifiers requires additional check that the unifier is indeed the most general. 
We are currently working on the implementation of such relational verifier. 

\begin{table}
\centering
\begin{tabular}{c|c|c|c}
\multirow{ 2}{*}{Terms} & 
f(X, a) & f(a \% b \% nil, c \% d \% nil, L) & f(X, X, g(Z, t))  \\
\cline{2-4} &
f(a, X) & f(X \% XS, YS, X \% ZS) & f(g(p, L), Y, Y)  \\
\hline\hline
Only conversion               & 0.01s  &  >300s & >300s \\
\hline
Backward oriented conversion  & 0.01s  &  0.11s & 2.26s  \\
\hline
Conversion and CPD            & 0.01s  &  0.07s & 0.90s  \\
\hline

Scheme interpreter            & 0.04s  & 5.15s & >300s    \\
\end{tabular}
 \caption{Searching for a unifier of two terms}
    \label{tab:uni}
\end{table}

Here four solvers compare to each other similarly to the previous problem: unnesting demonstrates the worst execution time, relational interpretation in Scheme is a little better, while unnesting aimed at backward execution and conjunctive partial deduction significantly improve the performance. 

There exist pairs of terms, for which either of the solvers fails to compute a unifier under 300 seconds. 
The example of such terms is ``\lstinline{f(A,B,C,A,B,C,D)}'' and ``\lstinline{f(B,C,D,x(R,S),x(a,T),x(Q,b),x(a,b))}''. 
This is caused by how general and declarative the verifier is: there is nothing in it to restrict the search space. 
We can modify the verifier with the additional check to ensure that there are no bound variables in the candidate unifier. 
This modification restricts the search space when there are many variables in the input terms.
But it also changes the semantics of the initial verifier and, as a consequence, the solvers: only idempotent unifiers can be found. 

To sum up, we demonstrated by two examples that it is possible to create problem solvers from verifiers by using relational conversion 
and conjunctive partial deduction. Currently conjunctive partial deduction improves the performance the most, as compared to 
interpreting verifiers with Scheme relational interpreter or doing relational conversion which is solely aimed at backward or 
forward execution.

\section{Related Works}
\label{sec:related}

Pattern matching can be considered as a generalization of conventional conditional control-flow construct ``\lstinline|if .. then .. else|'' and in principle
can be decomposed into a nested hierarchy of those; from this standpoint the problem of pattern matching implementation can be considered trivial. However,
some decompositions are obviously better than others. We repeat here an example from~\cite{maranget2008} to demonstrate this difference (see Fig.~\ref{fig:match-example}).
Here we match a triple of boolean values $x$, $y$, and $z$ against four patterns (Fig.~\ref{fig:matching-example1}; we use \textsc{OCaml}~\cite{ocaml} as
reference language). The na\"{i}ve implementation of this example is shown on Fig.~\ref{fig:matching-example2}; however if we decide to match $y$ first the result
becomes much better (Fig.~\ref{fig:matching-example3}).

\begin{figure}[t]
\begin{subfigure}[t]{0.25\linewidth}
\centering
\begin{lstlisting}
match x,y,z with
| _, F, T -> 1
| F, T, _ -> 2
| _, _, F -> 3
| _, _, T -> 4
\end{lstlisting}
\vskip18.5mm
\caption{Pattern matching}
\label{fig:matching-example1}
\end{subfigure}
\hspace{0.5cm}
\begin{subfigure}[t]{0.3\linewidth}
\centering
\begin{lstlisting}
if x then
  if y then
    if z then 4 else 3
  else
    if z then 1 else 3
else
  if y then 2
  else
    if z then 1 else 3
\end{lstlisting}
\caption{A correct but non-optimal implementation}
\label{fig:matching-example2}
\end{subfigure}
\hspace{1.0cm}
\begin{subfigure}[t]{0.3\linewidth}
\centering
\begin{lstlisting}
if y then
  if x then
    if z then 4 else 3
  else 2
else
  if z then 1 else 3
\end{lstlisting}
\vskip13.5mm
\caption{Optimal implementation}
\label{fig:matching-example3}
\end{subfigure}
\caption{Pattern matching implementation example} 
\label{fig:match-example}
\end{figure}

\begin{comment}
\begin{figure}[ht]
\begin{minipage}[b]{0.3\linewidth}
\centering
\label{fig:figure1}
\end{minipage}
\hspace{0.5cm}
\begin{minipage}[b]{0.3\linewidth}
\centering
\begin{lstlisting}
switch x with 
| true -> 
    switch y with 
    | true -> 
       switch z with 
       | true -> 4
       | _ -> 3
    | _ -> 
      switch z with 
      | true -> 1
      | _ -> 3 
| _ -> 
   switch y with 
   | true -> 2 
   | _ -> if z then 1 else 3
\end{lstlisting}
\end{minipage}
\hspace{0.5cm}
\begin{minipage}[b]{0.3\linewidth}
\centering
\end{minipage}
\end{figure}
\end{comment}


%clasification 1
%Although semantics of pattern matching can be given as a sequence of srutinee's sub expression comparisons (Figure~\ref{fig:matchpatts}) effective compilers don't follow
%this approach. 

The quality of a pattern matching implementation can be measured in various ways. One can either optimise the run time cost by minimizing the amount of checks performed, or the static
cost by minimizing the size of the generated code. \emph{Decision trees}
%~\cite{?} 
% I removed cite because 
% 1) can't find a good citation for it not in the field of compilers
% 2) folks in other papers don't refernece them. Only backtracking automata are begin referenced
are considered suitable for the first criterion as they check every subexpression no more than once.
However, minimizing the size of decision tree is known to be NP-hard~\cite{baudinet1985tree}, and as a rule various heuristics, using, for example,
the number of nodes, the length of the longest path and the average length of all paths are applied during compilation. In \cite{Scott2000WhenDM} the results of experimental
evaluation of nine heuristics for Standard ML of New Jersey are reported.

For minimizing the static cost \emph{backtracking automata} can be used since they admit a compact representation but in some cases can perform repeated checks.

%clasification 2
There is a certain difference in dealing with pattern matching in strict and non-strict languages. For strict languages checking sub-expressions of the scrutinee in any order is allowed.
The pattern matching implementation for strict languages can operate in \emph{direct} or \emph{indirect} styles. In the direct style the construction of an implementation is done explicitly. In indirect the construction of implementation requires some post-processing, which can vary from easy simplifications to complicated supercompilation
techniques~\cite{sestoft1996}. The main drawback of indirect approach is that the size of intermediate data structures can be exponentially large.

For non-strict languages pattern matching should evaluate only those sub-expressions which are necessary for performing pattern matching. If not done carefully pattern matching can
change the termination behavior of the program. In general non-strict languages put more constraints on pattern matching and thus admit a smaller  set of heuristics. 
A few approaches for checking sub-expressions in lazy languages have been proposed. In~\cite{augustsson1985} a simple left-to-right order of subexpression checking was proposed
with a proof that this particular order doesn't affect termination. The backtracking automaton being built takes a form of a DAG to reduce the code size. A few refinements have been added in~\cite{wadler1987}
as a part of textbook~\cite{peytonjones1987} on the implementation of lazy functional languages. The approach from this book is used in the current version of GHC~\cite{marlow2012the}.
%~\footnote{The Glasgow Haskel Compiler, \url{http://www.haskell.org/ghc/}}
\cite{laville1991} models values in lazy languages using \emph{partial terms}, although it doesn't scale to types with infinite sets of constructors (like integers). The approach doesn't
test all subexpressions from left to right as does~\cite{augustsson1985} but aims to  avoid performing unnecessary checks by constructing \emph{lazy automaton}. Pattern matching for
lazy languages has been compiled also to decision trees~\cite{maranget1992} and later into \emph{decision DAGs} which in some cases allows the compiler to make the code
smaller~\cite{maranget1994}.

%about automata
The inefficiency of backtracking automata have been
improved in~\cite{maranget2001}. The approach utilizes a matrix representation for pattern matching. It splits the current matrix according to constructors in the
first column and reduces the task to compiling matrices with fewer rows. The technique is indirect; in the end a few optimizations are performed by introducing
special \emph{exit} nodes to the compiled representation. %No preprocessing is required for this scheme: or-pattern receives a special treatment during compilation process.
The approach from this paper is used in the current implementation of the \textsc{OCaml} compiler.

The previous approach uses the first column to split the matrix. In~\cite{maranget2008} the \emph{necessity} heuristic has been introduced which recommends which column should be
used to perform the split. Good decision trees which are constructed in this work can perform better in corner cases than~\cite{maranget2001}, but for practical use the
difference is insignificant.

While existing approaches deliver appropriate solutions for certain forms of pattern matching constructs, they have to be extended in an \emph{ad hoc} manner each time
the syntax and semantics of pattern matching construct changes. For example, besides a simple conventional form of pattern matching there are a number of extensions:
guards (first appeared in KRC language~\cite{turner2013}), disjunctive patterns~\cite{ocaml}, non-linear patterns~\cite{mcbride1969symbol}, active patterns~\cite{activepatterns} and pattern matching for polymorphic variants~\cite{Garrigue98} 
%and generalized algebraic datatypes~\cite{?}) 
which require a separate customized algorithms to be developed.

\begin{comment}
There are a few different approaches for compiling pattern matching. GHC is using influential paper~\cite{Jones1987}, OCaml is currently based on~\cite{maranget2001} although a work~\cite{maranget2008} can slightly improve effectiveness of generated code. 

Although semantics of pattern matching can be given as a sequence of scrutinee's sub expression comparisons (Fig.~\ref{fig:matchpatts}) effective compilers don't follow this approach. One can either optimise run time cost by minimizing amount of checks performed or static cost by minimizing the size of generated code. \emph{Decision trees} are good for the first criteria, because they check every subexpression not more than once. \emph{Backtracking automata} are rather compact but in some cases can perform repeated checks.


Minimizing the size of decision tree is  NP-hard (\cite{baudinet1985tree}, without proof) and usually heuristics are applied during compilation, for example: count of nodes, length of the longest path, average length of all paths. The paper~\cite{Scott2000WhenDM} performs experimental evaluation of 9  heuristics on the base of for Standard ML of New Jersey.


The matching compilers for strict languages can work in \emph{direct} or \emph{indirect} styles. The first ones return effective code immediately. In the second style to construct final answer some post processing is required. It can vary from easy simplifications to complicated supercompilation techniques~\cite{sestoft1996}. The main drawback of indirect style is that size of intermediate data structures can be exponentially large.

For strict languages checking  sub expressions of scrutinee in any order is allowed. For lazy languages pattern matching should evaluate only these sub expressions which are necessary for performing pattern matching. If not careful pattern matching can change termination behavior of the program.  In general lazy languages setup more constraints on pattern matching and because of that allow lesser set of heuristics.

A few approaches for checking sub expressions in lazy languages has been proposed~\cite{augustsson1985,laville1991}. \cite{laville1991} models values in lazy languages using \emph{partial terms}, although this approach doesn't scale to types with infinite constructor sets (like integers). In  the \cite{suarez1993} the similar approach is extended by special treatment of overlapping patterns. Pattern matching has been compiled to decision trees~\cite{maranget1992} and later \cite{maranget1992} into \emph{decision DAGs} that allow in some cases to make code smaller.

The first works on compilation to backtracking automaton are~\cite{augustsson1985,wadler1987}. 

The inefficiency of backtracking automaton has been improved in~\cite{maranget2001}. The approach utilizes matrix representation for pattern matching. It splits current matrix  according to constructors in the first column and reduces the task to compiling matrices with less rows. The technique is indirect, in the end a few optimizations are performed by introducing special \emph{exit} nodes to the compiled representation.
No preprocessing is required for this scheme: or-pattern receive a special treatment during compilation process.
 The approach from this paper is used in current implementation of OCaml compiler.

Previous approach uses first column to split the matrix. In~\cite{maranget2008} has been introduced \emph{necessity} heuristic that recommends which column should be used to perform split. Good decision trees that are constructed in this work can perform better in corner cases than~\cite{maranget2001} but for practical cases the difference is insignificant.

To summarize, compilers can try to optimize pattern matching either for guaranteed code speed or for guaranteed code size. There are distinct techniques to minimize drawbacks of both approaches.
\end{comment}

\section{Conclusion and Future Work}

In this paper, we presented a certified formal semantics for core \textsc{miniKanren} and proved some of its basic properties
(including interleaving search completeness), which are believed to hold in existing implementations.
We also derived a semantics for conventional SLD resolution with cut and extracted two certified reference interpreters.
We consider our work as the initial setup for the future development of \textsc{miniKanren} semantics.

The language we considered here lacks many important features, which are already introduced
and employed in many implementations. Integrating these extensions~--- in the first hand, disequality constraints,~--- into
the semantics looks a natural direction for future work. We are also going to address the problems of proving some
properties of relational programs (equivalence, refutational completeness, etc.).


\bibliographystyle{ACM-Reference-Format}
\bibliography{main}

\end{document}
