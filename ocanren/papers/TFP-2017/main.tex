\documentclass{llncs}

\usepackage{makeidx}
\usepackage{amssymb}
\usepackage{listings}
\usepackage{indentfirst}
\usepackage{verbatim}
\usepackage{amsmath}
\usepackage{graphicx}
\usepackage{xcolor}
\usepackage{url}
\usepackage{stmaryrd}
\usepackage{xspace}
\usepackage{comment}
\usepackage{wrapfig}
\usepackage{placeins}
\usepackage{tabularx}
\usepackage{ragged2e}
\usepackage{subcaption}
\captionsetup{compatibility=false}
%\usepackage{natbib}
%\usepackage [sorting = none] {biblatex}
%\addbibresource {main.bib}

\def\transarrow{\xrightarrow}
\newcommand{\setarrow}[1]{\def\transarrow{#1}}

\newcommand{\trule}[2]{\frac{#1}{#2}}
\newcommand{\crule}[3]{\frac{#1}{#2},\;{#3}}
\newcommand{\withenv}[2]{{#1}\vdash{#2}}
\newcommand{\trans}[3]{{#1}\transarrow{#2}{#3}}
\newcommand{\ctrans}[4]{{#1}\transarrow{#2}{#3},\;{#4}}
\newcommand{\llang}[1]{\mbox{\lstinline[mathescape]|#1|}}
\newcommand{\pair}[2]{\inbr{{#1}\mid{#2}}}
\newcommand{\inbr}[1]{\left<{#1}\right>}
\newcommand{\highlight}[1]{\color{red}{#1}}
\newcommand{\ruleno}[1]{\eqno[\scriptsize\textsc{#1}]}
\newcommand{\inmath}[1]{\mbox{$#1$}}
\newcommand{\lfp}[1]{fix_{#1}}
\newcommand{\gfp}[1]{Fix_{#1}}
\newcommand{\vsep}{\vspace{-2mm}}
\newcommand{\supp}[1]{\scriptsize{#1}}
\newcommand{\G}{\mathfrak G}
\newcommand{\sembr}[1]{\llbracket{#1}\rrbracket}
\newcommand{\cd}[1]{\texttt{#1}}
\newcommand{\miniKanren}{miniKanren\xspace}
\newcommand{\ocanren}{OCanren\xspace}
\newcommand{\free}[1]{\boxed{#1}}
\newcommand{\binds}{\;\mapsto\;}
\newcommand{\dbi}[1]{\mbox{\bf{#1}}}

\let\emptyset\varnothing
\setlength{\abovecaptionskip}{3pt plus 3pt minus 2pt}
\setlength{\belowcaptionskip}{-10pt plus 3pt minus 2pt}

\lstdefinelanguage{ocanren}{
keywords={fresh, let, in, match, with, when, class, type,
object, method, of, rec, repeat, until, while, not, do, done, as, val, inherit,
new, module, sig, deriving, datatype, struct, if, then, else, open, private, virtual, include, success, failure,
true, false},
sensitive=true,
commentstyle=\small\itshape\ttfamily,
keywordstyle=\ttfamily\underbar,
identifierstyle=\ttfamily,
basewidth={0.5em,0.5em},
columns=fixed,
fontadjust=true,
literate={fun}{{$\lambda$}}1 {->}{{$\to$}}3 {===}{{$\equiv$}}1 {=/=}{{$\not\equiv$}}1 {|>}{{$\triangleright$}}3 {|||}{{$\vee$}}2 {/\\}{{$\wedge$}}2 {^}{{$\uparrow$}}1,
morecomment=[s]{(*}{*)}
}

\lstset{
mathescape=true,
%basicstyle=\small,
identifierstyle=\ttfamily,
keywordstyle=\bfseries,
commentstyle=\scriptsize\rmfamily,
basewidth={0.5em,0.5em},
fontadjust=true,
language=ocanren
}

\usepackage{letltxmacro}
\newcommand*{\SavedLstInline}{}
\LetLtxMacro\SavedLstInline\lstinline
\DeclareRobustCommand*{\lstinline}{%
  \ifmmode
    \let\SavedBGroup\bgroup
    \def\bgroup{%
      \let\bgroup\SavedBGroup
      \hbox\bgroup
    }%
  \fi
  \SavedLstInline
}
%\addtolength{\parskip}{-2pt}
%\setlength{\parskip}{0pt}
\setlength{\belowcaptionskip}{-15pt}

\pagestyle{plain}
\begin{document}
\sloppy
\mainmatter

\title{Typed Relational Conversion\thanks{This work is supported by RFBR grant No 18-01-00380.}}

\author{
  Petr Lozov\inst{1} \and Andrei Vyatkin\inst{2} \and Dmitry Boulytchev\inst{3}
}

\institute{
St.Petersburg State University,\\
Universitetski pr., 28, 198504, St.Petersburg, Russia,\\
\email{lozov.peter@gmail.com}
\and
St.Petersburg State University,\\
Universitetski pr., 28, 198504, St.Petersburg, Russia,\\
\email{dewshick@gmail.com}
\and
St.Petersburg State University,\\
Universitetski pr., 28, 198504, St.Petersburg, Russia,\\
JetBrains Research\\
Universitetskaya emb., 7-9-11, bldg. 5A, 199034, St.Petersburg, Russia,\\
\email{dboulytchev@math.spbu.ru}}

\maketitle

\begin{abstract}
We address the problem of transforming typed functional programs into relational form. 
In this form, a program can be run in various ``directions'' with some arguments left free, 
making it possible to acquire different behaviors from a single specification. We specify the 
syntax, typing rules and semantics for the source language as well as its relational extension, 
describe the conversion and prove its correctness both in terms of typing and dynamic semantics. 
We also discuss the limitations of our approach, present the implementation of the conversion for 
the subset of OCaml and evaluate it on a number of realistic examples.
\end{abstract}

\section{Introduction}
\label{sec:intro}

Verifying a solution for a problem is much easier than finding one~--- this common wisdom can be confirmed by anyone who used 
both to learn and to teach. This observation can be justified by its theoretical applications, thus being more than informal knowledge. For example, let us have a language $\mathcal{L}$. If there is a predicate $V_\mathcal{L}$ such~that
\[
\forall\omega\;:\;\omega\in\mathcal{L}\;\Longleftrightarrow\;\exists p_\omega\;:\;V_\mathcal{L}(\omega,p_\omega)
\]
(with $p_\omega$ being of size, polynomial on $\omega$) and we can recognize $V_\mathcal{L}$ in a polynomial time, then we call $\mathcal{L}$ to be in the class $NP$~\cite{Garey:1990:CIG:574848}. Here $p_\omega$ plays role of a justification (or proof) for the fact $\omega\in\mathcal{L}$. For example, if
$\mathcal{L}$ is a language of all hamiltonian graphs, then $V_\mathcal{L}$ is a predicate which takes a graph $\omega$ and some path $p_\omega$ and verifies that $p_\omega$ is indeed a hamiltonial path in $\omega$. The implementation of the predicate $V_\mathcal{L}$, however, tells us very little about the \emph{search procedure} which would calculate $p_\omega$ as a function of $\omega$. For the whole class of $NP$-complete problems no polynomial search procedures are known, and their existence at all is a long-standing problem in the complexity theory.

There is, however, a whole research area of \emph{relational interpreters}, in which a very close problem is addressed. Given a language $\mathcal{L}$, its \emph{interpreter} is a function \lstinline|eval$_\mathcal{L}$| which takes a program $p^\mathcal{L}$ in the language $\mathcal{L}$ and an input $i$ and calculates some output such that
\[
\mbox{\lstinline|eval$_\mathcal{L}$|}(p^\mathcal{L}, i)=\sembr{p^\mathcal{L}}_{\mathcal L}\,(i)
\]
where $\sembr{\bullet}_{\mathcal L}$ is the semantics of the language $\mathcal{L}$. In these terms, a verification predicate $V_\mathcal{L}$ can be
considered as an interpreter which takes a program $\omega$, its input $p_\omega$ and returns $true$ or \false. A \emph{relational} interpreter is an interpreter which is implemented not as a function \lstinline|eval$_\mathcal{L}$|, which calculates the output from a program and its input, but as a relation \lstinline|eval$^o_\mathcal{L}$|
which connects a program with its input and output. This alone would not have much sense, but if we allow the arguments of \lstinline|eval$^o_\mathcal{L}$|
to contain \emph{variables} we can consider relational interpreter as a generic search procedure which determines the values for these variables making the
relation hold. Thus, with relational interpreter it is possible not only to calculate the output from an input, but also to run a program in 
an opposite ``direction'', or to synthesize a program from an input-output pair, etc. In other words, relational verification predicate is capable
(in theory) to both \emph{verify} a solution and \emph{search} for it.

Implementing relational interpreters amounts to writing it in a relational language. In principle, any conventional language for logic programming
(Prolog~\cite{lozov:prolog}, Mercury~\cite{somogyi1996execution}, etc.) would make the job. However, the abundance of extra-logical features and the incompleteness of default search
strategy put a number of obstacles on the way. There is, however, a language specifically designed for pure relational programming, and, in a
narrow sense, for implementing relational interpreters~--- \textsc{miniKanren}~\cite{lozov:TheReasonedSchemer}. Relational interpreters, implemented
in \textsc{miniKanren}, demonstrate all their expected potential: they can synthesize programs by example, search for errors in partially defined programs~\cite{lozov:seven}, produce self-evaluated programs~\cite{lozov:quines}, etc. However, all these results are obtained for a family
of closely related Scheme-like languages and require a careful implementation and even some \emph{ad-hoc} optimizations in the relational
engine. 

From a theoretical standpoint a single relational interpreter for a Turing-complete language is sufficient: indeed, any other interpreter
can be turned into a relational one just by implementing it in a language, for which relational interpreter already exists. However, the overhead
of additional interpretation level can easily make this solution impractical. The standard way to tackle the problem is partial evaluation or specialization~\cite{jones1993partial}.
A \emph{specializer} \lstinline|spec$_\mathcal{M}$| for a language $\mathcal{M}$ for any program $p^\mathcal{M}$ in this language and its partial input $i$ returns some program which, being applied to the residual input $x$, works exactly as the original program on both $i$ and~$x$:
\[
\forall x\;:\;\sembr{\mbox{\lstinline|spec$_\mathcal{M}$|}\,(p^\mathcal{M}, i)}_\mathcal{M}\,(x)=\sembr{p^\mathcal{M}}_\mathcal{M}\,(i, x).
\]

If we apply a specializer to an interpreter and a source program, we obtain what is called \emph{the first Futamura projection}~\cite{futamura1971partial}:
\[
\forall i\;:\; \sembr{\mbox{\lstinline|spec$_\mathcal{M}$|}\,(\mbox{\lstinline|eval$^\mathcal{M}_\mathcal{L}$|}, p^\mathcal{L})}_\mathcal{M}\,(i)=\sembr{\mbox{\lstinline|eval$^\mathcal{M}_\mathcal{L}$|}}_\mathcal{M}\,(p^\mathcal{L}, i).
\]
Here we added an upper index $\mathcal{M}$ to \lstinline|eval$_\mathcal{L}$| to indicate that we consider it as a program in 
the language $\mathcal{M}$. In other words, the first Futamura projection specializes an interpreter for a concrete program, 
delivering the implementation of this program in the language of interpreter implementation. An important property of
a specializer is \emph{Jones-optimality}~\cite{jones1993partial}, which holds when it is capable to completely
eliminate the interpretation overhead in the first Futamura projection. In our case $\mathcal{M}=\mbox{\textsc{miniKanren}}$, 
from which we can conclude that in order to eliminate the interpretation overhead we need a Jones-optimal specializer for \textsc{miniKanren}. 
Although implementing a Jones-optimal specializer is not an easy task even for simple functional languages, there is a Jones-optimal specializer for a logical language~\cite{leuschel2004specialising}, but not for \textsc{miniKanren}. 

The contribution of this paper is as follows:

\begin{itemize}
\item We demonstrate the applicability of relational programming and, in particular, relational interpreters for the task of
turning verifiers into solvers.
\item To obtain a relational verifier from a functional specification we apply \emph{relational conversion}~\cite{lozov:miniKanren,lozov:conversion}~---
a technique which for a first-order functional program directly constructs its relational counterpart. Thus, we introduce a number
of new relational interpreters for concrete search problems.
\item We employ supercompilation in the form of conjunctive partial deduction (CPD)~\cite{de1999conjunctive} to
eliminate the redundancy of a generic search algorithm caused by partial knowledge of its input.
\item We give a number of examples and perform an evaluation of various solutions for the approach we address.
\end{itemize}

Both relational conversion and conjunctive partial deduction are done in an automatic manner. The only thing one needs to specify is the known arguments or the execution direction of a relation. 

As concrete implementation of \textsc{miniKanren} we use \textsc{OCanren}~\cite{lozov:ocanren}~--- its embedding in \textsc{OCaml}; we use
\textsc{OCaml} to write functional verifiers; our prototype implementation of conjunctive partial deduction is written in \textsc{Haskell}.

The paper is organized as follows. In Section~\ref{sec:example} we give a complete example of solving a concrete problem~--- searching for
a path in a graph,~--- with relational verifier. Section \ref{sec:conversion} recalls the cornerstones of relational programming in 
\textsc{miniKanren} and the relational conversion technique. In Section~\ref{sec:cpd} we describe how conjunctive partial deduction was 
adapted for relational programming. Section~\ref{sec:eva} presents the evaluation results for concrete solvers built using the technique
in question. The final section concludes.

\documentclass[submission,copyright,creativecommons]{eptcs}
%\providecommand{\event}{SOS 2007} % Name of the event you are submitting to

\usepackage{breakurl}             % Not needed if you use pdflatex only.
\usepackage{underscore}           % Only needed if you use pdflatex.
\usepackage{amssymb}
\usepackage{listings}
\usepackage{indentfirst}
\usepackage{verbatim}
\usepackage{amsmath, amsthm, amssymb}
\usepackage{graphicx}
\usepackage{url}
\usepackage{hyperref}
\usepackage{xspace}
\usepackage{placeins}

\lstdefinelanguage{scheme}{
keywords={define, conde, fresh},
sensitive=true,
%basicstyle=\small,
commentstyle=\scriptsize\rmfamily,
keywordstyle=\ttfamily\underbar,
identifierstyle=\ttfamily,
basewidth={0.5em,0.5em},
columns=fixed,
fontadjust=true,
literate={==}{{$\equiv$}}1
}

\lstdefinelanguage{ocaml}{
keywords={fresh, conde, let, begin, end, in, match, type, and, fun, function, try, with, when, class,
object, method, of, rec, repeat, until, while, not, do, done, as, val, inherit,
new, module, sig, deriving, datatype, struct, if, then, else, open, private, virtual, include, @type},
sensitive=true,
commentstyle=\small\itshape\ttfamily,
keywordstyle=\ttfamily\underbar,
identifierstyle=\ttfamily,
basewidth={0.5em,0.5em},
columns=fixed,
fontadjust=true,
literate={->}{{$\to\;\;$}}3 {===}{{$\equiv$}}3 {=/=}{{$\not\equiv$}}3 {|>}{{$\triangleright$}}3,
morecomment=[s]{(*}{*)}
}

\lstset{
mathescape=true,
identifierstyle=\ttfamily,
keywordstyle=\bfseries,
commentstyle=\scriptsize\rmfamily,
basewidth={0.5em,0.5em},
fontadjust=true,
language=ocaml
}

\sloppy

\newcommand{\miniKanren}{miniKanren\xspace}

\title{Typed Embedding of a Relational Language in OCaml}
\author{Dmitry Kosarev
\institute{Saint Petersburg State University\\ Saint Petersburg, Russia}
\email{Dmitrii.Kosarev@protonmail.ch}
\and
Dmitry Boulytchev
\institute{Saint Petersburg State University\\ Saint Petersburg, Russia}
\email{dboulytchev@math.spbu.ru}
}
\def\titlerunning{Typed Embedding of a Relational Language in OCaml}
\def\authorrunning{Dmitry Kosarev, Dmitry Boulytchev}
\begin{document}
\maketitle

\begin{abstract}
We present an implementation of the relational programming language \miniKanren as a set
of combinators and syntax extensions for OCaml. The key feature of our approach is
\emph{polymorphic unification}, which can be used to unify data structures of arbitrary types.
In addition we provide a useful generic programming pattern to systematically develop relational
specifications in a typed manner, and address the problem of integration of relational subsystems into
functional applications.
\end{abstract}

\section{Introduction}
\label{sec:intro}

Verifying a solution for a problem is much easier than finding one~--- this common wisdom can be confirmed by anyone who used 
both to learn and to teach. This observation can be justified by its theoretical applications, thus being more than informal knowledge. For example, let us have a language $\mathcal{L}$. If there is a predicate $V_\mathcal{L}$ such~that
\[
\forall\omega\;:\;\omega\in\mathcal{L}\;\Longleftrightarrow\;\exists p_\omega\;:\;V_\mathcal{L}(\omega,p_\omega)
\]
(with $p_\omega$ being of size, polynomial on $\omega$) and we can recognize $V_\mathcal{L}$ in a polynomial time, then we call $\mathcal{L}$ to be in the class $NP$~\cite{Garey:1990:CIG:574848}. Here $p_\omega$ plays role of a justification (or proof) for the fact $\omega\in\mathcal{L}$. For example, if
$\mathcal{L}$ is a language of all hamiltonian graphs, then $V_\mathcal{L}$ is a predicate which takes a graph $\omega$ and some path $p_\omega$ and verifies that $p_\omega$ is indeed a hamiltonial path in $\omega$. The implementation of the predicate $V_\mathcal{L}$, however, tells us very little about the \emph{search procedure} which would calculate $p_\omega$ as a function of $\omega$. For the whole class of $NP$-complete problems no polynomial search procedures are known, and their existence at all is a long-standing problem in the complexity theory.

There is, however, a whole research area of \emph{relational interpreters}, in which a very close problem is addressed. Given a language $\mathcal{L}$, its \emph{interpreter} is a function \lstinline|eval$_\mathcal{L}$| which takes a program $p^\mathcal{L}$ in the language $\mathcal{L}$ and an input $i$ and calculates some output such that
\[
\mbox{\lstinline|eval$_\mathcal{L}$|}(p^\mathcal{L}, i)=\sembr{p^\mathcal{L}}_{\mathcal L}\,(i)
\]
where $\sembr{\bullet}_{\mathcal L}$ is the semantics of the language $\mathcal{L}$. In these terms, a verification predicate $V_\mathcal{L}$ can be
considered as an interpreter which takes a program $\omega$, its input $p_\omega$ and returns $true$ or \false. A \emph{relational} interpreter is an interpreter which is implemented not as a function \lstinline|eval$_\mathcal{L}$|, which calculates the output from a program and its input, but as a relation \lstinline|eval$^o_\mathcal{L}$|
which connects a program with its input and output. This alone would not have much sense, but if we allow the arguments of \lstinline|eval$^o_\mathcal{L}$|
to contain \emph{variables} we can consider relational interpreter as a generic search procedure which determines the values for these variables making the
relation hold. Thus, with relational interpreter it is possible not only to calculate the output from an input, but also to run a program in 
an opposite ``direction'', or to synthesize a program from an input-output pair, etc. In other words, relational verification predicate is capable
(in theory) to both \emph{verify} a solution and \emph{search} for it.

Implementing relational interpreters amounts to writing it in a relational language. In principle, any conventional language for logic programming
(Prolog~\cite{lozov:prolog}, Mercury~\cite{somogyi1996execution}, etc.) would make the job. However, the abundance of extra-logical features and the incompleteness of default search
strategy put a number of obstacles on the way. There is, however, a language specifically designed for pure relational programming, and, in a
narrow sense, for implementing relational interpreters~--- \textsc{miniKanren}~\cite{lozov:TheReasonedSchemer}. Relational interpreters, implemented
in \textsc{miniKanren}, demonstrate all their expected potential: they can synthesize programs by example, search for errors in partially defined programs~\cite{lozov:seven}, produce self-evaluated programs~\cite{lozov:quines}, etc. However, all these results are obtained for a family
of closely related Scheme-like languages and require a careful implementation and even some \emph{ad-hoc} optimizations in the relational
engine. 

From a theoretical standpoint a single relational interpreter for a Turing-complete language is sufficient: indeed, any other interpreter
can be turned into a relational one just by implementing it in a language, for which relational interpreter already exists. However, the overhead
of additional interpretation level can easily make this solution impractical. The standard way to tackle the problem is partial evaluation or specialization~\cite{jones1993partial}.
A \emph{specializer} \lstinline|spec$_\mathcal{M}$| for a language $\mathcal{M}$ for any program $p^\mathcal{M}$ in this language and its partial input $i$ returns some program which, being applied to the residual input $x$, works exactly as the original program on both $i$ and~$x$:
\[
\forall x\;:\;\sembr{\mbox{\lstinline|spec$_\mathcal{M}$|}\,(p^\mathcal{M}, i)}_\mathcal{M}\,(x)=\sembr{p^\mathcal{M}}_\mathcal{M}\,(i, x).
\]

If we apply a specializer to an interpreter and a source program, we obtain what is called \emph{the first Futamura projection}~\cite{futamura1971partial}:
\[
\forall i\;:\; \sembr{\mbox{\lstinline|spec$_\mathcal{M}$|}\,(\mbox{\lstinline|eval$^\mathcal{M}_\mathcal{L}$|}, p^\mathcal{L})}_\mathcal{M}\,(i)=\sembr{\mbox{\lstinline|eval$^\mathcal{M}_\mathcal{L}$|}}_\mathcal{M}\,(p^\mathcal{L}, i).
\]
Here we added an upper index $\mathcal{M}$ to \lstinline|eval$_\mathcal{L}$| to indicate that we consider it as a program in 
the language $\mathcal{M}$. In other words, the first Futamura projection specializes an interpreter for a concrete program, 
delivering the implementation of this program in the language of interpreter implementation. An important property of
a specializer is \emph{Jones-optimality}~\cite{jones1993partial}, which holds when it is capable to completely
eliminate the interpretation overhead in the first Futamura projection. In our case $\mathcal{M}=\mbox{\textsc{miniKanren}}$, 
from which we can conclude that in order to eliminate the interpretation overhead we need a Jones-optimal specializer for \textsc{miniKanren}. 
Although implementing a Jones-optimal specializer is not an easy task even for simple functional languages, there is a Jones-optimal specializer for a logical language~\cite{leuschel2004specialising}, but not for \textsc{miniKanren}. 

The contribution of this paper is as follows:

\begin{itemize}
\item We demonstrate the applicability of relational programming and, in particular, relational interpreters for the task of
turning verifiers into solvers.
\item To obtain a relational verifier from a functional specification we apply \emph{relational conversion}~\cite{lozov:miniKanren,lozov:conversion}~---
a technique which for a first-order functional program directly constructs its relational counterpart. Thus, we introduce a number
of new relational interpreters for concrete search problems.
\item We employ supercompilation in the form of conjunctive partial deduction (CPD)~\cite{de1999conjunctive} to
eliminate the redundancy of a generic search algorithm caused by partial knowledge of its input.
\item We give a number of examples and perform an evaluation of various solutions for the approach we address.
\end{itemize}

Both relational conversion and conjunctive partial deduction are done in an automatic manner. The only thing one needs to specify is the known arguments or the execution direction of a relation. 

As concrete implementation of \textsc{miniKanren} we use \textsc{OCanren}~\cite{lozov:ocanren}~--- its embedding in \textsc{OCaml}; we use
\textsc{OCaml} to write functional verifiers; our prototype implementation of conjunctive partial deduction is written in \textsc{Haskell}.

The paper is organized as follows. In Section~\ref{sec:example} we give a complete example of solving a concrete problem~--- searching for
a path in a graph,~--- with relational verifier. Section \ref{sec:conversion} recalls the cornerstones of relational programming in 
\textsc{miniKanren} and the relational conversion technique. In Section~\ref{sec:cpd} we describe how conjunctive partial deduction was 
adapted for relational programming. Section~\ref{sec:eva} presents the evaluation results for concrete solvers built using the technique
in question. The final section concludes.

\section{Related Works}

\label{sec:related-works}

There are two directions of work in the process of 
incorporating negative reasoning in the logic programming:
the first considers the semantics of negation,
and the second is focused mainly on implementation aspects.

The first attempt to give a semantics 
for negation in logic programming 
was done by Clark~\cite{clark1978negation, chan1988constructive}
with his completion semantics.
It was then realized, that 
Clark's semantics has various drawbacks~\cite{van1991well}.

Przymusinski~\cite{przymusinski1989constructive} 
has studied the semantics of stratified logic programs.
He introduced the notion of \emph{perfect model semantics} for such programs.
Stratified logic programs have a variety of good properties,
including the property that each stratified program 
has a unique minimal model.

In an attempt to extend the semantics of negation
to non-stratified programs the 
\emph{well-founded semantics} was proposed~\cite{van1991well}.
However, this semantics is three-valued,
meaning that for some queries it 
can return answer \lstinline{unknown}.
For example, given the relation \lstinline{winning}
(section~\ref{sec:strat}, listing~\ref{lst:game}),
queries \lstinline{winning 'a'} and \lstinline{winning 'b'}
would return \lstinline{unknown}.

An alternative approach is 
\emph{stable model semantics}~\cite{gelfond1988stable}.
Under this semantics, non-stratified logic program
can have several stable models.
Program, that defines \lstinline{winning}, 
has two stable models, 
in one of these models goal 
\lstinline{winning 'a'} succeeds and 
\lstinline{winning 'b'} fails,
in the other \lstinline{winning 'a'} fails
and \lstinline{winning 'b'} succeeds.
Logic programming under stable model semantics
is also known under the name 
answer set programming (ASP).

The works~\cite{stuckey1991constructive, dovier2000necessary}
are theoretical studies of constructive negation
in the context of constraint logic programming.
They give a necessary and sufficient condition for the
constraint structures that are compatible with constructive negation.
Namely, the constraint structure should be \emph{admissible closed}.

From an implementation side,
Chen et al.~\cite{chen1995efficient, chen1996tabled}
developed a \textsc{Prolog} system
based on SLG resolution,
which is sound with respect to well-founded semantics.
However, they used negation as failure
with delaying of non-ground negative subgoals.
\cite{liu1999constructive} is an extension of this system
with the support of the constructive negation.
Works~\cite{bartak1998constructive, alvez2004constructive} 
implement a constraint logic programming systems
with the support of constructive negation.
Yet, as with our implementation, the
constructive negation in these systems
supports only equality and disequality
constraints over first-order terms.
We are not aware of any practical implementation 
that is parametric over arbitrary admissible closed constraint structures.

Many tools were developed to 
compute stable models of logic programs,
among them are~\cite{gebser2007clasp, giunchiglia2006answer}.
These systems usually require
to perform grounding of logic program.
The problem of finding stable models of 
ground logic program then is encoded 
as propositional formula and solved 
by some SAT solver.
Unfortunately, some logic programs
do not have finite grounding, 
but even if a program has it,
grounding may cause an exponential blow-up.
Recently, a goal-directed system
for computing stable models was developed~ 
\cite{marple2012goal, marple2017s, arias2018constraint}.
To the best of our knowledge, it is the only ASP system,
that does not require grounding.
The key components of this system are the usage of tabling, 
constructive negation, coinductive logic programming, 
and non-monotonic reasoning check.
It is an interesting and challenging task
to extend \textsc{MiniKanren} with the support of 
stable model semantics in the spirit of this line of work.

\section{\miniKanren~--- a Short Presentation}
\label{sec:demo}

In this section we briefly describe \miniKanren in its original form, using a canonical example.
\miniKanren is organized as a set of combinators and macros for Scheme/Racket, designated to describe
a search for the solution of a certain \emph{goal}. There are four domain-specific constructs
to build \emph{goals}:

\begin{itemize}
\item Syntactic unification~\cite{Unification} in the form \lstinline[language=scheme]{(== $t_1$ $t_2$)}, where $t_1$, $t_2$ are
some \emph{terms}; unification establishes a syntactic basis for all other goals. If there is a unifier for
two given terms, the goal is considered satisfied, a most general unifier is kept as a partial solution, and the execution
of current branch continues. Otherwise, the current branch backtracks.

\item Disequality constraint~\cite{CKanren} in the form \lstinline[language=scheme]{($\not\equiv$ $t_1$ $t_2$)}, where
$t_1$, $t_2$ are some terms; a disequality constraint prevents all branches (starting from the current), in which the
specified terms are equal (w.r.t. the search state), from being considered.

\item Conditional construct in the form

\begin{lstlisting}[language=scheme]
   (conde
      [$g^1_1\;\;g^1_2\;\;\dots\;\;g^1_{k_1}$]
      [$g^2_1\;\;g^2_2\;\;\dots\;\;g^2_{k_2}$]
      $\ldots$
      [$g^n_1\;\;g^n_2\;\;\dots\;\;g^n_{k_n}$]
   )
\end{lstlisting}

where each $g^i_j$ is a goal. A \lstinline{conde} goal considers each collection of subgoals, surrounded by square brackets, as
implicit conjunction (so \lstinline[language=scheme]{[$g^i_1\;\;g^i_2\;\;\dots\;\;g^i_{k_i}$] } is considered as a
conjunction of all $g^i_j$) and tries to satisfy each of them independently~--- in other words, operates on them
as a disjunction.

\item Fresh variable introduction construct in the form

\begin{lstlisting}[language=scheme]
   (fresh ($x_1\;\;x_2\;\;\dots\;\;x_k$)
     $g_1$
     $g_2$
     $\ldots$
     $g_n$
   )
\end{lstlisting}

where each $g_i$ is a goal. This form introduces fresh variables $x_1\;\;x_2\;\;\dots\;\;x_k$ and
tries to satisfy the conjunction of all subgoals $g_1\;\;g_2\;\;\dots\;\;g_n$ (these subgoals may contain
introduced fresh variables).
\end{itemize}

As an example consider a list concatenation relation; by a well-established tradition, the names
of relational objects are superscripted by ``$^o$'', hence \lstinline{append$^o$}:

\begin{lstlisting}[mathescape=true,language=scheme,numbers=left,numberstyle=\small,stepnumber=1,numbersep=-5pt]
   (define (append$^o$ x y xy)
      (conde
         [(== '() x) (== y xy)]
         [(fresh (h t ty)
            (== `(,h . ,t) x)
            (== `(,h . ,ty) xy)
            (append$^o$ t y ty))]))
\end{lstlisting}

We interpret the relation ``\lstinline{append$^o$ x y xy}'' as ``the concatenation of \lstinline{x} and \lstinline{y}
equals \lstinline{xy}''. Indeed, if the list \lstinline{x} is empty, then (regardless the content of \lstinline{y}) in order for the relation to hold
the value for \lstinline{xy} should by equal to that of \lstinline{y}~--- hence line 3. Otherwise, \lstinline{x} can be decomposed into the head
\lstinline{h} and the tail \lstinline{t}~--- so we need some fresh variables. We also need the additional variable \lstinline{ty} to designate the list
that is in the relation \lstinline{append$^o$} with \lstinline{t} and \lstinline{y}. Trivial relational reasonings complete the implementation (lines 5-7).

A goal, built with the aid of the aforementioned constructs, can be run by the following primitive:

\begin{lstlisting}[mathescape=true,language=scheme]
   run $n$ ($q_1\dots q_k$) $G$
\end{lstlisting}

Here $n$ is the number of requested answers (or ``*'' for all answers), $q_i$ are fresh query variables, and $G$ is a goal, which can
contain these variables.

The \lstinline{run} construct initiates the search for answers for a given goal and returns a (finite or infinite) list
of answers~--- the bindings for query variables, which represent individual solutions for that query. For example,

\begin{lstlisting}[mathescape=true,language=scheme]
   run 1 (q) (append$^o$ q '(3 4) '(1 2 3 4) )
\end{lstlisting}

\noindent returns a list \lstinline{((1 2))}, which constitutes the answer for a query variable $q$. The process of constructing
the answers from internal data structures of miniKanren interpreter is called \emph{reification}~\cite{WillThesis}.


\section{Streams, States, and Goals}
\label{sec:goals}

This section describes a top-level framework for our implementation. Even though it contains
nothing more than a reiteration of the original implementation~\cite{MicroKanren, CKanren}
in OCaml, we still need some notions to be properly established.

The search itself is implemented using a backtracking lazy stream monad~\cite{KiselyovBacktracking}:

\begin{lstlisting}
   type $\alpha$ stream

   val mplus : $\alpha$ stream -> $\alpha$ stream -> $\alpha$ stream
   val bind  : $\alpha$ stream -> ($\alpha$ -> $\beta$ stream) -> $\beta$ stream
\end{lstlisting}

Monadic primitives describe the shape of the search, and their implementations may
vary in concrete \miniKanren versions.

An essential component of the implementation is a bundle of the following types:

\begin{lstlisting}
   type env         = $\dots$
   type subst       = $\dots$
   type constraints = $\dots$

   type state = env * subst * constraints
\end{lstlisting}

Type \lstinline{state} describes a point in a lazily constructed search tree: type \lstinline{env} corresponds
to an \emph{environment}, which contains some supplementary information (in particular, an environment is needed to
correctly allocate fresh variables), type \lstinline{subst} describes a substitution, which keeps current bindings
for some logical variables, and, finally, type \lstinline{constraints} represents disequality constraints,
which have to be respected. In the simplest case \lstinline{env} is just a counter for the number of the next free
variable, \lstinline{subst} is a map-like structure and \lstinline{constraints} is a list of substitutions.

%In our case, the environment contains some extra information to make it possible to identify variables
%in a constant time. 

\begin{comment}
This check is not needed in faster-miniKanren or any other untyped approach to implement
relational DSL. In case of OCanren it is dangerous to allow reification of a logic variable
(attributed by a type one the first state) in the context of an another state where it is attributed by another type.
Cases like this one can trigger undefined behavior. In the current implementation of logical variables created in the foreign
state are not treated as logic variables. This restriction is difficult to express in types, so the check was moved
from compile-time to runtime. This design decision is unfortunate, however this check doesn't affect perfomance in
any significant way and we don't know any useful program where such a mistake can happen.
\end{comment}

The next cornerstone element is the \emph{goal} type, which is considered as a transformer of a state into
a lazy stream of states:

\begin{lstlisting}
   type goal = state -> state stream
\end{lstlisting}

In terms of the search, a goal nondeterministically performs one step of the search: for a given
node in a search tree it produces its immediate descendants. On the user level the type \lstinline{goal}
is abstract, and states are completely hidden.

Next to last, there are a number of predefined combinators:

\begin{lstlisting}
   val (&&&)      : goal -> goal -> goal
   val (|||)      : goal -> goal -> goal
   val call_fresh : ($v$ -> goal) -> goal
   ....
\end{lstlisting}

Conjunction ``\lstinline{&&&}'' combines the results of its argument goals using \lstinline{bind},
disjunction ``\lstinline{|||}'' concatenates the results using \lstinline{mplus}, abstraction
primitive \lstinline{call_fresh} takes an abstracted goal and applies it to a freshly created
variable. Type $v$ in the last case designates the type for a fresh variable, which we leave
abstract for now. These combinators serve as the bricks for the implementation of conventional
\miniKanren constructs and syntax extensions (\lstinline{conde}, \lstinline{fresh}, etc.)

Finally, there are two primitive goal constructors:

\begin{lstlisting}
   val (===) : $t$ -> $t$ -> goal
   val (=/=) : $t$ -> $t$ -> goal
\end{lstlisting}

The first one is a unification, while the other is a disequality constraint. Here, we again left
the type of terms $t$ abstract; it will be instantiated later.

In the implementation of \miniKanren both of these goals are implemented using unification~\cite{CKanren}; this
is true for us as well. However, due to a drastic difference between the host languages, the implementation of
efficient polymorphic unification itself leads to a number of tricks with typing and data representation, which are
absent in the original version.

In this setting, the run primitive is represented by the following function:

\begin{lstlisting}
   val run : goal -> state stream
\end{lstlisting}

This function creates an initial state and applies a goal to it. The states in the return stream describe
various solutions for the goal. As the stream is constructed lazily, taking elements one by one makes
the search progress.

To discover concrete answers, the state has to be queried for its contents. As a rule, a few variables
are reified in a state, i.e. their bindings in the corresponding substitution are retrieved.
Disequality constraints for free variables have to be reified additionally (e.g. represented as a list of
``forbidden'' terms). As forbidden terms can contain free variables, the constraint reification
process is recursive.

In our case, the reification is a subtle part, since, as we will see shortly, it can not be implemented in a
type-safe fragment of the language.

\section{Polymorphic Unification}
\label{sec:unification}

We consider it rather natural to employ polymorphic unification in a language already equipped
with polymorphic comparison~--- a convenient, but somewhat disputable\footnote{See, for example,
\url{https://blogs.janestreet.com/the-perils-of-polymorphic-compare}} feature. Like polymorphic comparison,
polymorphic unification performs a traversal of values, exploiting intrinsic knowledge of their runtime
representation. The undeniable benefits of this solution are, first, that in order to perform unification
for user types no ``boilerplate'' code is needed, and, second, that this approach seems to deliver the
most efficient implementation. On the other hand, all the pitfalls of polymorphic comparison are inherited as
well; in particular, polymorphic unification loops for cyclic data structures and does not work for functional
values. Since we generally do not expect any reasonable outcome in these cases, the only remaining problem is that
the compiler is incapable of providing any assistance in identifying and avoiding these cases. Another drawback is that
the implementation of polymorphic unification relies on the runtime representation of values and has to be fixed
every time the representation changes.  Finally, as it is written in an unsafe manner using the \lstinline{Obj} interface,
it has to be carefully developed and tested.

An important difference between polymorphic comparison and unification is that the former only inspects its operands,
while the results of unification are recorded in a substitution (a mapping from logical variables to terms), which
later is used to reify answers. So, generally speaking, we have to show that no ill-typed
terms are constructed as a result. Overall, this property seems to be maintained vacuously, since the very
nature of (syntactic) unification is to detect whether some things can be considered equal. Nevertheless there are
different type systems and different unification implementations; in addition \emph{equal things} can be
\emph{differently typed}, so we provide here a correctness justification for a well-defined abstract case, and will
reuse this conclusion for various concrete cases.

First, we consider three alphabets:

$$
\begin{array}{rcl}
  \tau,\dots&-&\mbox{types}\\
  x^\tau,\dots&-&\mbox{typed logic variables}\\
  C_k^{\tau_1\times\tau_2\times\dots\times\tau_k\to\tau} (k\ge 0),\dots&-&\mbox{typed constructors}
\end{array}
$$

The set of all well-formed typed terms is defined by mutual induction for all types:

$$
t^\tau=x^\tau\mid C_k^{\tau_1\times\tau_2\times\dots\times\tau_k\to\tau}(t^{\tau_1},\,t^{\tau_2},\,\dots,\,t^{\tau_k})
$$

For simplicity from now on we abbreviate the notation $C_k^{\tau_1\times\tau_2\times\dots\times\tau_k\to\tau}(t^{\tau_1},\,t^{\tau_2},\,\dots,\,t^{\tau_k})$ into
$C_k^\tau(t^{\tau_1},\,t^{\tau_2},\,\dots,\,t^{\tau_k})$, keeping in mind that for any concrete constructor and for all its occurrences
in arbitrary terms all its subterms in corresponding positions agree in types.

\begin{comment}
We need also to define the notion of a subterm  $t^\tau[p]$ of a term $t^\tau$ at given position $p$:

$$
\begin{array}{rcl}
 p=\epsilon\mid\{1, 2, 3,\dots\}\bullet p&-&\mbox{the set of positions}\\
 t^\tau[\epsilon]=t^\tau&-&\mbox{base case}\\
 C_k^\tau(t_1^{\tau_1},\,t_2^{\tau_2},\dots,\,t_k^{\tau_k})[i\bullet p]=t_i^{\tau_i}[p], 1\le i \le k&-&\mbox{inductive case}
\end{array}
$$
\end{comment}

In this formulation we do not consider any structure over the set of types besides type equality, and we assume all terms are explicitly
attributed to their types at runtime. We employ this property to implement a unification algorithm in regular OCaml, using some
representation for terms and types:

\begin{lstlisting}[mathescape=true]
    val unify : term -> term -> subst option -> subst option
\end{lstlisting}

\noindent where ``\lstinline{term}'' stands for the type representing typed terms, and ``\lstinline{subst}'' stands for the type of
substitution (a partial mapping from logic variables to terms). Unification can fail (hence ``\lstinline{option}'' in the result type),
is performed in the context of existing substitution (hence ``\lstinline{subst}'' in the third argument) and can be
chained (hence ``\lstinline{option}'' in the third argument).

We use exactly the same unification algorithm with triangular substitution as in the reference implementation~\cite{MicroKanren}. We
omit here some not-so-important details (like ``occurs check''), which are kept in the actual implementaion, and refrain from discussing 
the nature and properties of the algorithm
itself (an excellent description, including a certified correctness proof, can be found in~\cite{Kumar}).

The following snippet presents the implementation:

\begin{lstlisting}[mathescape=true,numbers=left,numberstyle=\small,stepnumber=1,numbersep=-5pt]
    let rec unify $t_1^\tau$ $t_2^\tau$ $subst$ =
      let rec walk $s$ $t^\tau$ =
        match $t^\tau$ with
        | $x^\tau$ when $x^\tau\in dom(s)$ -> $\;\;$walk $s$ $(s\;\;x^\tau)$
        | _ -> $t^\tau$
      in
      match $subst$ with
      | None -> None
      | Some $s$ ->
          match walk $s$ $t_1^\tau$, walk $s$ $t_2^\tau$ with
          | $x_1^\tau$, $x_2^\tau$ when $x_1^\tau$ = $x_2^\tau$ -> $subst$
          | $x_1^\tau$, $q_2^\tau$ -> Some ($s\;[x_1^\tau \gets q_2^\tau]$)
          | $q_1^\tau$, $x_2^\tau$ -> Some ($s\;[x_2^\tau \gets q_1^\tau]$)
          | $C^\tau(t_1^{\tau_1},\dots,t_k^{\tau_k})$, $C^\tau(p_1^{\tau_1},\dots,p_k^{\tau_k})$ ->
              unify $t_k^{\tau_k}$ $p_k^{\tau_k}$(.. (unify $t_1^{\tau_1}$ $p_1^{\tau_1}$ $subst$)$..$)
          | $\_$, $\_$ -> None
\end{lstlisting}

We remind the reader that all superscripts correspond to type attributes, which we consider here as
parts of values being manipulated. For example, line 1 means that we apply \lstinline{unify}
to terms $t_1$ and $t_2$, and expect their types to be equal $\tau$. We assume that
at the top level unification is always applied to some terms of the same type and that any
substitution can only be acquired from the empty one by a sequence of unifications.

We are going to show that under these assumptions all type attributes are superfluous~--- they
do not affect the execution of \lstinline{unify} and can be removed. Note that the only place where we
were incapable of providing an explicit type attribute was in line 4, where the result of
substitution application was returned. However, we can prove by induction that any substitution
respects the following property: if a substitution $s$ is defined for a variable $x^\tau$,
then $s\;\;x^\tau$ is attributed with the type $\tau$ (and, consequently, \lstinline{walk $s$ $t^\tau$} always
returns a term of type $\tau$).

Indeed, this property vacuously holds for the empty substitution. Let $s$ be some substitution, for which the
property holds. In the 11 we return an unchanged substitution; in line 10 we perform two calls~---
\lstinline{walk $s$ $t_1^\tau$} and \lstinline{walk $s$ $t_2^\tau$} and match their results. However,
by our induction hypothesis these results are again attributed to the type $\tau$, which justifies the
pattern matching. In line 11 we return the substitution unchanged, in lines 12 and 13 we extend the
existing substitution but preserve the property of interest. Finally, in line 15 we chain a few
applications of \lstinline{unify}; note that, again, all these calls are performed for terms of equal
types, starting from a substitution possessing the property of interest. A simple induction on the
chain length completes the proof.

So, type attributes are inessential~--- they are never analyzed and never restrict pattern matching; hence,
they can be erased completely.
We can notice now that for the representation of terms we can use OCaml's native runtime representation.
It can not be done, however, using regular OCaml~--- we have to utilize the low-level, unsafe interface \lstinline{Obj}.
Note also, we need some way to identify the occurrences of logical variables inside the terms (in the original \miniKanren
implementation the ambiguity between variable and non-variable datum representation is resolved by a convention~--- a luxury
we cannot afford).  We postpone the discussion on this subject until the next section.


%the correctness our of implementation is based on
%the following convention about logical variables: the representation of a logical variable $x^\tau$ must
%correspond to a representation of some value of type $\tau$. This, in turn, makes it somewhat problematic
%to detect a variable occurrence in a term. We postpone the discussion on this subject until Section~\ref{sec:injection}.

We call our implementation \emph{polymorphic}, since at the top level it is defined as

\begin{lstlisting}
   val unify : $\alpha$ -> $\alpha$ -> subst option -> subst option
\end{lstlisting}

The type of substitution is not polymorphic, which means that the compiler completely loses the track
of types of values stored in a substitution. These types are recovered later during the reification-of-answers phase (see Section~\ref{sec:reification}).
Outside the unification the compiler maintains typing, which means that all terms, subterms, and variables agree in their types
in all contexts.

\section{Term Representation and Injection}
\label{sec:injection}

Polymorphic unification, considered in the previous section, works for the values of any type under the assumption that we
are capable of identifying logical variables. The latter depends on the term representation. In the original
implementation all terms are represented as a conventional S-expressions, while logical variables (in a simplest case)~--- as one-element vectors; it's an end user responsibility to respect this convention and refrain from operating
with vectors as a user data.

In our case we want to preserve both strong typing and type inference. Since we have chosen to use polymorphic
unification, it is undesirable to represent logical variables of different types differently (while technically
possible, it would compromise the lightweight approach we used so far). This means that terms with logical
variables have to be typed differently from user-defined data~--- otherwise it would be possible to use
terms in contexts where logical variables are not handled properly. At the same time we do not want term types
to be completely different from user-defined types --- for example, we would like to reuse user-defined constructors, etc.
This considerations boil down to the idea of \emph{logical representation} for a user-defined type. Informally,
a logical representation for the type $t$ is a type $\rho_t$ with a couple of conversion functions:

$$
\begin{array}{rcl}
   \uparrow  \;: t \to \rho_t & - & \mbox{injection}\\
   \downarrow\;: \rho_t \to t & - & \mbox{projection}
\end{array}
$$

The type $\rho_t$ repeats the structure of $t$, but can contain logic variables. So, the injection is total,
while the projection is partial.

It is important to design representations as instances of some generic scheme (otherwise, \miniKanren combinators
could not be properly typed). In addition it is desirable to provide a generic way to build
injection/projection pair in a uniform manner (and, even better, automatically) to lift the burden of
their implementation off the end user shoulders and improve the reliability of the solution. Finally,
the representation must provide a way to detect logic variable occurrences.


%Unification, considered in Section~\ref{sec:unification}, works for values of any types. However, it
%is too generic to be used directly. As long as we use it to unify closed terms, it's OK (but rather meaningless);
%however, for terms with free variables the situation becomes more complicated. Indeed, as we've seen in the
%previous section, a variable $x^\tau$ must have a runtime representation of some value of type $\tau$. As different
%types may not have any values in common, if we wish to unify arbitrary types, we would need to specify the
%ground type for each logical variable explicitly, which would throw our implementation into a non-polymorphic
%realm with no type inference.

In this part we consider two approaches to implementing logical representations.  The first is rather easy to
develop and implement; unfortunately, the implementation demonstrates a poor performance for a number of
important applications. In order to fix this deficiency, we develop a more elaborate technique which
nevertheless reuses some components from the previous one. In Section~\ref{sec:evaluation}
we present the results of performance evaluation for both approaches.

\subsection{Tagged Logical Values}

The first natural solution is to use tagging for representing logical representations.
We introduce the following polymorphic type $[\alpha]$\footnote{In concrete syntax called ``$\alpha\;$\lstinline{logic}''}, which
corresponds to a logical representation of the type $\alpha$:

\begin{lstlisting}
   type $[\alpha]$ = Var of int | Value of $\alpha$
\end{lstlisting}

Informally speaking, any value of type $[\alpha]$ is either a value of type $\alpha$, or a free
logic variable. Note, the constructors of this type cannot be disclosed to an end user, since the only possible way to create a logic variable
should still be by using the ``\lstinline{fresh}'' construct; thus the logic type is abstract in the interface.
Now, we may redefine the signature of abstraction, unification and disequality primitives in the
following manner

\begin{lstlisting}
   val call_fresh : ($[\alpha]$ -> goal) -> goal

   val (===)      : $[\alpha]$ -> $[\alpha]$ -> goal
   val (=/=)      : $[\alpha]$ -> $[\alpha]$ -> goal
\end{lstlisting}

Both unification and disequality constraint would still use the same polymorphic unification; their external, visible type,
however, is restricted to logical types only.

Apart from variables, other logical values can be obtained by injection; conversely, sometimes a logical value can be projected to
a regular one. We supply two basic functions\footnote{In concrete syntax called ``\lstinline{inj}'' and ``\lstinline{prj}''.}
for these purposes

\begin{lstlisting}[mathescape=true]
   val ($\uparrow_\forall$) : $\alpha$ -> $[\alpha]$
   val ($\downarrow_\forall$) : $[\alpha]$ -> $\alpha$

   let ($\uparrow_\forall$) x = Value x
   let ($\downarrow_\forall$) = function Value x -> x | _ -> failwith $\mbox{``not a value''}$
\end{lstlisting}

\noindent which can be used to perform a \emph{shallow} injection/projection. As expected, the injection is total, while the projection is partial.

The shallow pair works well for primitive types; to implement injection/projection for arbitrary types we exploit the
idea of representing regular types as fixed points of functors~\cite{ALaCarte}. For our purposes it is desirable to make
the functors fully polymorphic~--- thus a type, in which we can place a logical variable into arbitrary position,
can be easily manufactured. In addition this approach makes it possible to refactor the existing code to use relational
programming with only minor changes.

To illustrate this approach, we consider an iconic example~--- the list type. Let us have a conventional definition
for a regular polymorphic list in OCaml:

\begin{lstlisting}
   type $\alpha$ list = Nil | Cons of $\alpha$ * $\alpha$ list
\end{lstlisting}

For this type we can only place a logical variable in the position of a list element, but not of the tail, since the tail
always has the type \lstinline{$\alpha$ list}, fixed in the definition of constructor \lstinline{Cons}. In order to create
a full-fledged logical representation, we first have to abstract the type into a fully-polymorphic functor:

\begin{lstlisting}
   type ($\alpha$, $\beta$) $\mathcal L$ = Nil | Cons of $\alpha$ * $\beta$
\end{lstlisting}

Now, the original type can be expressed as

\begin{lstlisting}
   type $\alpha$ list = ($\alpha$, $\alpha$ list) $\mathcal L$
\end{lstlisting}

\noindent and its logical representation~--- as

\begin{lstlisting}
   type $\alpha$ list$^o$ = $[$($[\alpha]$, $\alpha$ list$^o$) $\mathcal L]$
\end{lstlisting}

Moreover, with the aid of conventional functor-specific mapping function

\begin{lstlisting}
   val fmap$_{\mathcal L}$ : ($\alpha$ -> $\alpha^\prime$) -> ($\beta$ -> $\beta^\prime$) -> ($\alpha$, $\beta$) $\mathcal L$ -> ($\alpha^\prime$, $\beta^\prime$) $\mathcal L$
\end{lstlisting}

\noindent both the injection and the projection functions can be implemented:

\begin{lstlisting}[mathescape=true]
   let rec $\uparrow_{\mbox{\texttt{list}}}$ l = $\uparrow_{\forall}$(fmap$_{\mathcal L}$ ($\uparrow_\forall$) $\uparrow_{\mbox{\texttt{list}}}$ l)
   let rec $\downarrow_{\mbox{\texttt{list}}}$ l = fmap$_{\mathcal L}$ ($\downarrow_\forall$) $\downarrow_{\mbox{\texttt{list}}}$ ($\downarrow_\forall$ l)
\end{lstlisting}

As functor-specific mapping functions can be easily written or, better, derived automatically using a number of existing frameworks for
generic programming for OCaml, one can easily provide injection/projection pair for user-defined data types.

We now can address the problem of variable identification during polymorphic unification. As we do not know the types, we cannot discriminate logical
variables by their tags only and, thus, cannot simply use pattern matching. In our implementation we perform a variable test
as follows:

\begin{itemize}
\item in an environment, we additionally keep some unique boxed value~--- the \emph{anchor}~--- created by \lstinline{run} at the moment of initial
state generation; the anchor is inherited unchanged in all derived environments during the search session;
\item we change the logic type definition into

\begin{lstlisting}
   type $[\alpha]$ = Var of int * anchor | Value of $\alpha$
\end{lstlisting}

\noindent making it possible to save in each variable the anchor, inherited from the environment, in which the variable was created;

\item inside the unification, in order to check if we are dealing with a variable, we test the conjunction of the following properties:

  \begin{enumerate}
    \item the scrutinee is boxed;
    \item the scrutinee's tag and layout correspond to those for variables (i.e. the values, created with the constructor \lstinline{Var} of
type \lstinline{[$\alpha$]});
    \item the scrutinee's anchor and the current environment's anchor have equal addresses.
  \end{enumerate}
\end{itemize}

Taking into account that the state type is abstract at the user level, we guarantee that only those variables which were
created during the current run session would pass the test, since the pointer to the anchor is unique among all pointers to a boxed value
and could not be disclosed anywhere but in the variable-creation primitive.

The only thing to describe now is the implementation of the reification stage. The reification is represented by the following
function:

\begin{lstlisting}
   val reify : state -> $[\alpha]$ -> $[\alpha]$
\end{lstlisting}

This function takes a state and a logic value and recursively substitutes all logic variables in that value w.r.t.
the substitution in the state until no occurrences of bound variables are left. Since in our implementation the type of a substitution is
not polymorphic, \lstinline{reify} is also implemented in an unsafe manner. However, it is easy to see that \lstinline{reify}
does not produce ill-typed terms. Indeed, all original types of variables are preserved in a substitution; unification does not
change unified terms, so all terms bound in a substitution are well-typed. Hence, \lstinline{reify} always substitutes
some subterms in a well-typed term with other terms of the corresponding types, which preserves the well-typedness.

In addition to performing substitutions, \lstinline{reify} also reifies disequality constraints. Constraint reification
attaches to each free variable in a reified term a list of reified terms, describing the disequality constraints for that
free variable. Note, disequality can be established only for equally-typed terms, which justifies the type-safety of reification.
Note also, additional care has to be taken to avoid infinite looping, since reification of answers and constraints are
mutually recursive, and the reification of a variable can be potentially invoked from itself due to a chain of disequality
constraints. In the following example

\begin{lstlisting}
   let foo q =
      fresh (r s)
        (q === $\uparrow_{\forall}$ (Some r)) &&&
        (r =/= s) &&&
        (s =/= r)
\end{lstlisting}

\noindent the answer for the variable $q$ will contain a disequality constraint for the variable $r$; the reification of $r$ will in turn
lead to the reification of its own constraint, this time the variable $s$; finally, the reification of $s$ will again invoke the
reification of $r$, etc.

After the reification, the content of a logical value can be inspected via the following function:

\begin{lstlisting}
   val destruct : $[\alpha]$ -> [`Var of int * $[\alpha]$ list | `Value of $\alpha$]
\end{lstlisting}

Constructor \lstinline{`Var} corresponds to a free variable with unique integer identifier and a list of terms,
representing all disequality constraints for this variable.

\subsection{Tagless Logical Values and Type Bookkeeping}

The solution presented in the previous subsection suffers from the following deficiency: in order to perform unification,
we inject terms into the logic domain, making them as twice as large. As a result, this implementation loses to the original one in
terms of performance in many important applications, which compromises the very idea of using OCaml as a host language.

Here we develop an advanced version, which eliminates this penalty. As a first step, let's try to eliminate the tagging with
a drastic measure:

\begin{lstlisting}
   type $[\alpha]$ = $\alpha$
\end{lstlisting}

What consequences would this have? Of course, we would not be able to create logical variables in a conventional way. However,
we still could have a separate type of variables

\begin{lstlisting}
   type var = Var of int * anchor
\end{lstlisting}

\noindent and use \emph{the same} variable test procedure. As the type $[\alpha]$ is abstract, this modification does not change the interface.
As we reuse the variable test, polymorphic unification can continue to work \emph{almost} correctly. The problem is that
now it can introduce the occurrences of free logic variables in non-logical, tagless, data structures. These free logic variables
do not get in the way of unification itself (since it can handle them properly, thanks to the variable test), but they can not
be disclosed to the outer world as is.

Our idea is to use this generally unsound representation for all internal actions, and perform tagging only during the reification
stage. However, this scenario raises the following question: what would the type of \lstinline{reify} be? It can not be simply

\begin{lstlisting}
   val reify : state -> $[\alpha]$ -> $[\alpha]$
\end{lstlisting}

anymore since $[\alpha]$ now equals $\alpha$. We \emph{want}, however, it be something like

\begin{lstlisting}
   val reify : state -> $[\alpha]$ -> $(\mbox{``tagged'' } [\alpha])$
\end{lstlisting}

If $\alpha$ is not a parametric type, we can simply test if the value is a variable, and if yes, tag it with the constructor \lstinline{Var};
we tag it with \lstinline{Value} otherwise, and we're done. This trick, however, would not work for parametric types. Consider, for example,
the reification of a value of type \lstinline{$[[$int$]$ list$]$}. The (hypothetical) approach being described would return a value of
type \lstinline{$(\mbox{``tagged'' }[[$int$]$ list$])$}, i.e. tagged only on the top level; we need to repeat the procedure
recursively. In other words, we need the following (meta) type for the reification primitive:

\begin{lstlisting}
   val reify : state -> $[\alpha]$ -> $\mbox{(``tagged''} [\beta])$
\end{lstlisting}

\noindent where $\beta$ is the result of tagging $\alpha$.

These considerations can be boiled down to the following concrete implementation.

First, we roll back to the initial definition of $[\alpha]$~--- it will play the role of our ``tagged'' type.
We introduce a new, two-parameter type\footnote{In concrete syntax called ``$(\alpha,\;\beta)\;$\lstinline{injected}''.}

\begin{lstlisting}
   type $\{\alpha,\;\beta\}$ = $\alpha$
\end{lstlisting}

Of course, this type is kept abstract at the end-user level. Informally speaking, the type $\{\alpha,\;\beta\}$ designates the
injection of a tagless type $\alpha$ into a tagged type $\beta$; the value itself is kept in the tagless form, but
the tagged type can be used during the reify stage as a constraint, which would allow us to reify a tagless
representation only to a feasible tagged one. In other words, we record the injection steps using the second
type parameter of the type ``\{,\}'', performing the bookkeeping on the type level rather than on the value level.

We introduce the following primitives for the type $\{\alpha,\;\beta\}$:

\begin{lstlisting}
   val lift : $\alpha$ -> $\{\alpha,\;\alpha\}$
   val inj  : $\{\alpha,\;\beta\}$ -> $\{\alpha,\;[\beta]\}$

   let lift x = x
   let inj  x = x
\end{lstlisting}

The function \lstinline{lift} puts a value into the ``bookkeeping injection'' domain for the first time, while
\lstinline{inj} plays the role of the injection itself. Their composition is analogous to what was
called ``$\uparrow_\forall$'' in the previous implementation:

\begin{lstlisting}
   val $\uparrow_\forall$ : $\alpha$ -> $\{\alpha,\;[\alpha]\}$
   let $\uparrow_\forall$ x = inj (lift x)
\end{lstlisting}

In order to deal with parametric types, we can again utilize generic programming. To handle the types with
one parameter, we introduce the following functor:

\begin{lstlisting}
   module FMap (T : sig type $\alpha$ t val fmap : ($\alpha$ -> $\beta$) -> $\alpha$ t -> $\beta$ t end) :
     sig
       val distrib : $\{\alpha,\;\beta\}$ T.t -> $\{\alpha$ T.t, $\beta$ T.t$\}$
     end =
     struct
       let distrib x = x
     end
\end{lstlisting}

Note, that we do not use the function ``\lstinline{T.fmap}'' in the implementation; however, first, we need an inhabitant of the
corresponding type to make sure we are indeed dealing with a functor, and next, we actually will use it in the
implementation of type-specific reification, see below.

In order to handle two-, three-, etc. parameter types we need higher-kinded polymorphism, which is
not supported in a direct form in OCaml. So, unfortunately, we need to introduce separate
functors for the types with two-, three- etc. parameters; existing works on higher-kinded
polymorphism in OCaml~\cite{HKinded} require the similar scaffolding to be erected as a bootstrap step.

Given the functor(s) of the described shape, we can implement logic representations for
all type's constructors. For example, for standard type \lstinline{$\alpha$ option} with two constructors
\lstinline{None} and \lstinline{Some} the implementation looks like as follows:

\begin{lstlisting}
   module FOption = FMap (struct
     type $\alpha$ t = $\alpha$ option
     let fmap = fmap$_{\mbox{\texttt{option}}}$
   end)

   val some : $\{\alpha, \beta\}$ -> $\{\alpha\mbox{\texttt{ option}},\;\beta\mbox{\texttt{ option}}\}$
   val none : unit  -> $\{\alpha\mbox{\texttt{ option}},\;\beta\mbox{\texttt{ option}}\}$

   let some x  = inj (FOption.distrib (Some x))
   let none () = inj (FOption.distrib None)
\end{lstlisting}

In other words, we can in a very systematic manner define logic representations for all constructors
of types of interest. These representations can be used in the relational code, providing a well-bookkept
typing~--- for each logical type we would be able to reconstruct its original, tagless preimage.

With the new implementation, the types of basic goal constructors have to be adjusted:

\begin{lstlisting}
   val (===) : $\{\alpha,\;[\beta]\}$ -> $\{\alpha,\;[\beta]\}$ -> goal
   val (=/=) : $\{\alpha,\;[\beta]\}$ -> $\{\alpha,\;[\beta]\}$ -> goal
\end{lstlisting}

As always, we require both arguments of unification and disequality constraint to be of the same type; in addition
we require the injected part of the type to be logical.

During the reification stage the bindings for the top-level variables, reconstructed using the final
substitution, have to be properly tagged. This process is implemented in a datatype-generic manner as well:
first, we have reifiers for all primitive types:

\begin{lstlisting}
   val reify$_{\mbox{\texttt{int}}}$ : helper -> $\{$int,$[$int$]\}$ -> $[$int$]$
   val reify$_{\mbox{\texttt{string}}}$ : helper -> $\{$string,$[$string$]\}$ -> $[$string$]$
   ...
\end{lstlisting}

and, then, we add the reifier to the output signature in all \lstinline{FMap}-like functors:

\begin{lstlisting}
   val reify: (helper -> $\{\alpha,\;\beta\}$ -> $\beta$) -> helper -> $\{\alpha$ T.t, $[\beta$ T.t$]$ as $\gamma\}$ -> $\gamma$
\end{lstlisting}

Note, since now \lstinline{reify} is a type-specific and, hence, constructed at the user-level, we refrain from passing
it a state (which is inaccessible on the user level). Instead, we wrap all state-specific functionality in
an abstract \lstinline{helper} data type, which encapsulates all state-dependent functionality needed for \lstinline{reify}
to work properly.

\section{Reification and Top-Level Primitives}
\label{sec:reification}

In Section~\ref{sec:goals} we presented a top-level function \lstinline{run}, which
runs a goal and returns a stream of states. To acquire answers to the query,
represented by that goal, its free variables have to be reified in these states, and
we described the reification primitives in Section~\ref{sec:injection}. However,
the states keep answers in an untyped form, and the types of answers are
recovered solely on the basis of the types of variables being reified. So, the
type safety of the reification critically depends on the requirement to
reify each variable only in those states, which are descendants (w.r.t. the search tree)
of the state, in which that variable was created. In this section we describe a set of
top-level primitives, which enforce this requirement.

We provide a set of top-level combinators, which should be used to surround relational code
and perform reification in a transparent manner only in correct states.
We reimplement the top-level primitive \lstinline{run} to take three
arguments. The exact type of \lstinline{run} is rather complex and non-instructive,
so we prefer to describe the typical form of its application:

\begin{lstlisting}[mathescape=true]
   run $\overline{n}$ (fun $l_1\dots l_n$ -> $\;\;G$) (fun $a_1\dots a_n$ -> $\;\;H$)
\end{lstlisting}

Here $\overline{n}$ stands for a \emph{numeral}, which describes the number of
parameters for two other arguments of \lstinline{run}, \mbox{$l_1\dots l_n$}~---
free logical variables, $G$~--- a goal (which can make use of \mbox{$l_1\dots l_n$}),
\mbox{$a_1\dots a_n$}~--- reified answers for \mbox{$l_1\dots l_n$}, respectively, and,
finally, $H$~--- a \emph{handler} (which can make use of \mbox{$a_1\dots a_n$}).

The types of \mbox{$l_1\dots l_n$} are inferred from $G$ and always have a form

\begin{lstlisting}
   $\{\alpha,\;[\beta]\}$
\end{lstlisting}

\noindent since the types of variables can be constrained only in unification or disequality constraints.

The types of \mbox{$a_1\dots a_n$} are inferred from the types of \mbox{$l_1\dots l_n$} and
have the form

\begin{lstlisting}
   $(\alpha,\;\beta)$ reified stream
\end{lstlisting}

\noindent where the type \lstinline{reified}, in turn, is

\begin{lstlisting}
   type ($\alpha$, $\beta$) reified = $<\;$prj : $\alpha$; reify : (helper -> $\{\alpha,\;\beta\}$ -> $\beta$) -> $\beta>$
\end{lstlisting}

Two methods of this type can be used to perform two different styles of reification: first, a value without
free variables can be returned as is (using the method \lstinline{prj} which checks that in the value of
interest no free variables occur, and raises an exception otherwise). If the value contains some free
variables, it has to be properly injected into the logic domain~--- this is what \lstinline{reify} stands
for. It takes as an argument a type-specific tagging function, constructed using generic
primitives described in the previous section.

In other words a user-defined handler takes streams of reified answers for all variables supplied to the top-level
goal. All streams $a_i$ contain coherent elements, so they all have the same length and $n$-th elements of all
streams correspond to the $n$-th answer, produced by the goal $G$.

There are a few predefined numerals for one, two, etc. arguments (called, traditionally,
\lstinline{q}, \lstinline{qr}, \lstinline{qrs} etc.), and a successor function, which
can be applied to existing numeral to increment the number of expected arguments. The
implementation technique generally follows~\cite{Unparsing, DoWeNeed}.

Thus, the search and reification are tightly coupled; it is simply impossible to perform the reification
for arbitrarily-taken state and variable. This solution both guarantees the type safety and frees an end
user from the necessity to call reification primitives manually.

\section{Examples}
\label{sec:examples}

In this section we present some examples of a relational specification, written with the aid of our library.
Besides \miniKanren combinators themselves, our implementation contains two syntax extensions~--- one
for \lstinline{fresh} construct and another for \emph{inverse-$\eta$-delay}~\cite{MicroKanren}, which is
sometimes necessary to delay recursive calls in order to prevent infinite looping. In addition, we included a
small relational library of data structures like lists, numbers, booleans, etc. This library is written
completely on the user level using techniques described in Section~\ref{sec:injection} with no utilization
of any unsafe features. The examples given below illustrate the usage of all these elements as well.

\subsection{List Concatenation and Reversing}

List concatenation and reversing are usually the first relational programs considered, and we do not wish
to deviate from this tradition. We've already considered the implementation of \lstinline{append$^o$} in
original \miniKanren in Section~\ref{sec:demo}. In our case, the implementation looks familiar:

\begin{lstlisting}
   let rec append$^o$ x y xy =
     (x === nil ()) &&& (y === xy) |||
     (fresh (h t)
       (x === h % t)
       (fresh (ty)
         (h % ty === xy)
         (append$^o$ t y ty)
       )
     )

   let rec revers$^o$ a b =
     conde [
       (a === nil ()) &&& (b === nil ());
       (fresh (h t)
         (a === h % t)
         (fresh (a')
            (append$^o$ a' !< h b)
            (revers$^o$ t a')
         )
       )
     ]
\end{lstlisting}

Here we make use of our implementation of relational lists, which provides convenient shortcuts for
standard functional primitives:

\begin{itemize}
  \item ``\lstinline{nil ()}'' corresponds to ``\lstinline{[]}'';
  \item ``\lstinline{h % t}'' corresponds to ``\lstinline{h::t}'';
  \item ``\lstinline{a %< (b %< (c !< d))}'' corresponds to ``\lstinline{[a; b; c; d]}''.
\end{itemize}

In our implementation the basic \miniKanren primitive ``\lstinline{conde}'' is implemented as a
disjunction of a list of goals, not as a built-in syntax construct. We also make use of explicit
conjunction and disjunction infix operators instead of nested bracketed structures which, we
believe, would look too foreign here.

\subsection{Relational Sorting and Permutations}

For the next example we take list sorting; specifically, we present a sorting for lists of natural numbers
in Peano form since our library already contains built-in support for them. However, our example can be
easily extended for arbitrary (but linearly ordered) types.

List sorting can be implemented in \miniKanren in a variety of ways~--- virtually any existing algorithm can
be rewritten relationally. We, however, try to be as declarative as possible to demonstrate the
advantages of the relational approach. From this standpoint, we can claim that the sorted version of an empty list is an
empty list, and the sorted version of a non-empty list is its smallest element, concatenated with a sorted
version of the list containing all its remaining elements.

The following snippet literally implements this definition:

\begin{lstlisting}
   let rec sort$^o$ x y =
     conde [
       (x === nil ()) &&& (y === nil ());
       fresh (s xs xs')
         (y === s % xs')
         (sort$^o$ xs xs')
         (smallest$^o$ x s xs)
     ]
\end{lstlisting}

The meaning of the expression ``\lstinline{smallest$^o$ x s xs}'' is ``\lstinline{s} is the smallest element of a (non-empty) list \lstinline{x}, and \lstinline{xs} is the
list of all its remaining elements''. Now, \lstinline{smallest$^o$} can be implemented using a case analysis (note, ``\lstinline{l}'' here is a non-empty list):

\begin{lstlisting}
   let rec smallest$^o$ l s l' =
     conde [
       (l === s % nil ()) &&& (l' === nil ());
       fresh (h t s' t' max)
         (l' === max % t')
         (l === h % t)
         (minmax$^o$ h s' s max)
         (smallest$^o$ t s' t')
     ]
\end{lstlisting}

Finally, we implement a relational minimum-maximum calculation
primitive:

\begin{lstlisting}
   let minmax$^o$ a b min max =
     conde [
       (min === a) &&& (max === b) &&& (le$^o$ a b);
       (max === a) &&& (min === b) &&& (gt$^o$ a b)
     ]
\end{lstlisting}

Here ``\lstinline{le$^o$}'' and ``\lstinline{gt$^o$}'' are built-in comparison goals for natural numbers in Peano form.

Having relational \lstinline{sort$^o$}, we can implement sorting for regular integer lists:

\begin{lstlisting}
   let sort l =
     run q (sort$^o$ (inj_nat_list l))
           (fun qs -> from_nat_list ((Stream.hd qs)#prj) )
\end{lstlisting}

Here \lstinline{Stream.hd} is a function which takes a head from a lazy stream of answers,
\lstinline{inj_nat_list}~--- an injection from regular integer lists into logical lists of logical Peano numbers,
\lstinline{from_nat_list}~--- a projection from lists of Peano numbers to lists of integers.

It is interesting, that since \lstinline{sort$^o$} is
relational, it can be used to calculate a list of all \emph{permutations}
for a given list. Indeed, sorting each permutation results in the same list.
So, the problem of finding all permutations can be relationally reformulated into
the problem of finding all lists which are converted by sorting into the given one:

\begin{lstlisting}
let perm l = map (fun a -> from_nat_list a#prj)
  (run q (fun q -> fresh (r)
                     (sort$^o$ (inj_nat_list l) r)
                     (sort$^o$ q r)
         )
         (Stream.take ~n:(fact (length l))))
\end{lstlisting}

Note, for sorting the original list we used exactly the same primitive. Note also,
we requested exactly \lstinline{fact @@ length l} answers; requesting more
would result in an infinite search for non-existing answers.

\subsection{Type Inference for STLC}

Our final example is a type inference for Simply Typed Lambda Calculus~\cite{Lambda}. The problem and
solution themselves are rather textbook examples again~\cite{TRS, WillThesis}; however, with this example
we show once again the utilization of generic programming techniques we described in Section~\ref{sec:injection}.
As a supplementary generic programming library here we used object-oriented generic transformers\footnote{\url{https://github.com/dboulytchev/GT}};
we presume, however, that any other framework could equally be used.

We first describe the type of lambda terms and their logic representation:

\begin{lstlisting}
   module Term = struct
     module T = struct
       @type ('varname, 'self) t =
       | V   of 'varname
       | App of 'self    * 'self
       | Abs of 'varname * 'self
       with gmap

       let fmap f g x = gmap(t) f g x
     end

     include T
     include FMap2(T)

     let v   s   = inj (distrib (V s))
     let app x y = inj (distrib (App (x, y)))
     let abs x y = inj (distrib (Abs (x, y)))
   end
\end{lstlisting}

Now we have to repeat the work for the type of simple types:

\begin{lstlisting}
     module Type = struct
       module T = struct
         @type ('a, 'b) t =
         | P   of 'a
         | Arr of 'b * 'b
         with gmap

         let fmap f g x = gmap(t) f g x
       end

       include T
       include FMap2(T)

       let p   s   = inj (distrib (P s))
       let arr x y = inj (distrib (Arr (x, y)))
     end
\end{lstlisting}

Note, the ``relational'' part is trivial, boilerplate and short (and could even be generated
using a more advanced framework).

The relational type inferencer itself rather resembles the original implementation. The only
difference (besides the syntax) is that instead of data constructors we use their logic
counterparts:

\begin{lstlisting}
   let rec lookup$^o$ a g t =
     fresh (a' t' tl)
       (g === (inj_pair a' t') % tl)
       (conde [
         (a' === a) &&& (t' === t);
         (a' =/= a) &&& (lookup$^o$ a tl t)
       ])

   let infer$^o$ expr typ =
     let rec infer$^o$ gamma expr typ =
       conde [
         fresh (x)
           (expr === v x)
           (lookupo x gamma typ);
         fresh (m n t)
           (expr === app m n)
           (infer$^o$ gamma m (arr t typ))
           (infer$^o$ gamma n t);
         fresh (x l t t')
           (expr === abs x l)
           (typ  === arr t t')
           (infer$^o$ ((inj_pair x t) % gamma) l t')
       ]
     in
     infer$^o$ (nil()) expr typ
\end{lstlisting}

\section{Evaluation}
\label{sec:eva}

In this section we present an evaluation of the proposed approach. 
We have implemented several relational interpreters for different search problems which can be found in the repository mentioned before. 
Some of the simpler interpreters demonstrate good performance for different directions on their own and for them CPD transformation is not needed. 
Thus, we will focus on two search problems which are more complex: searching for a path in a graph and searching for a unifier~\cite{lozov:unification} of two terms. 
For each problem we compare four programs.
\begin{enumerate}
    \item The solver generated by the unnesting alone.
    \item The solver generated by the unnesting strategy aimed at backward execution. 
    \item The solver generated by the unnesting and then specialized by conjunctive partial deduction for the backward direction.
    \item The interpretation of the functional verifier with the relational interpreter implemented in Scheme~\cite{lozov:seven}. 
\end{enumerate}

First, let us compare the performance of the solvers for the path searching problem.
The implementation of the functional verifier for this problem is described in Section~\ref{sec:example}. 
We ran the search on a graph with 20 nodes and 30 edges, consequentially
 searching for paths of the length 5, 7, 9, 11, 13, and 15. 
We averaged the execution times over 10 runs of the same query. 
We the limited the execution time by 300 seconds, and if the execution time of some query exceeded the timeout, we put ``>300s'' in the result table and did not request the execution of queries for longer paths. The results are presented in Table~\ref{tab:isPath}. 

We can conclude that the execution time increases with the length of the path to search, which is expected, since with the length of the path the number of the subpaths to be explored is increasing as well.
The solver generated by the unnesting alone and the interpretation with the relational intepreter demonstrate poor performance. 
The first one is due to its inherently inefficient execution in backward direction, while the second suffers from the interpretation overhead. 
Both the unnesting aimed at the backward execution and the solver automatically transformed with conjunctive partial deduction show good performance. 
Conjunctive partial deduction performs more thorough specialization, thus producing more efficient program. 

\begin{table}
\centering
\begin{tabular}{c|c|c|c|c|c|c}
Path length                   & 5      & 7     & 9      & 11      & 13     & 15        \\
\hline\hline
Only conversion               & 0.01s  & 1.39s &  82.13s & >300s     & ---      & ---     \\
\hline
Backward oriented conversion  & 0.01s & 0.37s &  2.68s & 2.91s   & 4.88s    & 10.63s   \\
\hline
Conversion and CPD            & 0.01s  & 0.06s &  0.34s & 2.66s   & 3.65s    & 6.22s  \\
\hline
Scheme interpreter            & 0.80s  & 8.22s & 88.14s & 191.44s & >300s   & ---   \\
\end{tabular}

 \caption{Searching for paths in the graph}
    \label{tab:isPath}
\end{table}

Now let us consider the problem of finding a unifier of two terms which have free variables.
A term is either a variable ($X, Y, \dots$) or some constructor applied to terms ($nil, cons(H, T), \dots$). 
A substitution maps a variable to a term. 
A unifier of two terms $t$ and $s$ is a substitution $\sigma$ which equalizes them: $t \sigma = s \sigma$ by simultaneously substituting the variables for their images.
For example, a unifier of the terms $cons(42, X) \text{ and } cons(Y, cons(Y, nil)) \text{ is a substitution } \{X \mapsto cons(42, nil), Y \mapsto 42\} $.

We implemented a functional verifier which checks if a substitution equalizes two input terms. 
We represent a variable name as a unique Peano number. 
A substitution is represented as a list of terms, in which the index of the term is equal to the variable name to which the term is bound, so the substitution $\{X \mapsto cons(42, nil), Y \mapsto 42\}$ is represented as a list ``\lstinline{[cons(42, nil); 42]}''.
The verifier returns true if the input terms can be unified with the candidate substitution and false otherwise. 

As in the previous problem, we compare four solvers generated for the verifier described. 
With each solver, we search for a unifier of two terms and compare the execution times. 
The time comparison is presented in Table~\ref{tab:uni}. 
The first two rows of each column contain two terms being unified. 
We use uppercase letters from the end of the alphabet ($X, Y, \dots$) to denote variables, lowercase letters from the beginning of the alphabet ($a, b, \dots$) to denote constants (constructors with zero arguments), identifiers which start from the lowercase letter ($f, g,\dots$) to denote constructors.

Note, we compute a unifier for two terms, but not necessarily the most general unifier. 
We can implement the most general unification in \textsc{miniKanren}, but achieving the comparable performance using 
relational verifiers requires additional check that the unifier is indeed the most general. 
We are currently working on the implementation of such relational verifier. 

\begin{table}
\centering
\begin{tabular}{c|c|c|c}
\multirow{ 2}{*}{Terms} & 
f(X, a) & f(a \% b \% nil, c \% d \% nil, L) & f(X, X, g(Z, t))  \\
\cline{2-4} &
f(a, X) & f(X \% XS, YS, X \% ZS) & f(g(p, L), Y, Y)  \\
\hline\hline
Only conversion               & 0.01s  &  >300s & >300s \\
\hline
Backward oriented conversion  & 0.01s  &  0.11s & 2.26s  \\
\hline
Conversion and CPD            & 0.01s  &  0.07s & 0.90s  \\
\hline

Scheme interpreter            & 0.04s  & 5.15s & >300s    \\
\end{tabular}
 \caption{Searching for a unifier of two terms}
    \label{tab:uni}
\end{table}

Here four solvers compare to each other similarly to the previous problem: unnesting demonstrates the worst execution time, relational interpretation in Scheme is a little better, while unnesting aimed at backward execution and conjunctive partial deduction significantly improve the performance. 

There exist pairs of terms, for which either of the solvers fails to compute a unifier under 300 seconds. 
The example of such terms is ``\lstinline{f(A,B,C,A,B,C,D)}'' and ``\lstinline{f(B,C,D,x(R,S),x(a,T),x(Q,b),x(a,b))}''. 
This is caused by how general and declarative the verifier is: there is nothing in it to restrict the search space. 
We can modify the verifier with the additional check to ensure that there are no bound variables in the candidate unifier. 
This modification restricts the search space when there are many variables in the input terms.
But it also changes the semantics of the initial verifier and, as a consequence, the solvers: only idempotent unifiers can be found. 

To sum up, we demonstrated by two examples that it is possible to create problem solvers from verifiers by using relational conversion 
and conjunctive partial deduction. Currently conjunctive partial deduction improves the performance the most, as compared to 
interpreting verifiers with Scheme relational interpreter or doing relational conversion which is solely aimed at backward or 
forward execution.

\section{Conclusion and Future Work}

In this paper, we presented a certified formal semantics for core \textsc{miniKanren} and proved some of its basic properties
(including interleaving search completeness), which are believed to hold in existing implementations.
We also derived a semantics for conventional SLD resolution with cut and extracted two certified reference interpreters.
We consider our work as the initial setup for the future development of \textsc{miniKanren} semantics.

The language we considered here lacks many important features, which are already introduced
and employed in many implementations. Integrating these extensions~--- in the first hand, disequality constraints,~--- into
the semantics looks a natural direction for future work. We are also going to address the problems of proving some
properties of relational programs (equivalence, refutational completeness, etc.).


\nocite{*}
\bibliographystyle{eptcs}
\bibliography{ocanren}

\end{document}

\begin{figure*}[t]
\[
\begin{array}{cccll}
  &\mathcal{C} & = & \{C_i^{k_i}\} & \mbox{constructors with arities} \\
  &\mathcal{T}_X & = & X \cup \{C_i^{k_i} (t_1, \dots, t_{k_i}) \mid t_j\in\mathcal{T}_X\} & \mbox{terms over the set of variables $X$} \\
  &\mathcal{D} & = & \mathcal{T}_\emptyset & \mbox{ground terms}\\
  &\mathcal{X} & = & \{ x, y, z, \dots \} & \mbox{syntactic variables} \\
  &\mathcal{A} & = & \{ \alpha, \beta, \gamma, \dots \} & \mbox{semantic variables} \\
  &\mathcal{R} & = & \{ R_i^{k_i}\} &\mbox{relational symbols with arities} \\
  &\mathcal{G} & = & \mathcal{T_X}\equiv\mathcal{T_X}   &  \mbox{unification} \\
  &            &   & \mathcal{G}\wedge\mathcal{G}     & \mbox{conjunction} \\
  &            &   & \mathcal{G}\vee\mathcal{G}       &\mbox{disjunction} \\
  &            &   & \mbox{\lstinline|fresh|}\;\mathcal{X}\;.\;\mathcal{G} & \mbox{fresh variable introduction} \\
  &            &   & R_i^{k_i} (t_1,\dots,t_{k_i}),\;t_j\in\mathcal{T_X} & \mbox{relational symbol invocation} \\
  &\mathcal{S} & = & \{R_i^{k_i} = \lambda\;x_1^i\dots x_{k_i}^i\,.\, g_i;\}\; g & \mbox{specification}
\end{array}
\]
\caption{The syntax of the source language}
\label{syntax}
\end{figure*}

\begin{comment}
\begin{figure}[t]
%\centering
\[
\begin{array}{rcl}
  \mathcal{FV}\,(x)&=&\{x\}\\
  \mathcal{FV}\,(C_i^{k_i}\,(t_1,\dots,t_{k_i}))&=&\bigcup\mathcal{FV}\,(t_i)\\
  \mathcal{FV}\,(t_1\equiv t_2)&=&\mathcal{FV}\,(t_1)\cup\mathcal{FV}\,(t_2)\\
  \mathcal{FV}\,(g_1\wedge g_2)&=&\mathcal{FV}\,(g_1)\cup\mathcal{FV}\,(g_2)\\
  \mathcal{FV}\,(g_1\vee g_2)&=&\mathcal{FV}\,(g_1)\cup\mathcal{FV}\,(g_2)\\
  \mathcal{FV}\,(\mbox{\lstinline|fresh|}\;x\;.\;g)&=&\mathcal{FV}\,(g)\setminus\{x\}\\
  \mathcal{FV}\,(R_i^{k_i}\,(t_1,\dots,t_{k_i}))&=&\bigcup\mathcal{FV}\,(t_i)
\end{array}
\]
\caption{Free variables in terms and goals}
\label{free}
\end{figure}
\end{comment}

\section{The Language}
\label{language}
 
In this section, we introduce the syntax of the language we use throughout the paper, describe the informal semantics, and give some examples.

The syntax of the language is shown in Fig.~\ref{syntax}. First, we fix a set of constructors $\mathcal{C}$ with known arities and consider
a set of terms $\mathcal{T}_X$ with constructors as functional symbols and variables from $X$. We parameterize this set with an alphabet of
variables since in the semantic description we will need \emph{two} kinds of variables. The first kind, \emph{syntactic} variables, is denoted
by $\mathcal{X}$. The second kind, \emph{semantic} or \emph{logic} variables, is denoted by $\mathcal{A}$.
We also consider an alphabet of \emph{relational symbols} $\mathcal{R}$ which are used to name relational definitions.
The central syntactic category in the language is \emph{goal}. In our case, there are five types of goals: \emph{unification} of terms,
conjunction and disjunction of goals, fresh variable introduction, and invocation of some relational definition. Thus, unification is used
as a constraint, and multiple constraints can be combined using conjunction, disjunction, and recursion.
The final syntactic category is a \emph{specification} $\mathcal{S}$. It consists of a set
of relational definitions and a top-level goal. A top-level goal represents a search procedure which returns a stream of substitutions for
the free variables of the goal. The definition for a set of free variables for both terms and goals is conventional;
%given in Figure~\ref{free};
as ``\lstinline|fresh|''
is the sole binding construct the definition is rather trivial. The language we defined is first-order, as goals can not be passed as parameters,
returned or constructed at runtime.

We now informally describe how relational search works. As we said, a goal represents a search procedure. This procedure takes a \emph{state} as input and returns a
stream of states; a state (among other information) contains a substitution that maps semantic variables into the terms over semantic variables. Then five types of
scenarios are possible (depending on the type of the goal):

\begin{itemize}
\item Unification ``\lstinline|$t_1$ === $t_2$|'' unifies terms $t_1$ and $t_2$ in the context of the substitution in the current state. If terms are unifiable,
  then their MGU is integrated into the substitution, and a one-element stream is returned; otherwise the result is an empty stream.
\item Conjunction ``\lstinline|$g_1$ /\ $g_2$|'' applies $g_1$ to the current state and then applies $g_2$ to each element of the result, concatenating
  the streams.
\item Disjunction ``\lstinline|$g_1$ \/ $g_2$|'' applies both its goals to the current state independently and then concatenates the results.
\item Fresh construct ``\lstinline|fresh $x$ . $g$|'' allocates a new semantic variable $\alpha$, substitutes all free occurrences of $x$ in $g$ with $\alpha$, and
  runs the goal.
\item Invocation ``$\lstinline|$R_i^{k_i}$ ($t_1$,...,$t_{k_i}$)|$'' finds a definition for the relational symbol \mbox{$R_i^{k_i}=\lambda x_1\dots x_{k_i}\,.\,g_i$}, substitutes
  all free occurrences of a formal parameter $x_j$ in $g_i$ with term $t_j$ (for all $j$) and runs the goal in the current state.
\end{itemize}

We stipulate that the top-level goal is preceded by an implicit ``\lstinline|fresh|'' construct, which binds all its free variables, and that the final substitutions
for these variables constitute the result of the goal evaluation.

Conjunction and disjunction form a monadic~\cite{Monads} interface with conjunction playing role of ``\lstinline|bind|'' and disjunction~--- of ``\lstinline|mplus|''.
In this description, we swept a lot of important details under the carpet~--- for example, in actual implementations the components of disjunction are not evaluated in
isolation, but both disjuncts are being evaluated incrementally with the control passing from one disjunct to another (\emph{interleaving})~\cite{Search};
the evaluation of some goals can be additionally deferred (via so-called ``\emph{inverse-$\eta$-delay}'')~\cite{MicroKanren}; instead of streams
the implementation can be based on ``ferns''~\cite{BottomAvoiding} to defer divergent computations, etc. In the following sections, we present
a complete formal description of relational semantics which resolves these uncertainties in a conventional way.

As an example consider the following specification. For the sake of brevity we
abbreviate immediately nested ``\lstinline|fresh|'' constructs into the one, writing ``\lstinline|fresh $x$ $y$ $\dots$ . $g$|'' instead of
``\lstinline|fresh $x$ . fresh $y$ . $\dots$ $g$|''.

\begin{tabular}{p{5.5cm}p{5.5cm}}
\begin{lstlisting}
append$^o$ = fun x y xy .
 ((x === Nil) /\ (xy === y)) \/
 (fresh h t ty .
   (x  === Cons (h, t))  /\
   (xy === Cons (h, ty)) /\
   (append$^o$ t y ty));

revers$^o$ x x
\end{lstlisting} &
\begin{lstlisting}
revers$^o$ = fun x y .
 ((x === Nil) /\ (y === Nil)) \/
 (fresh h t tr .
   (x === Cons (h, t)) /\
   (append$^o$ tr (Cons (h, Nil)) y) /\
   (revers$^o$ t tr));
\end{lstlisting}
\end{tabular}

Here we defined\footnote{We respect here a conventional tradition for \textsc{miniKanren} programming to superscript all relational names with ``$^o$''.}
two relational symbols~--- ``\lstinline|append$^o$|'' and ``\lstinline|revers$^o$|'',~--- and specified a top-level goal ``\lstinline|revers$^o$ x x|''.
The symbol ``\lstinline|append$^o$|'' defines a relation of concatenation of lists~--- it takes three arguments and performs a case analysis on the first one. If the
first argument is an empty list (``\lstinline|Nil|''), then the second and the third arguments are unified. Otherwise, the first argument is deconstructed into a head ``\lstinline|h|''
and a tail ``\lstinline|t|'', and the tail is concatenated with the second argument using a recursive call to ``\lstinline|append$^o$|'' and additional variable ``\lstinline|ty|'', which
represents the concatenation of ``\lstinline|t|'' and ``\lstinline|y|''. Finally, we unify ``\lstinline|Cons (h, ty)|'' with ``\lstinline|xy|'' to form a final constraint. Similarly,
``\lstinline|revers$^o$|'' defines relational list reversing. The top-level goal represents a search procedure for all lists ``\lstinline|x|'', which are stable under reversing, i.e.
palindromes. Running it results in an infinite stream of substitutions:

\begin{lstlisting}
   $\alpha\;\mapsto\;$ Nil
   $\alpha\;\mapsto\;$ Cons ($\beta_0$, Nil)
   $\alpha\;\mapsto\;$ Cons ($\beta_0$, Cons ($\beta_0$, Nil))
   $\alpha\;\mapsto\;$ Cons ($\beta_0$, Cons ($\beta_1$, Cons ($\beta_0$, Nil)))
   $\dots$
\end{lstlisting}

where ``$\alpha$''~--- a \emph{semantic} variable, corresponding to ``\lstinline|x|'', ``$\beta_i$''~--- free semantics variables. Therefore, each substitution represents a set of all palindromes of a certain length.


\section{Relational Conversion}
\label{conversion}
\def\arraystretch{1}

Before we describe the relational conversion itself, we formulate some limitations for the source
programs. Functional programs tend to operate with higher-order values, while miniKanren is
limited by a first-order unification. Therefore, it would be unreasonable to expect, that arbitrary
functional program can be converted into a relational form (at least using reasonably simple 
transformations). 

We introduce the set of ground types $\mathcal G$:

$$
\mathcal G=\alpha \mid T^k(g_1,\dots,g_k)
$$

Informally, a value of a ground type cannot contain closures. Then we formulate the following limitations for
the programs to be converted into a relational form:

\begin{itemize}
  \item all constructor parameter types must be type variables;
  \item constructors and polymorphic equality can only be applied to the values of ground types;
  \item all \lstinline|match|-expressions must be of ground types.
\end{itemize}

The first condition means, that all algebraic datatypes (which we consider as defined implicitly, see Section~\ref{source_language}) 
have to be fully-polymorphic. The first two limitations then allow us to specify the polymorphism restriction for 
relational programs, which we mentioned informally in Section~\ref{ocanren}: all type variables are bounded to
range only over ground types (this condition, of course, is sufficient, but not necessary).

The third limitation is not essential and introduced only to simplify the presentation. If a \lstinline|match|-expression does not
have a ground type, it can always be transformed to have one by applying $\eta$-expansion:

\begin{lstlisting}
   match $e$ with {$p_i$ -> $e_i$} $\leadsto$ fun $\bar{x}$.match $e$ with {$p_i$ -> $e_i\,\bar{x}$}
\end{lstlisting}

\noindent where $\bar{x}$ is a vector of new variables, different from those in $e$, $e_i$, and $p_i$. In fact, our implementation,
described in Section~\ref{evaluation}, performs this expansion as long as a non-ground type \lstinline|match|-expression is encountered. 
This is the single case when we actually inspect types and perform $\eta$-expansion.

The general idea behind the conversion can be illustrated on a type level: an expression of type $t$ in the source
language is transformed into the expression of type $\sembr{t}^t$ in relational extension, where
the transformation $\sembr{\bullet}^t$ is defined as follows:

$$
\begin{array}{rcl}
\sembr{g}^t                     & = & g \to \G \\
\sembr{t_1 \to t_2}^t           & = & \sembr{t_1}^t \to \sembr{t_2}^t \\
%\sembr{\forall \alpha. \: t} & = & \forall \alpha. \: \sembr{t}
\end{array}
$$

In other words, an expression of a ground type is converted into a goal-returning function. The informal semantics
of this function is to make its argument respect a certain contract. As the argument can have some free variable occurrences, 
the goal tries to substitute these variables with some values in order to respect the contract this goal represents. 
For example, a constant \lstinline|Nil| is converted into a function \lstinline|fun $q$ . $q\,$=== ^Nil|.

The conversion itself is described in terms of transformation $\sembr{\bullet}^c$, see Fig.~7. %\ref{relational_conversion}. 
The first five rules
simply propagate the conversion through the expression; the last three actually do the work. These rules themselves may look complicated,
but the idea is rather simple.

\begin{figure}[t]
  \centering
  \begin{tabular}{rcp{6cm}}
     $\sembr{x}^c$                &=&$x$\\
     $\sembr{\lambda x.e}^c$      &=&$\lambda x.\sembr{e}^c$\\
     $\sembr{f\;e}^c$             &=&$\sembr{f}^c\;\sembr{e}^c$\\
     $\sembr{\lstinline|let $\;x\;$ = $\;e_1\;$ in $\;e_2$|}^c$&=&\lstinline|let $x$ = $\sembr{e_1}^c$ in $\sembr{e_2}^c$|\\
     $\sembr{\lstinline|let rec $\;f\;$ = $\lambda x.e_1\;$ in $\;e_2$|}^c$&=&\lstinline|let rec $f$ = $\sembr{\lambda x.e_1}^c$ in $\sembr{e_2}^c$|\\[2mm]
     $\sembr{C^k (e_1,\dots,e_k)}^c$&=&\lstinline|fun $q$.fresh ($q_1 \dots q_k$)|
\begin{lstlisting}
  ($\sembr{e_1}^c\; q_1$) /\
  ...
  ($\sembr{e_k}^c\; q_k$) /\
  ($q$ === $\;\uparrow(C^n (q_1, \dots, q_k)$))
\end{lstlisting}\\[-2mm]
     $\sembr{\lstinline|match $\;e\;$ with \{$C^{n_i}_i(x^i_1,\dots,x^i_{n_i})\;$ -> $\;e_i$\}|}^c$&=&\lstinline|fun $q$.fresh ($q_e$)|
\begin{lstlisting}
    ($\sembr{e}^c\;q_e$) /\
    $\bigvee_i$ ((fresh ($q^i_1\dots q^i_{n_i}$)
           ($q_e$ === $\;\uparrow C^{n_i}_i(q^i_1,\dots,q^i_{n_i})$) /\
           (fun $x^i_1\dots x^i_{n_i}$.$\sembr{e_i}^c$) ($\equiv q^i_1$) ... ($\equiv q^i_{n_i}$) $q$
     ) 
    )
\end{lstlisting}\\[-2mm]
     $\sembr{\lstinline|$e_1\,$=$\,e_2$|}^c$&=&\lstinline|fun $q$.fresh ($q_1\,q_2$)|
\begin{lstlisting}
  $\sembr{e_1}^c\,q_1$ /\
  $\sembr{e_2}^c\,q_2$ /\
  (($q_1$ === $\;q_2$ /\ $q$ === $\;$^true) |||
   ($q_1$ =/= $\;q_2$ /\ $q$ === $\;$^false)
  )
\end{lstlisting}
  \end{tabular}
\label{relational_conversion}
\caption{Relational conversion}
\end{figure}

In the case of constructor we know, that all expressions $e_i$ have ground types. Thus, their relational images are goal-returning
functions. We create a set of fresh variables (one for each expression) and pass them as arguments to these functions to associate
them with the values of the expressions. The result of conversion for the constructor application itself has to be a 
goal-returning function as well. We surround expression constructed so far with abstraction and unify its argument $q$ with the
constructor, applied to corresponding logical variables. We also apply logical constructor $\uparrow$ to respect the typing rule
for unification.

The rule for pattern-matching conversion operates similarly. First, the scrutinee must have a ground type (since it is matched against
constructors). We create a fresh variable $q_e$ and associate it with the value of the scrutinee exactly as in the previous
case. Then, for each branch we create a number of fresh variables (one for each variable in the pattern for the branch) and
express pattern-matching in terms of unification, using these variables and corresponding constructor. Finally, the body $e_i$ of the branch
is an expression with free variables, corresponding to those in the pattern. We, therefore, convert $e_i$ and surround the result with
lambdas, closing all these variables. To pass the bindings $q^i_j$ for pattern variables to the body, we apply this function to
 goal-returning functions $(\equiv q^i_j)$. This, again, gives us a goal-returning function, which we apply to the topmost result variable $q$.

The last rule follows the same pattern: both arguments of polymorphic equality are transformed into goal-returning functions, and we know, that
the arguments of these functions are of some ground type. We apply these functions to fresh variables and perform case analysis. Note, this is
the only case when we actually use disequality constraints.

An interesting property of relational conversion is that it does not change terms, which do not use constructors, equality, and pattern-matching. Thus,
a lot of useful higher-order functions~--- application, composition, fixed point, etc.~--- are already relational and can be used in
relational specifications.

Another observation is that our transformation is compositional (a relational image of application is an application of relational
images). This means, that relational conversion is compatible with separate compilation~--- multiple source files can be
converted independently without losing the possibility to work properly when combined.

Then, it is interesting, that the result of relational conversion runs in a forward direction
deterministically. Thus, relational conversion imposes only a constant-time slowdown in a forward
direction.

Finally, we formulate the following properties for relational conversion:

\begin{itemize}
\item Static correctness: if an expression $e$ has a type $t$ in the source language, then $\sembr{e}^c$ has a 
type $\sembr{t}^t$ in relational extension. In other words, relational conversion transforms properly typed
programs into properly typed. Proof is by structural induction (and trivial).
\item Partial semantic correctness: if an expression $e$ has a ground type $t$ and \mbox{$e \leadsto^f v$} for some
  value $v$, then \mbox{$\lstinline|fresh($x$)($\sembr{e}^c\;x$)| \leadsto^r (\theta,\emptyset)$}, and 
\mbox{$\theta(\mathfrak{s})=v$}, where $\mathfrak{s}$ is a semantic variable, associated with $x$ on the
first step of the relational evaluation.
%The essential part of the proof is given in the Appendix~\ref{appendix}.
%Proof
%is by induction on the length of derivation sequence (a number of lemmas have to be justified on the way).
\end{itemize}

In order to prove the complete correctness, we need some means to interpret the results of relational 
derivation with free variables in functional case. This is a subject of future research.

\section{Evaluation}
\label{sec:eva}

In this section we present an evaluation of the proposed approach. 
We have implemented several relational interpreters for different search problems which can be found in the repository mentioned before. 
Some of the simpler interpreters demonstrate good performance for different directions on their own and for them CPD transformation is not needed. 
Thus, we will focus on two search problems which are more complex: searching for a path in a graph and searching for a unifier~\cite{lozov:unification} of two terms. 
For each problem we compare four programs.
\begin{enumerate}
    \item The solver generated by the unnesting alone.
    \item The solver generated by the unnesting strategy aimed at backward execution. 
    \item The solver generated by the unnesting and then specialized by conjunctive partial deduction for the backward direction.
    \item The interpretation of the functional verifier with the relational interpreter implemented in Scheme~\cite{lozov:seven}. 
\end{enumerate}

First, let us compare the performance of the solvers for the path searching problem.
The implementation of the functional verifier for this problem is described in Section~\ref{sec:example}. 
We ran the search on a graph with 20 nodes and 30 edges, consequentially
 searching for paths of the length 5, 7, 9, 11, 13, and 15. 
We averaged the execution times over 10 runs of the same query. 
We the limited the execution time by 300 seconds, and if the execution time of some query exceeded the timeout, we put ``>300s'' in the result table and did not request the execution of queries for longer paths. The results are presented in Table~\ref{tab:isPath}. 

We can conclude that the execution time increases with the length of the path to search, which is expected, since with the length of the path the number of the subpaths to be explored is increasing as well.
The solver generated by the unnesting alone and the interpretation with the relational intepreter demonstrate poor performance. 
The first one is due to its inherently inefficient execution in backward direction, while the second suffers from the interpretation overhead. 
Both the unnesting aimed at the backward execution and the solver automatically transformed with conjunctive partial deduction show good performance. 
Conjunctive partial deduction performs more thorough specialization, thus producing more efficient program. 

\begin{table}
\centering
\begin{tabular}{c|c|c|c|c|c|c}
Path length                   & 5      & 7     & 9      & 11      & 13     & 15        \\
\hline\hline
Only conversion               & 0.01s  & 1.39s &  82.13s & >300s     & ---      & ---     \\
\hline
Backward oriented conversion  & 0.01s & 0.37s &  2.68s & 2.91s   & 4.88s    & 10.63s   \\
\hline
Conversion and CPD            & 0.01s  & 0.06s &  0.34s & 2.66s   & 3.65s    & 6.22s  \\
\hline
Scheme interpreter            & 0.80s  & 8.22s & 88.14s & 191.44s & >300s   & ---   \\
\end{tabular}

 \caption{Searching for paths in the graph}
    \label{tab:isPath}
\end{table}

Now let us consider the problem of finding a unifier of two terms which have free variables.
A term is either a variable ($X, Y, \dots$) or some constructor applied to terms ($nil, cons(H, T), \dots$). 
A substitution maps a variable to a term. 
A unifier of two terms $t$ and $s$ is a substitution $\sigma$ which equalizes them: $t \sigma = s \sigma$ by simultaneously substituting the variables for their images.
For example, a unifier of the terms $cons(42, X) \text{ and } cons(Y, cons(Y, nil)) \text{ is a substitution } \{X \mapsto cons(42, nil), Y \mapsto 42\} $.

We implemented a functional verifier which checks if a substitution equalizes two input terms. 
We represent a variable name as a unique Peano number. 
A substitution is represented as a list of terms, in which the index of the term is equal to the variable name to which the term is bound, so the substitution $\{X \mapsto cons(42, nil), Y \mapsto 42\}$ is represented as a list ``\lstinline{[cons(42, nil); 42]}''.
The verifier returns true if the input terms can be unified with the candidate substitution and false otherwise. 

As in the previous problem, we compare four solvers generated for the verifier described. 
With each solver, we search for a unifier of two terms and compare the execution times. 
The time comparison is presented in Table~\ref{tab:uni}. 
The first two rows of each column contain two terms being unified. 
We use uppercase letters from the end of the alphabet ($X, Y, \dots$) to denote variables, lowercase letters from the beginning of the alphabet ($a, b, \dots$) to denote constants (constructors with zero arguments), identifiers which start from the lowercase letter ($f, g,\dots$) to denote constructors.

Note, we compute a unifier for two terms, but not necessarily the most general unifier. 
We can implement the most general unification in \textsc{miniKanren}, but achieving the comparable performance using 
relational verifiers requires additional check that the unifier is indeed the most general. 
We are currently working on the implementation of such relational verifier. 

\begin{table}
\centering
\begin{tabular}{c|c|c|c}
\multirow{ 2}{*}{Terms} & 
f(X, a) & f(a \% b \% nil, c \% d \% nil, L) & f(X, X, g(Z, t))  \\
\cline{2-4} &
f(a, X) & f(X \% XS, YS, X \% ZS) & f(g(p, L), Y, Y)  \\
\hline\hline
Only conversion               & 0.01s  &  >300s & >300s \\
\hline
Backward oriented conversion  & 0.01s  &  0.11s & 2.26s  \\
\hline
Conversion and CPD            & 0.01s  &  0.07s & 0.90s  \\
\hline

Scheme interpreter            & 0.04s  & 5.15s & >300s    \\
\end{tabular}
 \caption{Searching for a unifier of two terms}
    \label{tab:uni}
\end{table}

Here four solvers compare to each other similarly to the previous problem: unnesting demonstrates the worst execution time, relational interpretation in Scheme is a little better, while unnesting aimed at backward execution and conjunctive partial deduction significantly improve the performance. 

There exist pairs of terms, for which either of the solvers fails to compute a unifier under 300 seconds. 
The example of such terms is ``\lstinline{f(A,B,C,A,B,C,D)}'' and ``\lstinline{f(B,C,D,x(R,S),x(a,T),x(Q,b),x(a,b))}''. 
This is caused by how general and declarative the verifier is: there is nothing in it to restrict the search space. 
We can modify the verifier with the additional check to ensure that there are no bound variables in the candidate unifier. 
This modification restricts the search space when there are many variables in the input terms.
But it also changes the semantics of the initial verifier and, as a consequence, the solvers: only idempotent unifiers can be found. 

To sum up, we demonstrated by two examples that it is possible to create problem solvers from verifiers by using relational conversion 
and conjunctive partial deduction. Currently conjunctive partial deduction improves the performance the most, as compared to 
interpreting verifiers with Scheme relational interpreter or doing relational conversion which is solely aimed at backward or 
forward execution.

\section{Conclusion and Future Work}

In this paper, we presented a certified formal semantics for core \textsc{miniKanren} and proved some of its basic properties
(including interleaving search completeness), which are believed to hold in existing implementations.
We also derived a semantics for conventional SLD resolution with cut and extracted two certified reference interpreters.
We consider our work as the initial setup for the future development of \textsc{miniKanren} semantics.

The language we considered here lacks many important features, which are already introduced
and employed in many implementations. Integrating these extensions~--- in the first hand, disequality constraints,~--- into
the semantics looks a natural direction for future work. We are also going to address the problems of proving some
properties of relational programs (equivalence, refutational completeness, etc.).


%\begin{comment}
\begin{thebibliography}{10}

\bibitem{CKanren}
C.~E. Alvis, J.~J. Willcock, K.~M. Carter, W.~E. Byrd, and D.~P. Friedman.
\newblock {cKanren}: {miniKanren} with Constraints.
\newblock In {\em Proceedings of the 2011 Annual Workshop on Scheme and
  Functional Programming}, Oct. 2011.

\bibitem{Lambda}
H.~P. Barendregt.
\newblock Lambda Calculi with Types.
\newblock In {\em Handbook of Logic in Computer Science (vol. 2)}, 
pages 117--309. Oxford University Press, Inc., New York, NY, USA, 1992.

\bibitem{WillOnHM}
W.~E. Byrd.
\newblock Private communications.

\bibitem{WillThesis}
W.~E. Byrd.
\newblock Relational Programming in miniKanren: Techniques, Applications,
  and Implementations.
\newblock PhD thesis, Indiana University, September 2009.

\bibitem{unified}
W.~E. Byrd, M.~Ballantyne, G.~Rosenblatt, and M.~Might.
\newblock A Unified Approach to Solving Seven Programming Problems (functional
  pearl).
\newblock {\em Proc. ACM Program. Lang.}, 1(ICFP):8:1--8:26, Aug. 2017.

\bibitem{alphaKanren}
W.~E. Byrd and D.~P. Friedman.
\newblock {$\alpha$Kanren}: A Fresh Name in Nominal Logic Programming.
\newblock In {\em Proceedings of the 2007 Annual Workshop on Scheme and
  Functional Programming}, pages 79--90, 2007.

\bibitem{Untagged}
W.~E. Byrd, E.~Holk, and D.~P. Friedman.
\newblock miniKanren, Live and Untagged: Quine Generation via Relational
  Interpreters (programming pearl).
\newblock In {\em Proceedings of the 2012 Annual Workshop on Scheme and
  Functional Programming}, Scheme '12, pages 8--29, New York, NY, USA, 2012.
  ACM.

\bibitem{cardelli}
L.~Cardelli and P.~Wegner.
\newblock On Understanding Types, Data Abstraction, and Polymorphism.
\newblock {\em ACM Comput. Surv.}, 17(4):471--523, Dec. 1985.

\bibitem{TRS}
D.~P. Friedman, W.~E. Byrd, and O.~Kiselyov.
\newblock The Reasoned Schemer.
\newblock The MIT Press, 2005.

\bibitem{MicroKanren}
J.~Hemann and D.~P. Friedman.
\newblock $\mu$Kanren: A Minimal Functional Core for Relational Programming.
\newblock In {\em Proceedings of the 2013 Annual Workshop on Scheme and
  Functional Programming}, 2013.

\bibitem{SmallEmbedding}
J.~Hemann, D.~P. Friedman, W.~E. Byrd, and M.~Might.
\newblock A Small Embedding of Logic Programming with a Simple Complete Search.
\newblock {\em SIGPLAN Not.}, 52(2):96--107, Nov. 2016.

\bibitem{ocanren}
D.~Kosarev and D.~Boulytchev.
\newblock Typed Embedding of a Relational Language in OCaml.
\newblock {\em ACM SIGPLAN Workshop on ML}, 2016.

\bibitem{UnificationRevisited}
J.-L. Lassez, M.~J. Maher, and K.~Marriott.
\newblock Unification Revisited.
\newblock In {\em Foundations of Deductive Databases and Logic Programming},
pages 587--625. Morgan Kaufmann Publishers Inc., San Francisco, CA, USA, 1988.

\bibitem{Types}
B.~C. Pierce.
\newblock Types and Programming Languages.
\newblock The MIT Press, 1st edition, 2002.

\bibitem{Unification}
F.~Baader and W.~Snyder. 
\newblock{Unification Theory.}
\newblock In {\em Handbook of Automated Reasoning},
Elsevier Science Publishers B. V., Amsterdam, The Netherlands, The Netherlands, 2001.

\bibitem{Felleisen}
A.~Wright and M.~Felleisen.
\newblock A Syntactic Approach to Type Soundness.
\newblock {\em Inf. Comput.}, 115(1):38--94, Nov. 1994.

\end{thebibliography}
%\end{comment}

%\renewcommand{\clearpage}{} 
%\bibliographystyle{abbrv}
%\bibliography{main}

%\clearpage
%\appendix
%\section{Appendix}
\label{appendix}

In this appendix we present a proof of partial semantic correctness of relational conversion, or, to be precise, 
a number of observations, definitions, and claims, which, we believe, are sufficient to reconstruct
the complete proof. 

We remind, that our goal is to prove the following statement:

\begin{theorem} 
\normalfont For arbitrary functional program $p$ of a ground type $t$, arbitrary value $v$, and
arbitrary variable $x$

$$
\begin{array}{c}
p\leadsto^f v \Rightarrow \lstinline|fresh ($x$) ($\sembr{p}^c x$)| \leadsto^r (\theta, \emptyset)\\
\mbox{and}\\
\theta(\mathfrak{s})=v
\end{array}
$$

\noindent where $\mathfrak{s}$ is a semantic variable, associated with
$x$ on the first step of the relational evaluation.
\end{theorem}
  
We first comment on the empty set as the set of negative substitutions. A disequality constraint can
come only from a polymorphic equality, which is applied when both its operands are reduced to
values. In the relational counterpart, being run in a forward direction, this corresponds to the evaluation of disequality constraints for
closed terms only, which, in turn, means, that they will immediately succeed or fail. Both cases
add nothing to the set of negative substitutions, which is initially empty. 

Next, we cannot prove the theorem, using an induction by a derivation length, since in the case of
application, for example, the type of the term in the head position is not ground. This 
obstacle could be lifted, if we could prove the following generalization:

$$
p\leadsto^f f \Rightarrow \sembr{p}^c\leadsto^r\sembr{f}^c
$$ 

\noindent for arbitrary $p$ of any type. This claim, however, turned out to be false~--- a term
\lstinline|C ((fun x.x) A)| can be taken as an example.  

The origin of the problem is that we \emph{functionalize} the constructors, \lstinline|match|, and
equality expressions, and, hence, change the order of reductions in the relational counterpart in 
comparison with the original functional program. Thus, we need to take this change into account.

First, we develop a modified functional semantics, which corresponds better to the reduction
order in the relational case. We call this semantics \emph{deferred}, as it defers the evaluation
of constructors, \lstinline|match|, and equality expressions. This semantics can be acquired in
two steps: first, we consider a reduced version of the original functional semantics, in which
we treat arbitrary constructor, \lstinline|match|, and equality expressions as values. Then, the
deferred semantics is just an iterative application of the reduced version to the arguments 
of these new values (arguments of constructors or equality operator, or scrutinees of \lstinline|match| 
expressions).

Next, we claim, that if a term of some ground type is reduced to some value by the original semantics,
then it as well is reduced to the same value by the deferred one. This claim is based on the following
observations:

\begin{itemize}
\item progress and type preservation properties for both semantics (which can be proven in a standard
way);
\item Church-Rosser property for lambda-calculus;
\item the fact, that the reduced semantics applies a proper subset of rules of the original one.
\end{itemize}

Now, we are going to prove the theorem by a simulation between the deferred semantics for the original program
and the relational one for the relationally converted. Before that, we formulate the number of lemmas and 
definitions.

\begin{lemma}
\label{stack_split}
\normalfont Let us separate all the contexts into two disjoint kinds: 

\begin{itemize}
\item functional

$$
C_f = \Box\;e\mid v\;\Box\mid\lstinline|let $x$ = $\Box$ in $e$|
$$

\item ground

$$
C_g = \lstinline|match $\;\Box\;$ with $\{p_i$->$e_i\}$|\mid C^n(\bar{v},\Box,\bar{e})\mid\Box\lstinline|=e|\mid\lstinline|v=|\Box
$$

Let $\left<{\mathcal S},\,e\right>$ be an arbitrary state in a derivation sequence w.r.t. the deferred
semantics. Then $\mathcal S=C_f^*C_g^*$.
\end{itemize}

In other words, during the evaluation w.r.t. the deferred semantics, the stack of contexts is separated into the two
(possibly empty) segments: all ground contexts reside below all functional. The proof is by the induction on the
length of derivation sequence.
\end{lemma}

\begin{definition}
\normalfont
We as well separate all terms of the source language into the two disjoint kinds:

\begin{itemize}
\item functional

$$
e_1\,e_2\mid \lambda x.e \mid \mu f.\lambda x.e \mid \lstinline|let $x$ = $e_1$ in $e_2$| \mid \lstinline|let rec $f$ = $\lambda x.e_1$ in $e_2$|
$$

\item ground

$$
e_1 = e_2 \mid \lstinline|match $e$ with {$p_i$ -> $e_i$ }| \mid \lstinline|C$^k$ ($e_1\dots e_k$)|
$$

\end{itemize}

\end{definition}

\begin{definition}
\normalfont Augmented conversion of a term w.r.t. to a substitution $\sembr{\bullet}_\theta$ is defined as follows: 

$$
\begin{array}{rcl}
\sembr{p}_\theta&=&\sembr{p}^c\\
\sembr{v}_\theta&=&(\lambda x.x\equiv\mathfrak{s}),\,\mbox{if}\;\;\theta(\mathfrak s)=v
\end{array}
$$

Here $\theta$ is a substitution, $p$~--- arbitrary functional term, $v$~--- arbitrary value of a
ground type in the sense of the original semantics (i.e. the composition of constructors). Note, the
cases in this definition are not disjoint, and in the second case there can be more, than one
variable with the requested property, so augmented conversion defines a set of relational terms.
\end{definition}

\begin{lemma}
\label{substitution}
\normalfont Let $f$, $e$ be two arbitrary terms of the source language, $\theta$~--- arbitrary
substitution. Then

$$
\sembr{f[x\gets e]}_\theta=\sembr{f}_\theta[x\gets\sembr{e}_\theta]
$$

The equality here is understood in a set-theoretic sense. The proof is by structural 
induction.
\end{lemma}

\begin{definition}
\normalfont For arbitrary substitution $\theta$ define a conversion of a functional context  
$\sembr{\bullet}_\theta$ as follows:

$$
\begin{array}{rcl}
\sembr{\Box\,e}_\theta&=&\Box\,\sembr{e}_\theta\\
\sembr{v\,\Box}_\theta&=&\sembr{v}_\theta\,\Box\\
\sembr{\lstinline|let $\;x\; = \;\Box\;$ in $\;e$|}_\theta&=&\lstinline|let $\;x\; = \;\Box\;$ in $\;\sembr{e}_\theta$|
\end{array}
$$

Here $e$ is an arbitrary functional term, $v$~--- abstraction. This conversion is an extension of augmented
conversion for functional contexts, hence the same denotation.
\end{definition}

\begin{definition}
\normalfont For arbitrary semantic variables ${\mathfrak s}_1$, ${\mathfrak s}_2$ and arbitrary substitution $\theta$ 
define a conversion of ground context $\sembr{\bullet}^{{\mathfrak s}_1{\mathfrak s}_2}_\theta$ as follows:

$$ 
\begin{array}{rcl}
\sembr{C^k(v_1, \ldots, v_{i-1}, \Box, e_{i+1}, \ldots, e_k)}^{{\mathfrak s}_1{\mathfrak s}_2}_\theta&=&\Box \; \wedge \\
       & & (\sembr{e_{i+1}}_\theta \; {\mathfrak s}^\prime_{i+1}) \; \wedge \\
       & & \ldots  \\
       & & (\sembr{e_k}_\theta \; {\mathfrak s}^\prime_k) \; \wedge \\
       & & ({\mathfrak s}_2 \equiv\; \uparrow C^k({\mathfrak s}^\prime_1, \ldots, {\mathfrak s}^\prime_{i-1}, {\mathfrak s}_1, {\mathfrak s}^\prime_{i+1}, \ldots, {\mathfrak s}_k)),\,\mbox{if}\;\theta({\mathfrak s}^\prime_j)=v_j,\,j<i
\end{array}
$$

$$
\begin{array}{rcl}
\sembr{\Box = e}^{{\mathfrak s}_1{\mathfrak s}_2}_\theta&=&\Box\, \wedge \\
 & & (\sembr{e}_\theta\; {\mathfrak s}^\prime) \wedge \\
 & & ((({\mathfrak s}_1 \equiv {\mathfrak s}^\prime) \wedge ({\mathfrak s}_2 \equiv \lstinline|^true|))\, \vee \\ 
 & & (({\mathfrak s}_1 \not \equiv {\mathfrak s}^\prime) \wedge ({\mathfrak s}_2 \equiv \lstinline|^false|))) 
\end{array}
$$

$$
\begin{array}{rcl}
\sembr{v = \Box}^{{\mathfrak s}_1{\mathfrak s}_2}_\theta&=&\Box\,\wedge \\
 & & ((({\mathfrak s}^\prime \equiv {\mathfrak s}_1) \wedge ({\mathfrak s}_2 \equiv \lstinline|^true|))\, \vee \\ 
 & & (({\mathfrak s}^\prime \not \equiv {\mathfrak s}_1) \wedge ({\mathfrak s}_2 \equiv \lstinline|^false|))),\,\mbox{if}\;\theta({\mathfrak s})=v 
\end{array}
$$

$$
\begin{array}{rcl}
\sembr{\lstinline|match $\;\Box\;$ with \{$C^{n_i}_i$($y^i_1$, ..., $y^i_{n_i}$) -> $\;e_i$\}|}^{{\mathfrak s}_1{\mathfrak s}_2}_\theta&=&\Box \; \wedge \bigvee_i\\
& &(\lstinline|fresh ($s^i_1 \ldots s^i_{n_i}$)| \\
& &\qquad({\mathfrak s}_1 \equiv \;\uparrow C_i^{n_i}(s^i_1, \ldots, s^i_{n_i})) \\
& &\qquad(\lambda y^i_1. \ldots \lambda  y^i_{n_i}. \sembr{e_i}_\theta) \; (\equiv s^i_1) \ldots (\equiv s^i_{n_i})\;{\mathfrak s}_2)
\end{array}
$$

Here we assume ${\mathfrak s}^\prime$ and ${\mathfrak s}^\prime_i$ to be arbitrary semantic variables, $v_i$~--- arbitrary values w.r.t. the original 
functional semantics, $e_i$~--- arbitrary terms of the source language. We also claim, that $\theta$ is
undefined for all mentioned semantic variables, unless the opposite is specified explicitly.

\end{definition}

\begin{definition}
\normalfont For arbitrary substitution $\theta$, arbitrary semantic variable ${\mathfrak s}_m$ and a functional 
term $e$ define a conversion of a stack $\sembr{\bullet}^{e,{\mathfrak s}_m}_\theta$ as follows:

$$
\def\arraystretch{1.5}
\sembr{f_n\dots f_1g_m\dots g_1}^{e,{\mathfrak s}_m}_\theta=\left\{
\begin{array}{lcl}
\sembr{g_m}^{{\mathfrak s}_m{\mathfrak s}_{m-1}}_\theta\dots\sembr{g_1}^{{\mathfrak s}_1{\mathfrak s}_0}_\theta&,&n=0\;\;\mbox{and $e$~--- ground}\\
\sembr{f_n}_\theta\dots\sembr{f_1}_\theta(\Box\,{\mathfrak s}_m)\sembr{g_m}^{{\mathfrak s}_m{\mathfrak s}_{m-1}}_\theta\dots\sembr{g_1}^{{\mathfrak s}_1{\mathfrak s}_0}_\theta&,&\mbox{otherwise}
\end{array}
\right.
$$

Here ${\mathfrak s}_0\dots {\mathfrak s}_{m-1}$ designate arbitrary distinct semantic variables.
\end{definition}

\begin{definition}
\normalfont For arbitrary substitution $\theta$ and arbitrary semantic variable ${\mathfrak s}_m$ define a simulation
conversion $\sembr{\bullet}^{{\mathfrak s}_m}_\theta$ of the source language term as follows:

$$
\begin{array}{rcl}
\sembr{e_1 = e_2}^{{\mathfrak s}_m}_\theta&=& (\sembr{e_1}_\theta\; {\mathfrak s}^\prime_1) \wedge \\
                           & & (\sembr{e_2}_\theta\; {\mathfrak s}^\prime_2) \wedge \\
                           & & ((({\mathfrak s}^\prime_1 \equiv {\mathfrak s}^\prime_2) \wedge ({\mathfrak s}_m \equiv \lstinline|^true|))\, \vee \\ 
                           & & (({\mathfrak s}^\prime_1 \not \equiv {\mathfrak s}^\prime_2) \wedge ({\mathfrak s}_m \equiv \lstinline|^false|)))
\end{array}
$$

$$
\begin{array}{rcl}
\sembr{v = e}^{{\mathfrak s}_m}_\theta&=& (\sembr{e}_\theta\; {\mathfrak s}^\prime_2) \wedge \\
                        & & ((({\mathfrak s}^\prime_1 \equiv {\mathfrak s}^\prime_2) \wedge ({\mathfrak s}_m \equiv \lstinline|^true|))\, \vee \\ 
                        & & (({\mathfrak s}^\prime_1 \not \equiv {\mathfrak s}^\prime_2) \wedge ({\mathfrak s}_m \equiv \lstinline|^false|))),\,\mbox{if}\;\theta({\mathfrak s}^\prime_1)=v
\end{array}
$$

$$
\begin{array}{rcl}
\sembr{v_1 = v_2}^{{\mathfrak s}_m}_\theta&=& ((({\mathfrak s}^\prime_1 \equiv {\mathfrak s}^\prime_2) \wedge ({\mathfrak s}_m \equiv \lstinline|^true|))\, \vee \\ 
                           & & (({\mathfrak s}^\prime_1 \not \equiv {\mathfrak s}^\prime_2) \wedge ({\mathfrak s}_m \equiv \lstinline|^false|))),\,\mbox{if}\;\theta({\mathfrak s}^\prime_j)=v_j
\end{array}
$$

$$ 
\begin{array}{rcl}
\sembr{C^k(v_1, \ldots, v_{i-1}, e_i, \ldots, e_k)}^{{\mathfrak s}_m}_\theta&=&(\sembr{e_i}_\theta \; {\mathfrak s}^\prime_i) \; \wedge \\
       & & \ldots  \\
       & & (\sembr{e_k}_\theta \; {\mathfrak s}^\prime_k) \; \wedge \\
       & & ({\mathfrak s}_m \equiv\; \uparrow C^k({\mathfrak s}^\prime_1, \ldots, {\mathfrak s}^\prime_k)),\,\mbox{if}\;\theta({\mathfrak s}^\prime_j)=v_j,\,j<i
\end{array}
$$

$$ 
\sembr{C^k(v_1, \ldots, v_k)}^{{\mathfrak s}_m}_\theta = ({\mathfrak s}_m \equiv\; \uparrow C^k({\mathfrak s}^\prime_1, \ldots, {\mathfrak s}^\prime_k)),\,\mbox{if}\;\theta({\mathfrak s}^\prime_j)=v_j
$$

$$ 
\sembr{C^k(v_1, \ldots, v_k)}^{{\mathfrak s}_m}_\theta = ({\mathfrak s}_m \equiv\; {\mathfrak s}^\prime),\;\mbox{if}\;\theta({\mathfrak s}^\prime)=C^k(v_1, \ldots, v_k)
$$

$$
\begin{array}{rcl}
\sembr{\lstinline|match $\;e\;$ with \{$C^{n_i}_i$($y^i_1$, ..., $y^i_{n_i}$) -> $\;e_i$\}|}^{{\mathfrak s}_m}_\theta&=&\sembr{e}_\theta\;{\mathfrak s}^\prime\;\wedge\;\bigvee_i\\
& &(\lstinline|fresh ($s^i_1 \ldots s^i_{n_i}$)| \\
& &\qquad({\mathfrak s}^\prime \equiv \;\uparrow C_i^{n_i}(s^i_1, \ldots, s^i_{n_i})) \\
& &\qquad(\lambda y^i_1. \ldots \lambda  y^i_{n_i}. \sembr{e_i}_\theta) \; (\equiv s^i_1) \ldots (\equiv s^i_{n_i})\;{\mathfrak s}_m)
\end{array}
$$

$$
\begin{array}{rcl}
\sembr{\lstinline|match $\;v\;$ with \{$C^{n_i}_i$($y^i_1$, ..., $y^i_{n_i}$) -> $\;e_i$\}|}^{{\mathfrak s}_m}_\theta&=&\bigvee_i\\
& &(\lstinline|fresh ($s^i_1 \ldots s^i_{n_i}$)| \\
& &\qquad({\mathfrak s}^\prime \equiv \;\uparrow C_i^{n_i}(s^i_1, \ldots, s^i_{n_i})) \\
& &\qquad(\lambda y^i_1. \ldots \lambda  y^i_{n_i}. \sembr{e_i}_\theta) \; (\equiv s^i_1) \ldots (\equiv s^i_{n_i})\;{\mathfrak s}_m),\,\mbox{if}\;\theta({\mathfrak s}^\prime)=v
\end{array}
$$

Here all ${\mathfrak s}^\prime$ and ${\mathfrak s}^\prime_i$ designate arbitrary semantic variables, $e$~--- arbitrary term, $v$~--- arbitrary value w.r.t. the
original semantics. We also claim, that $\theta$ is undefined for all mentioned semantic variables, unless the opposite is specified explicitly.
\end{definition}

\begin{definition}
\normalfont Let 
\begin{itemize}
\item \mbox{$\left<\mathcal S,\,e\right>$}~--- a state w.r.t. the deferred semantics;
\item \mbox{$\left<\Sigma, \hat{\mathcal S}, \hat{e}, (\theta, \emptyset)\right>$}~--- a state w.r.t. the
relational semantics.
\end{itemize} 

We say, that these states are connected, if there exists a semantic variable $q_m$, such, that:\vspace{1mm}

\begin{enumerate}
\item \mbox{$\hat{\mathcal S}\in\sembr{\mathcal S}^{e,{\mathfrak s}_m}_\theta$}\vspace{1mm}
\item \mbox{$\hat{e}\in\left\{
                          \begin{array}{lcl}
                            \sembr{e}^{{\mathfrak s}_m}_\theta&,&e\mbox{~--- ground and }\mathcal S\mbox{ does not contain functional contexts}\\[1mm]
                            \sembr{e}_\theta&,&\mbox{otherwise}
                          \end{array}
                       \right.
            $} 
\item $\Sigma$ contains all semantic variables from $\hat{e}$, $\hat{\mathcal S}$, and $\theta$.
\end{enumerate}

\end{definition}

\begin{lemma}
\label{constructor}
\normalfont Let $v=\lstinline|C$^k$($v_1$,...,$v_k$)|$ be a value. Then
for arbitrary $\Sigma$, $\mathcal S$, $\theta$, $\hat{v}\in \sembr{v}_\theta$, and 
semantic variable ${\mathfrak s}$, such, that ${\mathfrak s}\not\in dom(\theta)$ either

$$
\left<\Sigma,\,\mathcal S, (\hat{v}\,{\mathfrak s}),\, (\theta,\,\emptyset)\right>\leadsto^*\left<\Sigma^\prime,\,\mathcal S,\,{\mathfrak s}\equiv\lstinline|C$^k$(${\mathfrak s}^\prime_1$,...,${\mathfrak s}^\prime_k$)|,\,(\theta^\prime,\,\emptyset)\right>\;\mbox{and}\;\theta^\prime({\mathfrak s}^\prime_i)=v_i
$$

or

$$
\left<\Sigma,\,\mathcal S, (\hat{v}\,{\mathfrak s}),\, (\theta,\,\emptyset)\right>\leadsto\left<\Sigma,\,\mathcal S,\,{\mathfrak s}\equiv {\mathfrak s}^\prime,\,(\theta,\,\emptyset)\right>\;\mbox{and}\;\theta({\mathfrak s}^\prime)=v
$$
 
The proof is by induction on the height of $v$.
\end{lemma}

\begin{lemma}
\label{evaluation_lemma}
\normalfont Let $s=\left<\mathcal S=g_m\dots g_1,\,e\right>$ be a state w.r.t. the deferred semantics, 
$g_i$~--- ground contexts, $e$~--- expression of a ground type, $\theta$~--- some substitution,
${\mathfrak s}_m$~--- some semantic variable, \mbox{$\hat{\mathcal{S}}\in\sembr{\mathcal S}^{e,\,{\mathfrak s}_m}_\theta$}, 
\mbox{$\hat{e} \in \sembr{e}_\theta$}. Then there is a sequence of steps w.r.t. the relational
semantics, such, that

$$
\left<\Sigma, \hat{\mathcal S}, (\hat{e} \, {\mathfrak s}_m), (\theta,\,\emptyset) \right>\leadsto^*\hat{s}
$$

\noindent and $s$ and $\hat{s}$ are connected. Here we assume $\Sigma$ to contain all semantic variables from
$\hat{\mathcal S}$ and $\theta$. The proof is by case analysis on $e$, using Lemma~\ref{constructor}.
\end{lemma}

\begin{lemma} 
\label{connection}
\normalfont Let \mbox{$s_1 \to s_2$}~--- a single evaluation step w.r.t. the deferred semantics,
$\hat{s_1}$~--- a state of the relational semantics, such, that $s_1$ and $\hat{s_1}$ are connected. Then
there exists a sequence of steps in the relational semantics \mbox{$\hat{s_1}\leadsto^*\hat{s_2}$}, such, 
that $s_2$ and $\hat{s_2}$ are connected. The proof is by case analysis and definition of connection
relation, using Lemmas~\ref{substitution},~\ref{constructor},~\ref{evaluation_lemma}. 
\end{lemma}

\begin{lemma}
\label{prefix}
\normalfont Let $s_0=\left<\emptyset,\,\epsilon,\,\lstinline|fresh ($x$) $(\sembr{e}^c\;x)$|,\,\iota\right>$ be an
initial state of evaluation w.r.t. the relational semantics. Then there is a sequence of steps
\mbox{$s_0\leadsto^*\hat{s}$}, such, that \mbox{$\left<\epsilon,\,e\right>$} (an initial state of
evaluation of $e$ w.r.t. the deferred semantics) and $\hat{s}$ are connected. Immediately follows from
Lemma~\ref{evaluation_lemma}.
\end{lemma}

Now we can prove the partial correctness theorem. Let us have a term $e$ of a ground type in the source language, which
reduces to a value $v=\lstinline|C$^k$($v_1$,...,$v_k$)|$ w.r.t. the original call-by-value semantics. Then it reduces to the same value w.r.t. the
deferred semantics: 

$$
\left<\epsilon,\,e\right>\to^*\left<\epsilon,\,v\right>
$$

By Lemma~\ref{prefix} 

$$
\left<\emptyset,\,\epsilon,\lstinline|fresh ($x$) $(\sembr{e}^c\;x)$|,\iota\right>\leadsto^*\hat{s}
$$

\noindent where \mbox{$\left<\epsilon,\,e\right>$} and $\hat{s}$ are connected. By Lemma~\ref{connection}, there is
a state $\hat{s^\prime}$ w.r.t. the relational semantics, such, that

$$
\hat{s}\leadsto^*\hat{s^\prime}
$$

\noindent and \mbox{$\left<\epsilon,\,v\right>$} and $\hat{s^\prime}$ are connected. By the definition of
the connection relation, $\hat{s^\prime}$ has one of the following forms:

$$
\left<\Sigma,\,\epsilon,\,{\mathfrak s}_0\equiv\lstinline|C$^k$(${\mathfrak s}^\prime_1$,...,${\mathfrak s}^\prime_k$)|,\,(\theta,\,\emptyset)\right>,\,\theta({\mathfrak s}^\prime_i)=v_i
$$

\noindent or

$$
\left<\Sigma,\,\epsilon,\,{\mathfrak s}_0\equiv {\mathfrak s}^\prime,\,(\theta,\,\emptyset)\right>,\,\theta({\mathfrak s}^\prime)=v
$$

\noindent where ${\mathfrak s}_0$ is the first semantic variable, added to $\Sigma$, and \mbox{${\mathfrak s}_0\not\in dom(\theta)$}. In
both cases, we can make the one last step in the relational semantics, which completes the proof. 


\end{document}

